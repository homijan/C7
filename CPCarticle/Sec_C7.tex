\subsection{C7}
\label{sec:C7code}
In order to eliminate the~dimensions of the~above transport problem 
the~first-two-moment model based on approximation 
\begin{equation}
  f = \frac{\fzero}{4\pi} + \frac{3}{4\pi}\vn\cdot\fone , 
  \nonumber \label{eq:OOE_P1approximation}
\end{equation}
can be adopted and reads
\begin{eqnarray}
  \vmag\frac{\nue}{2}\pdv{}{\vmag}\left(\fzero - 4\pi\fM \right) &=&
  \vmag\nabla\cdot\fone + \tE\cdot
  \pdv{\fone}{\vmag} + \frac{2}{\vmag}\tE\cdot\fone , 
  \nonumber \label{eq:OOE_P1f0}\\
  \vmag\frac{\nue}{2}\pdv{\fone}{\vmag}
  - \left(\nuei + \frac{\nue}{2}\right)\fone &=& 
  \frac{\vmag}{3}\nabla\fzero + 
  \frac{\tE}{3}\pdv{\fzero}{\vmag} ,
  \nonumber \label{eq:OOE_P1f1}
\end{eqnarray}
\begin{equation}
  %\frac{\vect{j}}{\qe} = 
  \vect{q}_c \equiv
  \qe \intv \left(\frac{\nue\vmag^2}{\nuei + \frac{\nue}{2}}\pdv{\fone}{\vmag}
  - \frac{\vmag^2}{3\left(\nuei + \frac{\nue}{2}\right)}
  \nabla\fzero - 
  \frac{\vmag}{3\left(\nuei + \frac{\nue}{2}\right)}\pdv{\fzero}{\vmag}\tE
  \right) \vmag^2\, \dI\vmag = 0
  , \nonumber
\end{equation}
\subsubsection{Nonlocal electric field treatment}
\label{sec:Efield}

%\begin{eqnarray}
%  \pdv{\fzero}{\vmag} &=&
%  \frac{2}{\nue}\pdv{\fonez}{z} + \frac{2\tEz}{\vmag\nue} \pdv{\fonez}{\vmag} 
%  + \frac{4}{\vmag^2\nue}\tEz\fonez 
%  + 4\pi\pdv{\fM}{\vmag}
%  , \nonumber\\
%  \vmag\frac{\nue}{2}\pdv{\fonez}{\vmag} 
%  &=&  
%  \frac{\tEz}{3}\pdv{\fzero}{\vmag} + \frac{\vmag}{3}\pdv{\fzero}{z} 
%  + \left(\nuei + \frac{\nue}{2}\right)\fonez
%  , \nonumber \label{eq:OOE_P1f1}
%\end{eqnarray}

\begin{multline}
  %\frac{2}{3\vmag\nue} 
  %\left(\left(\sqrt{3}\vmag\frac{\nue}{2}\right)^2 - \tEz^2\right)  
  \left(\vmag\frac{\nue}{2} - \frac{2\tEz^2}{3\vmag\nue}\right) 
  \pdv{\fonez}{\vmag} 
  =\\
  \frac{2\tEz}{3\nue}\pdv{\fonez}{z}  
  + \frac{4\pi\tEz}{3}\pdv{\fM}{\vmag}
  + \frac{\vmag}{3}\pdv{\fzero}{z} 
  + \left(\frac{4\tEz^2}{3\vmag^2\nue}
  + \left(\nuei + \frac{\nue}{2}\right) \right)\fonez
  , \nonumber
\end{multline}

%\begin{multline}
%  \frac{2}{3\vmag\nue} 
%  \left(\left(\sqrt{3}\vmag\frac{\nue}{2}\right)^2 - \tEz^2\right)
%  \frac{\fonez^{n+1} - \fonez^n}{\Delta\vmag} 
%  =\\
%  \frac{2\tEz}{3\nue}\pdv{\fonez^{n+1}}{z}  
%  + \frac{4\pi\tEz}{3}\pdv{\fM^{n+1}}{\vmag}
%  + \frac{\vmag}{3}\pdv{\fzero^{n+1}}{z} 
%  + \left(\frac{4\tEz^2}{3\vmag^2\nue}
%  + \left(\nuei + \frac{\nue}{2}\right) \right)\fonez^{n+1}
%  , \nonumber
%\end{multline}

\begin{table}
\begin{center}
  \begin{tabular}{c|ccccc}
    \hline\hline\\
    Kn & $\,\,10^{-3}\,\,$ & $\,\,5\times10^{-3}\,\,$ & $\,\,10^{-2}\,\,$ & $\,\,5\times10^{-2}\,\,$ & $\,\,10^{-1}\,\,$ \\\\
    \hline\\
    $\vmag_{lim}^{Z=2} / \vth$ & 14.8 & 6.8 & 5.0 & 2.8 & 2.6 \\\\
    \hline\\
    $\vmag_{lim}^{Z=10} / \vth$ & 6.7 & 3.4 & 2.6 & 1.6 & 1.3 \\\\
    \hline\hline
  \end{tabular}
  \caption{
  $\sqrt{3}\vmag\frac{\nue}{2} > \tEz$.
  }
\end{center}
\label{tab:vlim}
\end{table}

\begin{comment} % too complicated
\begin{equation}
  |\tE_{red}| = (1 + \alpha^E) \vmag \frac{\nue}{2} ,
  \label{eq:reducedEfield}
\end{equation}

\begin{eqnarray}
  |\tE_{red}| &=& |\tE_{d}| + |\tE_{iso}| ,
  \nonumber \\
  \vmag \frac{\nue}{2} + |\tE_{iso}| &=& |\tE_{d}| ,
  \nonumber 
\end{eqnarray}
where we define the~isotropic effect of E field as
$|\vect{E}_{iso}| = \vmag \nue^E$ by introducing the~effective collisional 
frequency $\nue^E$.

Since the~effect of the~original E field $\tE$ has been reduced in 
\eqref{eq:reducedEfield}, an~additional collision term
\begin{equation}
  \vmag \nuei^E = |\tE| - |\tE_{red}| ,
  \nonumber
\end{equation}
is added to scattering on ions. The~improved collision AWBS operator then takes 
the~following form
\begin{equation}
  C_{AWBS} = \vmag \left(\frac{1}{2} + \alpha^E\right)\nue \pdv{}{\vmag}
  \left(f - \fM \right) 
  + \left(\nuei + \nuei^E + \frac{\nue}{2}\right) (\fzero - f) ,
  \nonumber
\end{equation}
where both $\alpha^E$ and $\nuei^E$ apply only if 
$|\tE| > \sqrt{3} \vmag \frac{\nue}{2}$ 
and are set to zero otherwise.
\end{comment} % too comlicated

\begin{comment} % directional E field condition
Since in the~laser heated plasmas the~Knudsen number 
Kn$ = \frac{\vth}{\nu_t(\vth) L} \in (0, 1)$, i.e. the~collisionality in 
the~kinetics of electrons plays always an~important effect for thermal-like 
particles, the~electron distribution 
function can be treated as out-of-equilibrium approximation 
\begin{equation}
  f = \fM + \daf ,
  \label{eq:OOE_outofeq}
\end{equation}  
where the~consequent AWBS model reads
\begin{multline}
  \vmag\vn\cdot\nabla (\fM + \daf) + \tE\cdot\vn \left(\pdv{\fM}{\vmag} 
  + \pdv{\daf}{\vmag}\right) 
  + \frac{\tE\cdot\vect{e}_\theta}{\vmag}\pdv{\daf}{\theta}
  = \\
  \vmag \frac{\nue}{2} \pdv{\daf}{\vmag} 
  + \left(\nuei + \frac{\nue}{2}\right) (\fzero - (\fM + \daf)) ,
  \label{eq:OOE_AWBS_model}
\end{multline}
or its 1D version
\begin{multline}
  \vmag\mu\pdv{}{z}(\fM + \daf) + \tEz\mu\left(\pdv{\fM}{\vmag} 
  + \pdv{\daf}{\vmag}\right) 
  + \frac{\tEz(1-\mu^2)}{\vmag}\pdv{\daf}{\mu}
  = \\
  \vmag \frac{\nue}{2} \pdv{\daf}{\vmag} 
  + \left(\nuei + \frac{\nue}{2}\right) (\fzero - (\fM + \daf)) ,
  \label{eq:OOE_AWBS_model_1D}
\end{multline}
where $\tE\cdot\vect{e}_\theta = \tEz\sin(\theta)$ and $\pdv{}{\theta} = \sin(\theta)\pdv{}{\mu}$, $\mu = \cos(\theta)$.
\begin{multline}
  \left(\vmag \frac{\nue}{2} - \tEz\mu\right) \pdv{\daf}{\vmag} = 
  \vmag\mu\pdv{\daf}{z} + \vmag\mu\pdv{\fM}{z} + \tEz\mu\pdv{\fM}{\vmag} \\ 
  + \frac{\tEz(1-\mu^2)}{\vmag}\pdv{\daf}{\mu}
  - \left(\nuei + \frac{\nue}{2}\right) (\fzero - (\fM + \daf))
  ,
  \nonumber
\end{multline}
we adopt $\daf(\vmag, \mu) = \dafzero(\vmag) + \mu\dafone(\vmag)$, 
which leads to
\begin{multline}
  \left(\vmag \frac{\nue}{2} - \tEz\mu\right) \pdv{}{\vmag}
  (\dafzero + \mu\dafone) = 
  \vmag\mu\pdv{}{z}(\dafzero + \mu\dafone) + \vmag\mu\pdv{\fM}{z} 
  + \tEz\mu\pdv{\fM}{\vmag} \\ 
  + \frac{\tEz(1-\mu^2)}{\vmag}\dafone
  + \left(\nuei + \frac{\nue}{2}\right) \mu\dafone
  ,
  \nonumber
\end{multline}
\begin{eqnarray}
  \vmag \frac{\nue}{2}\pdv{\dafzero}{\vmag}
  - \tEz\mu^2 \pdv{\dafone}{\vmag}
  &=& 
  \vmag\mu^2\pdv{\dafone}{z}
  + \frac{\tEz(1-\mu^2)}{\vmag}\dafone
  , \nonumber \\
  \mu\vmag \frac{\nue}{2}\pdv{\dafone}{\vmag} - \tEz\mu \pdv{\dafzero}{\vmag}
  &=& 
  \vmag\mu\pdv{\dafzero}{z}
  + \vmag\mu\pdv{\fM}{z} + \tEz\mu\pdv{\fM}{\vmag}  
  + \left(\nuei + \frac{\nue}{2}\right) \mu\dafone
  ,
  \nonumber
\end{eqnarray}
\end{comment} % directional E field condition

\begin{eqnarray}
  |\tE_{red}| &=& \vmag\frac{\nue}{2} ,
  \nonumber \\
  \nuei^E &=& \frac{|\tE| - |\tE_{red}|}{\vmag} ,
  \nonumber
\end{eqnarray}
where $\omega_{red} = |\tE_{red}| / |\tE|$.

P1 approximation equivalent
\begin{equation}
  f = \frac{4\pi \fM + \dafzero}{4\pi} + \frac{3}{4\pi}\vn\cdot\fone .
  \label{eq:OOE_P1outofeq}
\end{equation}
where the~moment model reads
\begin{eqnarray}
  \vmag \frac{\nue}{2}\pdv{\dafzero}{\vmag} &=&
  \vmag\nabla\cdot\fone + \tE\cdot\left(\omega_{red} \pdv{\fone}{\vmag} 
  + \frac{2}{\vmag}\fone\right) , 
  \nonumber\\
  \vmag\frac{\nue}{2}\pdv{\fone}{\vmag} 
  &=& \left(\nuei + \nuei^E + \frac{\nue}{2}\right)\fone 
  + \frac{\vmag}{3}\nabla\left(4\pi\fM + \dafzero\right)
  \nonumber \\
  && 
  + \frac{\tE}{3}\left(4\pi \pdv{\fM}{\vmag} + \omega_{red} \pdv{\dafzero}{\vmag} 
  \right) ,
  \nonumber
\end{eqnarray}

\begin{equation}
  \tE =
  \frac{\intv \left(\frac{\frac{\nue}{2}\vmag^2}
  {\nuei + \nuei^E + \frac{\nue}{2}}\pdv{\fone}{\vmag}
  - \frac{\vmag^2}{3\left(\nuei + \nuei^E + \frac{\nue}{2}\right)}
  \nabla\left(4\pi\fM + \dafzero\right)\right) \vmag^2\, \dI\vmag}
  {\intv \frac{\vmag}
  {3\left(\nuei + \nuei^E + \frac{\nue}{2}\right)}
  \left(4\pi\pdv{\fM}{\vmag} + \omega_{red} \pdv{\dafzero}{\vmag}\right)
  \vmag^2\, \dI\vmag} ,
  \nonumber
\end{equation}

