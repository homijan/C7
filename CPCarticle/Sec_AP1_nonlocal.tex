\section{The AWBS nonlocal transport model of electrons}
\label{sec:AWBSnonlocal}

In order to define a~nonlocal transport model of electrons, 
we use the~AWBS collision operator and the~P1 angular 
discretization of the~electron distribution function
\begin{equation}
  \tilde{\ft}(\vect{x}, \vn, \vmag) = 
  \fzero(\vect{x}, \vmag) + \vn\cdot\fone(\vect{x}, \vmag) , 
  \label{eq:P1approximation}
\end{equation}
consisting of the~isotropic part represented by the zeroth angular moment 
$\fzero = \frac{1}{4\pi}\int_{4\pi} \tilde{\ft} \dI\vn$ 
and the~directional part represented by the~first angular moment 
$\fone = \frac{3}{4\pi}\int_{4\pi} \vn
\tilde{\ft} \dI\vn$, where $\vn$ is the~transport direction (the~solid angle).
%It should be noticed, that \eqref{eq:P1approximation} represents 
%a~multi-dimensional equivalent to \eqref{eq:f_approximation}, where 
%the~following relations between the~spherical harmonics method
%and the~moments method hold $\ft^0 = \frac{\fzero}{4\pi}$ and 
%$\ft^1 = \frac{3}{4\pi}|\fone|$.
Then, the~first two angular moments
\cite{Shkarofsky_Particle_Kinetics_book_1966_24} applied to the~steady form of 
\eqref{eq:kinetic_equation} with collision operator \eqref{eq:AWBS_model} 
(extended by \eqref{eq:qAWBS_approximation}) lead to the~model equations
\begin{eqnarray}
  \vmag\frac{\nue}{2}\pdv{}{\vmag}\left(\fzero - \fM \right) &=&
  \frac{\vmag}{3}\nabla\cdot\fone + \frac{\qe}{\me}\frac{\E}{3}\cdot\left(
  \pdv{\fone}{\vmag} + \frac{2}{\vmag}\fone\right) , 
  \nonumber \\
  \label{eq:AP1f0}\\
  \vmag\frac{\nue}{2}\pdv{\fone}{\vmag}
  - \nuscat\fone &=& 
  \vmag\nabla\fzero + 
  \frac{\qe}{\me}\E\pdv{\fzero}{\vmag} 
  +\frac{\qe\B}{\me c}\vect{\times} \fone
  ,
  \nonumber \\
  \label{eq:AP1f1}
\end{eqnarray}
where $\nuscat = \nuei + \frac{\nue}{2}$. The system of equations 
\eqref{eq:AP1f0} and \eqref{eq:AP1f1} is called the~{\bf AP1 model} 
(AWBS + P1).  

On the~one hand side AP1 model gives us with the~information about 
the~electron distribution function, on the~other hand side 
a~macroscopic interpretation of the~microscopic EDF properties are of great
importance and provide the~bridge between kinetic and fluid description of 
plasma. For example the~\textit{flux} quantities as 
electric current and heat flux due to the~motion of electrons
\begin{equation}
  \vect{j} = \frac{4\pi}{3}\qe \int \vmag \fone~\dI\tilde{\vmag},~~ 
  \vect{q}_h = \frac{4\pi}{3}\frac{\me}{2} \int \vmag^3 \fone~\dI\tilde{\vmag},
  \nonumber
\end{equation}
where $\dI\tilde{\vmag} = \vmag^2\dI\vmag$ 
the~spherical coordinates metric,
are based on corresponding velocity moments (integrals) of the~first angular 
moment of EDF. Consequently, the~explicit formula for the~first angular moment
from \eqref{eq:AP1f1} (the~inversion inspired by \cite{Shkarofsky_1979_73})
proves to be extremely useful
\begin{equation}
  \fone = \frac{\nuscat^2 \vect{F}^* + \omegaB~\omegaB\cdot\vect{F}^* 
  - \nuscat~\omegaB \vect{\times} \vect{F}^*}{\nuscat (\omegaB^2 + \nuscat^2)}
  ,
  \label{eq:f1_explicit}
\end{equation} 
because it provides a~valuable 
information about the~dependence of macroscopic \textit{flux} quantities on
electric and magnetic fields in plasma, 
where $\omegaB = \frac{\qe\B}{\me c}$ is the~electron gyro-frequency and 
$\vect{F}^* = \vmag\frac{\nue}{2}\pdv{\fone}{\vmag} - \vmag\nabla\fzero 
 - \frac{\qe}{\me}\E\pdv{\fzero}{\vmag}$.

\subsection{Nonlocal Ohm's Law}
\label{sec:Efield}
The~expression \eqref{eq:f1_explicit} becomes extremely useful when used
to describe the~electron fluid momentum, i.e. the~current velocity moment
%\begin{equation}
%  \vect{j}(f, \vect{E}) = e \int \vmag \fone \vmag^2~\dI \vmag =
%  e \int \vmag \frac{\E^*}
%  {\nuei} \vmag^2~\dI \vmag ,
%  \label{eq:NonlocalOhm}
%\end{equation}
\begin{multline}
  \vect{j}_{(f, \E, \B)} = \qe \int \vmag \fone \vmag^2~\dI \vmag = \\
  - \frac{\qe^2}{\me} \int \vmag \frac{\nuei^2 \E^* 
  + \omegaB~\omegaB\cdot\E^* - \nuei~\omegaB \vect{\times} \E^*}
  {\nuei (\omegaB^2 + \nuei^2)}~\dI \tilde{\vmag} ,
  \nonumber %\label{eq:NonlocalOhm}
\end{multline}
where $\E^* = \E~\pdv{\fzero}{\vmag} + \frac{\me}{\qe}\vmag \nabla \fzero$ 
is the~effective electric field in plasma. 
This can be written as
\begin{equation}
  \vect{j}_{(f, \E, \B)} = \Iohm\pdv{\fzero}{\vmag}\E 
  + \frac{\me}{\qe} \Iohm\vmag \nabla \fzero
  ,
  \label{eq:NonlocalOhm}
\end{equation}
where we used the~following notation 
$\Iohm\vect{g} = - \frac{\qe^2}{\me} \int \vmag \frac{\nuei^2 \vect{g} 
  + \omegaB~\omegaB\cdot\vect{g} - \nuei~\omegaB \vect{\times} \vect{g}}
  {\nuei (\omegaB^2 + \nuei^2)}~\dI \tilde{\vmag}$ showing how $\Iohm$
acts on a~general vector field $\vect{g}$.
We refer to 
\eqref{eq:NonlocalOhm} as to the~{\bf nonlocal Ohm's law}.
Its relation and a~proper local asymptotic to the~standard Ohm's law  
can be found when $\fzero\rightarrow\fM$ and weak magnetization 
($\omegaB \ll \nuei$) is considered. Then \eqref{eq:NonlocalOhm} simplifies to
\begin{multline}
  \vect{j} =- \frac{\qe^2}{\me} \int \frac{\vmag^3}{\nuei}
  \left( \E~\pdv{\fM}{\vmag} + \frac{\me}{\qe}\vmag \nabla \fM \right)
  ~\dI \vmag =\\
  \frac{16\sqrt{\frac{2}{\pi}}\qe^2 \kB^\frac{3}{2} \Te^\frac{3}{2}}{\me^\frac{5}{2} \Gamma\Zbar}
  \left[\E - \frac{\frac{5}{2} \ed \kB \nabla \Te 
  + \nabla \ed \kB \Te}{\qe \ed}  \right] 
  ,
  \label{eq:AsymptoticOhm}
\end{multline}
which can be directly compared to the~local fluid theory 
%represented by the~\textit{generalized Ohm's law} 
\begin{equation}
  \E = \matr{\sigma}(\fzero)^{-1} \vect{j} 
  - \frac{\nabla P (\fzero)}{\qe \ed}
  \xrightarrow{\fzero \rightarrow \fM}
  \E_{l} = \frac{\vect{j}}{\sigma_{l}}
  + \frac{\nabla p_e - \vect{R}_{\Te}}{\qe \ed} 
  ,
  \label{eq:GeneralOhm} 
\end{equation}
while addressing properly the~local electric field $\E_{l}$ given by
the~pressure $p_e = \ed \kB \Te$,
the~thermal force $\vect{R}_{\Te} = - \frac{3}{2} \ed \kB \nabla \Te$ 
and the~local electrical conductivity 
$\sigma_{l} = 16\sqrt{\frac{2}{\pi}}\qe^2 \kB^\frac{3}{2} \Te^\frac{3}{2}
/ \me^\frac{5}{2} \Gamma\Zbar$ \cite{Braginskii_1965_3}.
In \eqref{eq:GeneralOhm} we defined the~nonlocal electrical tensor conductivity 
\begin{equation}
  \matr{\sigma} = \Iohm\pdv{\fzero}{\vmag}
  ,
  \label{eq:NonlocalSigma}
\end{equation}
and the~nonlocal microscopic force
\begin{equation}
  \nabla P = \matr{\sigma}^{-1}\me\ed \Iohm\vmag \nabla \fzero
  ,
  \label{eq:NonlocalGradP}
\end{equation}
based on \eqref{eq:NonlocalOhm}.

The~local dependence of the~AP1 current \eqref{eq:AsymptoticOhm} 
on electric field and gradients of $\ed$ and $\Te$ clearly demonstrates, 
that \eqref{eq:GeneralOhm} is a~local version of \eqref{eq:NonlocalOhm}.
This also implies that \eqref{eq:NonlocalOhm} provides a~very important 
physics related to the~magnetic field source in terms of nonlocal 
Biermann battery, since the~curl on the~electric field 
\eqref{eq:GeneralOhm} gives
\begin{equation}
  \nabla\vect{\times} \frac{\nabla P}{\qe \ed} 
  \xrightarrow{\fzero \rightarrow \fM} 
  \nabla\vect{\times} \frac{\nabla p_e - \vect{R}_{\Te}}{\qe \ed} =
  \frac{\kB}{\qe\ed}\nabla \Te \vect{\times}\nabla \ed
  .
  \label{eq:NonlocalBiermann}
\end{equation}

A~local version of the~{\bf nonlocal Ohm's law} \eqref{eq:NonlocalOhm} 
compared to the~\textit{generalized Ohm's law} \eqref{eq:GeneralOhm} when
a~magnetic field is applied would require a~much more delicate analysis and 
we leave it as a~future complementary work.

It should be noted, that $\nue$-related terms in \eqref{eq:f1_explicit} 
have been omitted in \eqref{eq:NonlocalOhm}, 
since the~e-e collisions do not contribute 
(cancel out when integrated over velocity) to the~momentum
change, i.e. 
$\int \vmag \left( \vmag\frac{\nue}{2}\pdv{\fone}{\vmag}
  - \frac{\nue}{2}\fone\right) \dI\tilde{\vmag} = 0$.

\begin{comment} % Ohm's law with B.
\begin{eqnarray} 
  \E &=&  
  \frac{\nabla p_e - \vect{R}_{\Te}}{\qe \ed}
  +~~~~~~~~ 
  \frac{\vect{j}}{\sigma} 
  ~~~~~~-~~~~~~ 
  \frac{\vect{j}\vect{\times}\B}{\qe\ed c} 
  ,
  \nonumber \\
  %\E 
  %&=& 
  %\frac{\int \vmag^2 \nabla \fzero \vmag^2\, \dI \vmag}
  %{\int \vmag \pdv{\fzero}{\vmag} \vmag^2\, \dI \vmag}
  %+  
  %\frac{\int \vmag \nuei\fone \vmag^2\, \dI \vmag}
  %{\int \vmag \pdv{\fzero}{\vmag} \vmag^2\, \dI \vmag} 
  %+ 
  %\frac{\int \vmag \fone\times\B \vmag^2\, \dI \vmag}
  %{\int \vmag \pdv{\fzero}{\vmag} \vmag^2\, \dI \vmag} 
  %.
  %\nonumber
  \frac{\qe}{\me}\E 
  &=& 
  \frac{\int \vmag^2 \nabla \fzero~\dI \tilde{\vmag}}
  {\int \vmag \pdv{\fzero}{\vmag}~\dI \tilde{\vmag}}
  +  
  \frac{\int \vmag \nuei\fone~\dI \tilde{\vmag}}
  {\int \vmag \pdv{\fzero}{\vmag}~\dI \tilde{\vmag}} 
  + 
  \frac{\qe\int \vmag \fone\vect{\times}\B~\dI \tilde{\vmag}}
  {\me c \int \vmag \pdv{\fzero}{\vmag}~\dI \tilde{\vmag}} 
  .
  \nonumber
\end{eqnarray}
\end{comment} % Ohm's law with B.

\subsection{AWBS Nonlocal Magneto-Hydrodynamics}
\label{sec:ANTH}
The~\textit{AWBS nonlocal~magneto-hydrodynamic~model} (Nonlocal-MHD)
refers to two~temperature single-fluid hydrodynamic model 
extended by a~kinetic model of electrons using the~AWBS transport equation,
which provides a~direct coupling between hydrodynamics and Maxwell equations.

Mass, momentum density, and total energy 
$\rho$, $\rho\vect{u}$, and 
$E = \frac{1}{2}\rho\vect{u}\cdot\vect{u} + 
 \rho \varepsilon_i + \rho \varepsilon_e$, 
where $\rho$ is 
the~density of plasma, $\vect{u}$ the~plasma fluid velocity, $\varepsilon_i$ 
the~specific internal ion energy density, 
and $\varepsilon_e$ the~specific internal 
electron energy density,
are modeled by the~Euler equations in Lagrangian frame 
\cite{Holec_DGBGKT_2016, Holec_PoPNTH2018}
\begin{eqnarray}
 \frac{\dI \rho}{\dI t} &=& - \rho\nabla\cdot\vect{u}
 , 
 \label{eq:NTH_rho}\\
 \rho\, \frac{\dI \vect{u}}{\dI t} &=& - \nabla (p_i + p_e) 
 + \vect{j}_{(f, \E, \B)} \vect{\times}\B
 ,  
 \label{eq:NTH_v}\\
 \rho~C_{V_i}\frac{\dI \Ti}{\dI t} 
 &=& 
 \left(\rho^2 C_{\Ti} - p_i\right)\nabla\cdot\vect{u} 
 - G(\Ti - \Te)
 ,  
 \label{eq:NTH_Ti}\\
 \rho~C_{V_e} \frac{\dI \Te}{\dI t}
  %+ \frac{\dI \epsilon_R}{\dI t} 
  &=& 
 \left(\rho^2 C_{\Te} - p_e \right) \nabla\cdot\vect{u}  
 + G(T_i - \Te)
 \nonumber\\ 
 && - \nabla\cdot\vect{q}_{h(f, \E, \B)} + Q_{\text{IB}} 
 , 
 \label{eq:NTH_Te}
\end{eqnarray}
where $\Ti$ is the~temperature of ions, $\Te$ the~temperature of electrons,
$p_i$ the~ion pressure, $p_e$ the~electron pressure,
$\vect{q}_h$ the~heat flux, $Q_{\text{IB}}$ the~inverse-bremsstrahlung laser 
absorption, and 
$G = \rho C_{V_e} \nuei$ is 
the~ion-electron energy exchange rate. 
The~thermodynamic closure terms 
$p_e$, $p_i$, 
$C_{V_i} = \frac{\partial \varepsilon_i}{\partial \Ti}$, 
$C_{\Ti} = \pdv{\varepsilon_i}{\rho}$,
$C_{V_e} = \frac{\partial \varepsilon_e}{\partial \Te}$, 
$C_{\Te} = \pdv{\varepsilon_e}{\rho}$
%$\frac{\partial \varepsilon_e}{\partial T_e}$,
%$\frac{\partial \varepsilon_i}{\partial T_i}$,
%$\frac{\partial \varepsilon_e}{\partial \rho}$,
%$\frac{\partial \varepsilon_i}{\partial \rho}$ 
are obtained from an~equation of state (EOS), e.g.
the~SESAME equation of state tables
\cite{T4_SESAME_83, Lyon_SESAME_EOS_database-TechRep-92}.

The~magnetic and electric fields are modeled by Maxwell equations
\begin{eqnarray}
  \frac{1}{c}\frac{\dI \B}{\dI t} &=& - \nabla\vect{\times}\E
  ,
  \label{eq:Faraday} \\
  \frac{1}{c}\frac{\dI \E}{\dI t} &=& \nabla\vect{\times}\B - \frac{4\pi}{c}
  \vect{j}_{(f, \E, \B)}
  ,
  \label{eq:Ampere}
  %\frac{4\pi}{c}
  %\vect{j}(f, \vect{E}) &=& \nabla\vect{\times}\vect{B} ,
\end{eqnarray}
where the~initial state of $\B$ and $\E$ obeys the~Gauss's law.

We have explicitly written the~current and heat flux as dependent on
electron kinetics, represented by the~electron distribution function $f$,
and electric and magnetic fields. In principal, $\vect{j}_{(f, \E, \B)}$
and $\vect{q}_{h(f, \E, \B)}$ can be referred to as the~\textit{kinetic closure}
and is provided by the~{\bf AP1 model} \eqref{eq:AP1f0} and \eqref{eq:AP1f1},
where $\fM$ is given on the~spatial profile of $\Te$ 
governed by \eqref{eq:NTH_Te}.
All quantities are defined in the~fluid frame in the aforementioned 
Nonlocal-MHD model.
%especially that 
%$\nabla\vect{\times} \nabla \fzero \sim \nabla\vect{\times} \nabla p_e \sim \nabla \ed \vect{\times} \nabla \Te$

\subsection{Numerical Implementation of the AWBS Electron Kinetics}
\label{sec:Numerics}

It is well known, that as the~electron transport exhibits a~quasi-steady
behavior, the~same holds for the~electric field in \eqref{eq:Ampere} on 
the~time scale of the~fluid. Consequently, the~Ampere's law \eqref{eq:Ampere} 
usually takes a~quasi-steady form 
$\frac{4\pi}{c} \vect{j}_{(f, \E, \B)} = \nabla\vect{\times}\B$
when used in hydrodynamics. Proceeding further, one can make use of 
the~{\bf nonlocal Ohm's law} \eqref{eq:NonlocalOhm} to write 
a~\textit{fully kinetic form of the~Ampere's law} 
governing the~electric field $\E$
\begin{equation}
  \Iohm\pdv{\fzero}{\vmag}\E 
  + \frac{\me}{\qe} \Iohm\vmag \nabla \fzero = 
  \frac{c}{4\pi} \nabla\vect{\times}\B 
  .
  \label{eq:AmpereKinetic}
\end{equation}

In order to solve the~kinetics of electrons, we adopt a~high-order 
finite element discretization 
\cite{Dobrev_Kolev_Rieben-High-order_curvilinear_finite_element_methods_for_Lagrangian_hydrodynamics, mfem-library} 
of the~model equations \eqref{eq:AP1f0}, \eqref{eq:AP1f1}, 
\eqref{eq:AmpereKinetic}
\begin{eqnarray}
  \matr{M}^{L_2}_{_{(\frac{\vmag\nue}{2})}}\cdot\ddv{\vfzero}{\vmag} 
  - \matr{V}^{L_2}_{_{(\frac{\qe\E}{3\me})}}\cdot\ddv{\fone}{\vmag}
  &=&
  \matr{D}^{L_2}_{_{(\frac{\vmag}{3})}}\cdot\fone 
  + \matr{M}^{L_2}_{_{(\frac{2\qe\E}{3\me\vmag})}}\cdot\fone
  \nonumber \\ 
  &&+ \vect{b}^{L_2}_{_{(\frac{\vmag\nue}{2}\pdv{\fM}{\vmag})}}, 
  \label{eq:FEMAP1f0}
  \\
  \matr{M}^{H_1}_{_{(\frac{\vmag\nue}{2})}}\cdot\ddv{\fone}{\vmag}
  - \matr{V}^{H_1}_{_{(\frac{\qe\E}{\me})}}\cdot\ddv{\vfzero}{\vmag}
   &=& 
  \matr{G}^{H_1}_{_{(\vmag)}}\cdot\vfzero 
  + \matr{M}^{H_1}_{_{(\nuscat)}}\cdot\fone 
  \nonumber \\
  && + \matr{C}^{H_1}_{_{(\frac{\qe\B}{\me c}\vect{\times})}}\cdot\fone
  ,
  \label{eq:FEMAP1f1}\\
  \matr{J}^{ND}_{_{(\pdv{\fzero}{\vmag})}}\cdot\E 
  &=& 
  \matr{JG}^{ND}_{_{(\frac{\me\vmag}{\qe})}}\cdot\vfzero
  + \vect{b}^{ND}_{_{(\frac{c}{4\pi} \nabla\vect{\times}\B)}} 
  ,
  \nonumber \\
  \label{eq:FEMAmpereKinetic}
\end{eqnarray}
where the~continuous differential operators are represented by standard 
discrete analogs (matrices of bilinear forms) 
$\matr{M}, \matr{G}, \matr{D}, \matr{V}, \matr{C}$, i.e. mass, gradient, 
divergence, vector field dot product, and vector field curl, and
by $\matr{J}, \matr{JG}$ matrices specific to {\bf nonlocal Ohm's law} 
\eqref{eq:NonlocalOhm}. The~linear form $\vect{b}$ represents sources, i.e.
temperature $\Te$ via $\pdv{\fM}{\vmag}$ and the~curl of 
the~magnetic field $\B$. These finite element discrete analogs are defined
on piece-wise continuous $L_2$ finite element space (domain of $\vfzero$),
continuous $H_1$ finite element space (domain of $\fone$) 
\cite{Dobrev_Kolev_Rieben-High-order_curvilinear_finite_element_methods_for_Lagrangian_hydrodynamics}, 
and Nedelec finite element space (domain of $\E$). We do not show their
definitions since it is out of the~scope of this article. 

The~strategy of solving 
\eqref{eq:FEMAP1f0} and \eqref{eq:FEMAP1f1} resides in integrating 
$\ddv{\vfzero}{\vmag}$
and $\ddv{\fone}{\vmag}$ along the~velocity magnitude. 
This is done by starting the~integration
from infinite velocity ($\vmag = 7 \vth^{max}$ is a~sufficiently high limit) 
to zero velocity using the~Implicit Runge-Kutta method. The~value
$\vth^{max}$ equals the~electron thermal velocity corresponding to the~maximum 
electron temperature in the~current profile of plasma.
It should be noted, that the~backward integration concept is crucial for 
the~model, since it corresponds to the~deceleration of electrons due to 
collisions \cite{Touati_2014}, which leads to some limitations described in 
\appref{app:AP1limit}. 

