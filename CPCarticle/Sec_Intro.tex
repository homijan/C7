\section{Introduction}
\label{sec:Intro}
%In dealing with the nonequilibrium properties of
%systems whose particles obey an inverse-square law
%of interaction, it is convenient to make use of the fact
%that under most circumstances small-angle collisions
%are much more important than collisions resulting in
%large momentum changes. This leads to the method
%often used for treating such systems, in which one
%expands the collision integrand of the Boltzmann change per
%equation in powers of the momentum collision.

The~first attempts of modern kinetic modeling of plasma can be tracked back 
to the~fifties, when Cohen, Spitzer, and Routly (CSR) \cite{CSR_1950} 
in detail demonstrated the~fact, that in the~ionized gas the~effect of 
Coulomb collisions between electrons and ions predominantly results 
from frequently occurring events of cumulative small deflections 
rather than occasional close encounters. This effect was originally described
in \cite{Jeans_BOOK1929} and Chandrasekhar \cite{Chandrasekhar_RMP1943} 
proposed to use the~diffusion equation model of the~Fokker-Planck type (FP) 
\cite{Planck_1917}.

%A more generally valid approach to the problem of
%treating changes in a distribution function resulting
%each of which from frequently occurring "events",
%produces a small change in the momentum of a particle,
%is to use the Fokker-Planck equation \cite{Planck_1917}, 
%which has been discussed by Spitzer and collaborators 
%\cite{SpitzerHarm_PR1953} . 
As a~result, a~classical paper by Spitzer and Harm (SH) 
\cite{SpitzerHarm_PR1953} provides the~computed electron distribution function
spanning from low to high Z plasmas, and more importantly, the~current and 
heat flux formulas, which are widely used in almost every plasma hydrodynamic
code nowadays.
%used the formalism of the FP equation to evaluate the collision terms of 
%the~Boltzmann equation under the assumptions that 
%(a) the events producing changes in particle momenta
%are two-body interactions described by the associated
%differential scattering cross sections, and 
%(b) that 
The~distribution function is of the form $f^0+\mu f^1$, where $f^0$ and $f^1$ 
are isotropic and $\mu$, is the direction cosine between the particle 
trajectory and some preferred direction in space. It should be emphasized that
the~SH solution expresses a~small perturbation of equilibrium, i.e. that 
$f^0$ is the~Maxwell-Boltzmann distribution and $\mu f^1$ represents 
a~very small deviation.

%It is the purpose of this paper to present the mechanics 
%of two-body collisions in a somewhat simplified
%form, and to derive the Fokker-Planck equation for an function. 
%From this general arbitrary distribution equation such special cases 
%as those of Chandrasekhar and Spitzer may easily be obtained. 
%For more complex situations the equation is suitable for integration by an
%electronic computer \cite{Rosenbluth_PR1957}.

The~actual cornerstone of the~modern FP simulations was set in place
by Rosenbluth \cite{Rosenbluth_PR1957}, when he derived a~simplified form 
of the~FP equation for a~finite expansion of the~distribution function,
where all the~terms are computed according to plasma conditions, including
$f^0$, which of course needs to tend to the~Maxwell-Boltzmann distribution.

In is the~purpose of this paper to present an~efficient alternative 
to FP model based on the~Albritton-Williams-Bernstein-Swartz 
collision operator (AWBS) \cite{AWBS_PRL1986}.
In Section \ref{sec:AWBSmodel} we propose a~modified form of 
the~AWBS collision operator, where its important properties are further
presented in Section \ref{sec:DiffusiveKinetics} with the~emphasis on its
comparison to the~full FP solution in local diffusive regime. 
Section \ref{sec:BenchmarkingAWBS} focuses on the~performance of the~AWBS 
transport equation model compared to modern kinetic codes including FP codes
Aladin and Impact, and PIC code Calder, where the~cases related to real
laser generated plasma conditions are studied. Finally, the~most important
outcomes of our research are concluded in Section \ref{sec:Conclusions}. 
