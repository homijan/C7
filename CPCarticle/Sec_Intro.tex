\section{Introduction}
\label{sec:Intro}
%In dealing with the nonequilibrium properties of
%systems whose particles obey an inverse-square law
%of interaction, it is convenient to make use of the fact
%that under most circumstances small-angle collisions
%are much more important than collisions resulting in
%large momentum changes. This leads to the method
%often used for treating such systems, in which one
%expands the collision integrand of the Boltzmann change per
%equation in powers of the momentum collision.

The~first attempts of modern kinetic modeling of plasma can be tracked back 
to the~fifties, when Cohen, Spitzer, and Routly (CSR) \cite{CSR_1950} 
in detail demonstrated the~fact, that in the~ionized gas the~effect of 
Coulomb collisions between electrons and ions predominantly results 
from frequently occurring events of cumulative small deflections 
rather than occasional close encounters. This effect was originally described
by Jeans in \cite{Jeans_BOOK1929} and 
Chandrasekhar \cite{Chandrasekhar_RMP1943} 
proposed to use the~diffusion equation model of the~Fokker-Planck type (FP) 
\cite{Planck_1917}.

%A more generally valid approach to the problem of
%treating changes in a distribution function resulting
%each of which from frequently occurring "events",
%produces a small change in the momentum of a particle,
%is to use the Fokker-Planck equation \cite{Planck_1917}, 
%which has been discussed by Spitzer and collaborators 
%\cite{SpitzerHarm_PR1953} . 
As a~result, a~classical paper by Spitzer and Harm (SH) 
\cite{SpitzerHarm_PR1953} provides the~computed electron distribution function
spanning from low to high Z plasmas, and more importantly, the~current and 
heat flux formulas, which are widely used in almost every plasma hydrodynamic
code nowadays.
%used the formalism of the FP equation to evaluate the collision terms of 
%the~Boltzmann equation under the assumptions that 
%(a) the events producing changes in particle momenta
%are two-body interactions described by the associated
%differential scattering cross sections, and 
%(b) that 
The~distribution function is of the form $f^0+\mu f^1$, where $f^0$ and $f^1$ 
are isotropic and $\mu$, is the direction cosine between the particle 
trajectory and some preferred direction in space. It should be emphasized that
the~SH solution expresses a~small perturbation of equilibrium, i.e. that 
$f^0$ is the~Maxwell-Boltzmann distribution and $\mu f^1$ represents 
a~very small deviation.

%It is the purpose of this paper to present the mechanics 
%of two-body collisions in a somewhat simplified
%form, and to derive the Fokker-Planck equation for an function. 
%From this general arbitrary distribution equation such special cases 
%as those of Chandrasekhar and Spitzer may easily be obtained. 
%For more complex situations the equation is suitable for integration by an
%electronic computer \cite{Rosenbluth_PR1957}.

The~actual cornerstone of the~modern FP simulations was set in place
by Rosenbluth \cite{Rosenbluth_PR1957}, when he derived a~simplified form 
of the~FP equation for a~finite expansion of the~distribution function,
where all the~terms are computed according to plasma conditions, including
$f^0$, which of course needs to tend to the~Maxwell-Boltzmann distribution.

Previously the Vlasov–Fokker–Planck (VFP) equation has been solved
numerically ignoring magnetic fields in 1-D (1 spatial dimension) 
\cite{Bell_1981_83, Matte_1982_86} to address heatflow down steep temperature 
gradients in unmagnetised plasma. Under these conditions the classical, fluid
description of transport \cite{SpitzerHarm_PR1953, Braginskii_1965_3}, 
which makes the local approximation, 
breaks down. They found that non-local effects are responsible for 
thermal transport inhibition \cite{Bell_1981_83}.

Particle-in-cell (PIC) codes \cite{PIC_Birdsall_Langdon_1985}
%\cite{Dawson_PoF1962, PIC_Birdsall_Langdon_1985} 
%and Vlasov codes $[23]$ 
are fully kinetic and are ideally suited to collisionless plasmas. 
Typically PIC codes tend to use explicit methods and consequently require a
time step small enough to resolve the fastest frequency present in the problem.
They also suffer from the so called "finite grid instability" 
\cite{PIC_Birdsall_Langdon_1985} 
whereby the plasma numerically heats up until the electron debye length is
resolved by the grid. These limitations mean that multidimensional, explicit, 
PIC cannot readily simulate the low temperature, high density plasma created in 
laser–solid interactions. 
%PIC codes utilising implicit methods do exist 
%$[24]$. Implicitness greatly relaxes the time step constraint and mitigates 
%the effects of the finite grid instability, so that lower temperatures 
%and higher densities can be dealt with. 
One spatial dimension PIC codes with electron–ion collisions have successfully 
been applied to fast electron transport through solid density plasma 
\cite{Guerin_PPCF1999, Perez_PoP2012}. 
%Two spatial dimension PIC codes with electron collisions also exist, both
%explicit $[25]$ and implicit $[26]$. 
Explicit 2-D PIC is unable to access conditions where collisions dominate for
the bulk of the electrons, though. Even though implicit methods overcome 
this particular problem, the PIC method in general struggles to adequately 
resolve the distribution function in a given cell when a realistic sized, 
2-D problem is addressed. Statistical noise and under resolution of 
the electron distribution lead to an inaccurate treatment of collisions and 
can overwhelm real physical effects present.


In is the~purpose of this paper to present an~efficient alternative 
to FP model based on the~Albritton-Williams-Bernstein-Swartz 
collision operator (AWBS) \cite{AWBS_PRL1986}.
In Section \ref{sec:AWBSmodel} we propose a~modified form of 
the~AWBS collision operator, where its important properties are further
presented in Section \ref{sec:DiffusiveKinetics} with the~emphasis on its
comparison to the~full FP solution in local diffusive regime. 
Section \ref{sec:BenchmarkingAWBS} focuses on the~performance of the~AWBS 
transport equation model compared to modern kinetic codes including FP codes
Aladin and Impact, and PIC code Calder, where the~cases related to real
laser generated plasma conditions are studied. Finally, the~most important
outcomes of our research are concluded in Section \ref{sec:Conclusions}. 
