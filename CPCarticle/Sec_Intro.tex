\section{Introduction}
\label{sec:Intro}
%In dealing with the nonequilibrium properties of
%systems whose particles obey an inverse-square law
%of interaction, it is convenient to make use of the fact
%that under most circumstances small-angle collisions
%are much more important than collisions resulting in
%large momentum changes. This leads to the method
%often used for treating such systems, in which one
%expands the collision integrand of the Boltzmann change per
%equation in powers of the momentum collision.

A more generally valid approach to the problem of
treating changes in a distribution function resulting
each of which from frequently occurring "events",
produces a small change in the momentum of a particle,
is to use the Fokker-Planck equation \cite{Planck_1917}, 
which has been discussed by Spitzer and collaborators 
\cite{CSR_1950, SpitzerHarm_PR1953} . They used the formalism 
of this equation to evaluate the collision terms of the~Boltzmann equation 
under the assumptions that 
(a) the events producing changes in particle momenta
are two-body interactions described by the associated
differential scattering cross sections, and 
(b) that the~distribution 
function is of the form $f^0+\mu f^1$, where $f^0$ and $f^1$ are isotropic
and $\mu$, is the direction cosine between the particle trajectory 
and some preferred direction in space, and is assumed to be small.

It is the purpose of this paper to present the mechanics 
of two-body collisions in a somewhat simplified
form, and to derive the Fokker-Planck equation for an function. 
From this general arbitrary distribution equation such special cases 
as those of Chandrasekhar and Spitzer may easily be obtained. 
For more complex situations the equation is suitable for integration by an
electronic computer \cite{Rosenbluth_PR1957}.

AWBS \cite{AWBS_PRL1986} 
