\section{Conclusions}
\label{sec:Conclusions}

\begin{table}
\begin{center}
  \begin{tabular}{c|ccccc}
    \hline\hline\\
    %Kn$^e$ & $10^{-4}$ & $10^{-3}$ & $10^{-2}$ & $10^{-1}$ & $1$ \\\\
    Kn$^e$ & $\,\,10^{-4}\,\,$ & $\,\,10^{-3}\,\,$ & $\,\,10^{-2}\,\,$ & $\,\,10^{-1}\,\,$ & $\,\,1\,\,$ \\\\
    \hline\\
    $\vmag_{lim} / \vth$ & 70.8 & 22.4 & 7.3 & 3.1 & 1.8\\\\
    \hline\hline
  \end{tabular}
  \caption{
  Scan over varying nonlocality (Kn$^e$) showing the~limit of 
  the~collision friction dominance over the~acceleration of electrons 
  due to the~electric field force. The~electric field effect is dominant
  for electrons with higher velocity than $\vmag_{lim}$ defined in 
  \eqref{eq:v_limit}. Kn$^e$ and $\vth$ are evaluated from the~same 
  plasma profiles.
  %$\sqrt{3}\vmag\frac{\me}{2\qe}\nue > |\E|$.
  }
\label{tab:vlim}
\end{center}
\end{table}

For practical reasons we present in \tabref{tab:vlim} 
some explicit values of velocity limit corresponding to varying transport 
conditions expressed in terms of "$\Zbar$" Knudsen number 
$\text{Kn}^e = \frac{\mfpe}{\sqrt{\Zbar + 1}L_{T_e}}$, 
where $\sqrt{\Zbar + 1}$ provides a~proper scaling of nonlocality with respect
to ionization, i.e. the~effect of scattering of electrons on ions 
\cite{LMV_1983_7}.

\begin{itemize}
  \item The~most important point is that we introduce a~collision operator, 
    which is coherent with the full FP, i.e. no extra dependence on $\Zbar$.
  \item Touch pros/contras of linearized FP in Aladin and Impact vs AWBS
  \item Raise discussion about what is the weakest point of AP1 for high Kns: 
    the~velocity limit or phenomenological Maxwellization?
  \item Summarize useful outcomes related to plasma physics as 
    the~tendency of the~velocity maximum in $q_1$ with respect to $\Zbar$ and
	Kn$^e$.
  \item Emphasize the~good results of Aladin (compared to Impact) and also
    outstanding results of Calder while being PIC. 
\end{itemize}
