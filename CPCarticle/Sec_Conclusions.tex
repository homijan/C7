\section{Conclusions}
\label{sec:Conclusions}

%\begin{itemize}
  %\item The~most important point is that we introduce a~collision operator, 
  %  which is coherent with the full FP, i.e. no extra dependence on $\Zbar$.
  %\item Touch pros/contras of linearized FP in Aladin and Impact vs AWBS
  %\item Raise discussion about what is the weakest point of AP1 for high Kns: 
  %\item Summarize useful outcomes related to plasma physics as
  %  the competition between collisions and electric field in the electron 
  %	stopping, then the knowledge about the nonlocal electron population 
  %	(preheat electrons can be tracked back to the point of source according 
  %	to their dominant velocity), and the last information about the~tendency 
  %	of the~velocity maximum in $q_1$ with respect to $\Zbar$ and Kn$^e$.
  %\item Emphasize the~good results of Aladin (compared to Impact) and also
  %  outstanding results of Calder while being PIC. 
  %\item Electric field plays an important role in nonlocal electron kinetics
  %  nonlocal Ohm's law provides the necessary equation to treat it properly.
  %\item In coherence with the latter, the dfdv must be treated properly in
  %  nolocal electron kinetics, and so, AWBS can be included without any extra
  %  effort thus making it way much suitable than BGK, which is outperformed
  %  by AWBS when compared to FP.
%\end{itemize}

In conclusion, we have performed a~thorough analysis of the~AWBS transport 
equation for electrons originally introduced in \cite{Sorbo_2015} and extended
it by adding the~nonlocal version of Ohm's law.
After properly redefining the~e-e colission term, we have shown that the~AWBS
simplified linear form of the~Fokker-Planck collision operator keeps
some important kinetic properties analyzed in local diffusive regime, 
e.g. provides a~correct dependence 
on $\Zbar$ (BGK requires an~additional fix) and inherently includes
the~anisotropic part of the~distribution function $\fone$, which compares
very well to the~full Fokker-Planck operator.
Under nonlocal transport plasma conditions, we benchmarked AP1 against 
the~reference VFP codes Aladin and Impact, collisional PIC code Calder, 
and we also included the~standard nonlocal approach SNB.
%It was shown in Sec. IV A that inclusion of this in KIPP and EIC
%predicts a noticeably different nonlocal deviation (consider,
%for example, the value b) than would be predicted by using
%the phenomenological collision fix. 
AP1 performed very well over all simulation cases while capturing the~important
kinetic features compared to the~reference kinetic codes. 
Furthermore, our extensive analysis of the~anisotropic part of the~EDF 
provided by AP1 showed a~very good match to all Aladin, Impact, and Calder. 
This suggests 
a~promising AP1's capability in predicting general transport coefficients and 
the~seeding of parameteric laser plasma instabilities sensitive 
to the~Landau damping of longitudinal plasma waves 
\cite{goldston1995introduction, Sorbo_2015},
which is of great importance in ICF related plasmas 
\cite{Kirkwood_NIFLPI_PPCF2013}.
Other kinetic effects as perpendicular transport, e.g heat flow
or magnetic field advection, ocurring in magnetised plasma  
\cite{Walsh_Nernst_PRL2017} are introduced in
AP1 via the~nonlocal Ohm's law. 
%which in general, features the~essential way
%of coupling electromagnetic fields to plasma.
On the~oher hand we observed an~inacurate kinetic results of 
the~\textit{decelerating} AP1 computation for highly nonlocal plasma 
conditions, which was explained by the~velocity limit applied to the~action
of the~electric field. A~general (not only the~\textit{decelerating}) numerical
solution of the~AP1 model, which avoids the~eletric field limit, 
will be the~main focus of our future work.
%LPI Kirkwood 2013 \cite{Kirkwood_NIFLPI_PPCF2013}
%Nernst Walsh 2017 \cite{Walsh_Nernst_PRL2017}
%Heat wave velocity measurement Schurtz 2007 \cite{Schurtz_2007}
%Nonlocal heat flux in laser-plasma corona Henchen 2018 \cite{Henchen_PRL2018}
%Return current instability Glenzer 2002 \cite{Glenzer_PRL2002}
%PETE code \cite{pete-code}
