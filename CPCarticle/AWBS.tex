%% This template can be used to write a paper for
%% Computer Physics Communications using LaTeX.
%% For authors who want to write a computer program description,
%% an example Program Summary is included that only has to be
%% completed and which will give the correct layout in the
%% preprint and the journal.
%% The `elsarticle' style is used and more information on this style
%% can be found at 
%% http://www.elsevier.com/wps/find/authorsview.authors/elsarticle.
%%
%%
%\documentclass[preprint,12pt]{elsarticle}

%% Use the option review to obtain double line spacing
%% \documentclass[preprint,review,12pt]{elsarticle}

%% Use the options 1p,twocolumn; 3p; 3p,twocolumn; 5p; or 5p,twocolumn
%% for a journal layout:
%% \documentclass[final,1p,times]{elsarticle}
%% \documentclass[final,1p,times,twocolumn]{elsarticle}
%% \documentclass[final,3p,times]{elsarticle}
%% \documentclass[final,3p,times,twocolumn]{elsarticle}
%% \documentclass[final,5p,times]{elsarticle}
%% \documentclass[final,5p,times,twocolumn]{elsarticle}

%% if you use PostScript figures in your article
%% use the graphics package for simple commands
%% \usepackage{graphics}
%% or use the graphicx package for more complicated commands
%\usepackage{graphicx}
%% or use the epsfig package if you prefer to use the old commands
%% \usepackage{epsfig}

%% The amssymb package provides various useful mathematical symbols
%\usepackage{amssymb}
%% The amsthm package provides extended theorem environments
%% \usepackage{amsthm}

%% Additional needed packages.
%\usepackage{amsmath}
%\usepackage{verbatim}
%\usepackage{color}

%% The lineno packages adds line numbers. Start line numbering with
%% \begin{linenumbers}, end it with \end{linenumbers}. Or switch it on
%% for the whole article with \linenumbers after \end{frontmatter}.
%\usepackage{lineno}
%\usepackage{hyperref}

%% natbib.sty is loaded by default. However, natbib options can be
%% provided with \biboptions{...} command. Following options are
%% valid:

%%   round  -  round parentheses are used (default)
%%   square -  square brackets are used   [option]
%%   curly  -  curly braces are used      {option}
%%   angle  -  angle brackets are used    <option>
%%   semicolon  -  multiple citations separated by semi-colon
%%   colon  - same as semicolon, an earlier confusion
%%   comma  -  separated by comma
%%   numbers-  selects numerical citations
%%   super  -  numerical citations as superscripts
%%   sort   -  sorts multiple citations according to order in ref. list
%%   sort&compress   -  like sort, but also compresses numerical citations
%%   compress - compresses without sorting
%%
%% \biboptions{comma,round}

% \biboptions{}

%% This list environment is used for the references in the
%% Program Summary
%%
%\newcounter{bla}
%\newenvironment{refnummer}{%
%\list{[\arabic{bla}]}%
%{\usecounter{bla}%
% \setlength{\itemindent}{0pt}%
% \setlength{\topsep}{0pt}%
% \setlength{\itemsep}{0pt}%
% \setlength{\labelsep}{2pt}%
% \setlength{\listparindent}{0pt}%
% \settowidth{\labelwidth}{[9]}%
% \setlength{\leftmargin}{\labelwidth}%
% \addtolength{\leftmargin}{\labelsep}%
% \setlength{\rightmargin}{0pt}}}
% {\endlist}

%\journal{Computer Physics Communications}

% ****** Start of file aipsamp.tex ******
%
%   This file is part of the AIP files in the AIP distribution for REVTeX 4.
%   Version 4.1 of REVTeX, October 2009
%
%   Copyright (c) 2009 American Institute of Physics.
%
%   See the AIP README file for restrictions and more information.
%
% TeX'ing this file requires that you have AMS-LaTeX 2.0 installed
% as well as the rest of the prerequisites for REVTeX 4.1
%
% It also requires running BibTeX. The commands are as follows:
%
%  1)  latex  aipsamp
%  2)  bibtex aipsamp
%  3)  latex  aipsamp
%  4)  latex  aipsamp
%
% Use this file as a source of example code for your aip document.
% Use the file aiptemplate.tex as a template for your document.
%\documentclass[%
% aip,
%%jmp,%
%%bmf,%
% sd,%
%%rsi,%
% amsmath,amssymb,
%%preprint,%
% reprint,%
%%author-year,%
%%author-numerical,%
%]{revtex4-1}

% Use this file as a source of example code for your aip document.
% Use the file aiptemplate.tex as a template for your document.
\documentclass[
 aps,
 jmp,%
 amsmath,amssymb,
 twocolumn,
% preprint,%
% reprint,%
%author-year,%
%author-numerical,%
]{revtex4-1}

\usepackage{graphicx}% Include figure files
\usepackage{dcolumn}% Align table columns on decimal point
\usepackage{bm}% bold math
%\usepackage[mathlines]{lineno}% Enable numbering of text and display math
%\linenumbers\relax % Commence numbering lines
\usepackage{verbatim}
%\usepackage{subfloat}
%\usepackage{subcaption}
%\usepackage{placeins}
%\usepackage{caption}
%\captionsetup{font=scriptsize}
\usepackage{color}

\usepackage{lineno}
\usepackage{hyperref}

\newcommand{\pdv}[2]{\frac{\partial{#1}}{\partial{#2}}}
\newcommand{\fp}[2]{\pdv{#1}{#2}}
\newcommand{\pdwc}[3]{\left(\fp{#1}{#2}\right)_{#3}}
%\newcommand{\ppdv}[3]{\displaystyle \frac{\partial^2 #1}{\partial #2\partial #3}}
\newcommand{\ppdv}[3]{\frac{\partial^2 #1}{\partial #2\partial #3}}
\newcommand{\oover}[1]{\ensuremath{\frac{1}{#1}}}

\newcommand{\vect}[1]{\boldsymbol{#1}}
\newcommand{\matr}[1]{\mathbf{#1}}
\newcommand{\dI}{\text{d}}
\newcommand{\odv}[2]{\frac{\dI #1}{\dI #2}}
\newcommand{\ddv}[2]{\odv{#1}{#2}}
\newcommand{\christ}[3]{\genfrac{\{}{\}}{0pt}{}{#1}{#2 #3}}
%\newcommand{\christ}[3]{\Gamma^{#1}_{#2 #3}}
%\newcommand{\ddv}[2]{\frac{\Delta #1}{\Delta #2}}
\newcommand{\collpdv}[2]{\pdv{#1}{#2}\Big|_{coll}}
\newcommand{\mfp}{\lambda}
\newcommand{\tmfp}{\tilde{\lambda}}
\newcommand{\mfpe}{\lambda_e}
\newcommand{\mfpei}{\lambda_{ei}}
\newcommand{\Zbar}{\bar{Z}}
\newcommand{\nue}{\nu_{e}}
\newcommand{\nuei}{\nu_{ei}}
\newcommand{\nutot}{\nu_{t}}
\newcommand{\vmag}{v}
\newcommand{\vth}{v_{th}}
\newcommand{\vn}{\vect{n}}
\newcommand{\E}{\vect{E}}
\newcommand{\B}{\vect{B}}
\newcommand{\tE}{\vect{\tilde{E}}}
\newcommand{\tEz}{\tilde{E}_z}
\newcommand{\tB}{\vect{\tilde{B}}}
\newcommand{\qe}{q_e}
\newcommand{\me}{m_e}
\newcommand{\kB}{k_B}
\newcommand{\crs}{\sigma}
\newcommand{\fM}{f_M}
\newcommand{\tfM}{\tilde{f}_M}
\newcommand{\tvfM}{\vect{\tilde{f}_M}}
\newcommand{\daf}{\delta f}
\newcommand{\fzero}{f_0}
\newcommand{\dafzero}{\delta f_0}
\newcommand{\vfzero}{\vect{f_0}}
\newcommand{\davfzero}{\vect{\delta f_0}}
\newcommand{\fone}{\vect{f_1}}
\newcommand{\fonez}{f_{1_z}}
\newcommand{\SM}{\vect{S}_M}
\newcommand{\MI}{\matr{I}}
\newcommand{\MA}{\matr{A}}
\newcommand{\intO}{\int_{\Omega}}
\newcommand{\intv}{\int_{\vmag}}
\newcommand{\IM}{\boldsymbol{\mathcal{M}}}
\newcommand{\ID}{\boldsymbol{\mathcal{D}}}
\newcommand{\IV}{\boldsymbol{\mathcal{V}}}
\newcommand{\IB}{\boldsymbol{\mathcal{B}}}
\newcommand{\anisomega}{\fone/\fzero}
\newcommand{\acl}{\vect{M}_{\left(\anisomega\right)}}

\newcommand{\Wzero}{\psi}
\newcommand{\Wone}{\matr{w}}
\newcommand{\Wcurl}{\vect{W_c}}
\newcommand{\Wdiv}{\vect{W_d}}



\newcommand{\figref}[1]{FIG.~\ref{#1}}
\newcommand{\refeq}[1]{\eqref{#1}}
\newcommand{\secref}[1]{Sec.~\ref{#1}}
\newcommand{\tabref}[1]{TABLE~\ref{#1}}
\newcommand{\figscale}{0.48}


\begin{document}

\preprint{AIP/123-QED}

\title[An~efficient kinetic modeling in plasmas relevant to inertial confinement fusion by using the~AWBS transport equation]{An~efficient kinetic modeling in plasmas relevant to inertial confinement fusion by using the~AWBS transport equation}% Force line breaks with \\
%\thanks{Footnote to title of article.}

\author{Authors}
%\author{M. Holec}
 \email{milan.holec@u-bordeaux.fr}
 \affiliation{
Centre Lasers Intenses et Applications, Universite de Bordeaux-CNRS-CEA,\\ 
UMR 5107, F-33405 Talence, France.
}
%\author{J. Nikl}
%\affiliation{ 
%ELI-Beamlines Institute of Physics, AS CR, v.v.i, 
%Na Slovance 2, Praha 8, 180 00, Czech Republic%\\This line break forced with \textbackslash\textbackslash
%}
%\affiliation{Czech Technical University in Prague, Faculty of 
% Nuclear Sciences and Physical Engineering,
% Brehova 7, 115 19 - Praha 1, Czech Republic.}
%\author{S. Weber}%
%%\email{stefan.weber@eli-beams.eu.}
%\affiliation{ 
%ELI-Beamlines Institute of Physics, AS CR, v.v.i, 
%Na Slovance 2, Praha 8, 180 00, Czech Republic.%\\This line break forced with \textbackslash\textbackslash
%}%

\date{\today}% It is always \today, today,
             %  but any date may be explicitly specified

\begin{abstract}
Text of abstract.
\end{abstract}

\pacs{Valid PACS appear here}% PACS, the Physics and Astronomy
                             % Classification Scheme.
\keywords{kinetics; nonlocal electron transport; laser-heated plasmas; hydrodynamics.}%Use showkeys class option if keyword
                              %display desired
\maketitle


%%
%% Start line numbering here if you want
%%
 \linenumbers

%\pagebreak
%\tableofcontents
%\pagebreak
%% main text
%\section{}
%\label{}
%---------------------------------------------------------------------
%---------------------------------------------------------------------

\section{Introduction}
\label{sec:Intro}
%In dealing with the nonequilibrium properties of
%systems whose particles obey an inverse-square law
%of interaction, it is convenient to make use of the fact
%that under most circumstances small-angle collisions
%are much more important than collisions resulting in
%large momentum changes. This leads to the method
%often used for treating such systems, in which one
%expands the collision integrand of the Boltzmann change per
%equation in powers of the momentum collision.

A more generally valid approach to the problem of
treating changes in a distribution function resulting
each of which from frequently occurring "events",
produces a small change in the momentum of a particle,
is to use the Fokker-Planck equation \cite{Planck_1917}, 
which has been discussed by Spitzer and collaborators 
\cite{CSR_1950, SpitzerHarm_PR1953} . They used the formalism 
of this equation to evaluate the collision terms of the~Boltzmann equation 
under the assumptions that 
(a) the events producing changes in particle momenta
are two-body interactions described by the associated
differential scattering cross sections, and 
(b) that the~distribution 
function is of the form $f^0+\mu f^1$, where $f^0$ and $f^1$ are isotropic
and $\mu$, is the direction cosine between the particle trajectory 
and some preferred direction in space, and is assumed to be small.

It is the purpose of this paper to present the mechanics 
of two-body collisions in a somewhat simplified
form, and to derive the Fokker-Planck equation for an function. 
From this general arbitrary distribution equation such special cases 
as those of Chandrasekhar and Spitzer may easily be obtained. 
For more complex situations the equation is suitable for integration by an
electronic computer \cite{Rosenbluth_PR1957}.

AWBS \cite{AWBS_PRL1986} 

\section{The~AWBS kinetic model}
\label{sec:AWBSmodel}

The~electrons in plasma can be modeled by the~deterministic Vlasov model 
of charged particles
\begin{equation}
  \pdv{\ft}{t} + \vv\cdot\gx \ft + 
  \frac{\qe}{\me}\left(\E + \frac{\vv}{c}\vect{\times}\B\right)\cdot\gv \ft = 
  C_{ee}(f) + C_{ei}(f) ,
  \label{eq:kinetic_equation}
\end{equation}
where $\ft(t, \vect{x}, \vect{v})$ represents the~density function of electrons
at time $t$, spatial point $\vect{x}$, and velocity $\vv$, $\E$ and $\B$ are 
the~electric and magnetic fields in plasma, $\qe$ and $\me$ being 
the~charge and mass of electron.

The~general form of the~e-e collision operator 
$C_{ee}$ is the~Fokker-Planck form published by Landau \cite{Landau_1936}
\begin{equation}
  C_{FP}(\ft) =
  \Gamma\int \gv\gv(\vv - \vvb) \cdot \left(
  \ft\, \gvb \ft - \ft\, \gv \ft \right)\, \dI\vvb ,
  \label{eq:LFP_model}
\end{equation}
where $\Gamma = \frac{4\pi\qe^4\lnc}{\me^2}$ and 
$\lnc$ is the~Coulomb logarithm.
The~e-i collision operator $C_{ei}$ could be expressed
in a~simpler form since massive ions are considered 
to be motionless compared to electrons \corrCPR{during a collision}. %CPR comment - sorry for the pedantic distinction just clarifying that it is possible to account for ion motion in the VFP equation while ignoring it in the collision operator
 The~scattering operator accounts
for the~change of electron velocity without change in the~velocity magnitude\corrCPR{, i.e. angular scattering}. 
It is expressed in spherical coordinates as
\begin{equation}
  C_{ei}(\ft) = \frac{\nuei}{2}
  \left(\pdv{}{\mu}\left((1 - \mu^2)\pdv{\ft}{\mu}\right)
  + \frac{1}{\sin^2\phi}\frac{\partial^2 \ft}{\partial\theta^2} \right) ,
  \label{eq:ei_scattering}
\end{equation}
where $\mu = \cos\phi$, $\phi$ and $\theta$ are the~polar and azimuthal 
angles, and $\nuei = \frac{\Zbar n_e \Gamma}{\vmag^3}$ is the~e-i
collision frequency.

The~e-e collision operator needs to be linearized for efficient computation\corrCPR{\st{s}}.
Fis\corrCPR{c}h introduced a~linear form of $C_{ee}$ in \cite{Fish_RMP1987} 
in the~high-velocity limit ($\vmag\gg\vth$) electron collision operator
\begin{multline}
  C_{H}(\ft) = \vmag \nue \pdv{}{\vmag}\left(\ft + 
  \frac{\vth^2}{\vmag}\pdv{f}{\vmag}\right) \\
  + \frac{\nue}{2}\left( 1 - \frac{\vth^2}{2\vmag^2}\right) 
  \left(\pdv{}{\mu}\left((1 - \mu^2)\pdv{f}{\mu}\right)
  + \frac{1}{\sin^2\phi}\frac{\partial^2f}{\partial\theta^2} \right)
  , \label{eq:HighVelocity_model}
\end{multline}
where $\nue = \frac{n_e \Gamma}{\vmag^3}$ is the~e-e collision 
frequency and $\vth = \sqrt{\frac{\kB T_e}{\me}}$ is the~electron thermal 
velocity.
The~linear form of $C_{H}$ arises from an~assumption that the~fast electrons 
predominantly interact with the~thermal (slow) electrons, 
which \corrCPR{ is an important simplification to} the~form \eqref{eq:LFP_model}.
However the~diffusion term in the~e-e collision operator 
\eqref{eq:HighVelocity_model} still presents numerical difficulties.

A~yet simpler form of the~collision operator of electrons was proposed in 
\cite{Sorbo_2015}
\begin{multline}
  C_{AWBS}(\ft) = \vmag \nue^*\pdv{}{\vmag}\left(\ft - \fM\right) \\
  + \frac{\nuei + \nue^*}{2} 
  \left(\pdv{}{\mu}\left((1 - \mu^2)\pdv{f}{\mu}\right)
  + \frac{1}{\sin^2\phi}\frac{\partial^2f}{\partial\theta^2} \right)
  , \label{eq:AWBS_model}
\end{multline}
where $\fM = \frac{n_e}{(2\pi)^{\frac{3}{2}}\vth^3}
\exp\left(-\frac{\vmag^2}{2\vth^2}\right)$ 
is the~Maxwell-Boltzmann equilibrium distribution.
Here, the~first term representing the AWBS operator \cite{AWBS_PRL1986}
accounts for relaxation to equilibrium due to the~e-e collisions, and 
the~second term accounts for the~e-i and e-e collisions contribution 
to scattering.

A~method of angular momenta for the~solution of the~electron kinetic equation
with the~collision operator \eqref{eq:AWBS_model} 
was introduced in \cite{Sorbo_2015, Sorbo_2016}. 

In \eqref{eq:AWBS_model} we have introduced a~modified e-e collision frequency
$\nue^*$ in order to address a~proper behavior with respect to $\Zbar$, which
is further analyzed in Section \ref{sec:DiffusiveKinetics} and promising 
results compared to the~full FP operator are presented.

%The~Maxwell-Boltzmann averaged e-e scattering in 
%\eqref{eq:HighVelocity_model} can be approximated as 
%$\nue \int\left(1 - \frac{\vth^2}{2\vmag^2}\right)\fM 4\pi\vmag^2\, \dI\vmag = 
%\frac{\nue}{2}$.

\section{BGK, AWBS, and Fokker-Planck models in local diffusive regime}
\label{sec:DiffusiveKinetics}

We can try to find an~approximate solution while using the~first term of
expansion in $\mfpei$ and $\mu$ as
\begin{equation}
  \tilde{f}(z, \vmag, \mu) = f^0(z, \vmag) + f^1(z, \vmag) \mfpei\mu ,
  \label{eq:f_approximation}
\end{equation}
where $\mfpei = \frac{\vmag^4}{\Zbar n_e \Gamma^{ee}}$.

\subsection{The~BGK local diffusive electron transport}
\label{sec:BGKDiffusiveRegime}

\begin{equation}
  \mu\left(\pdv{f}{z} + \frac{\tEz}{\vmag}\pdv{f}{\vmag}\right) 
  + \frac{\tEz(1-\mu^2)}{\vmag^2}\pdv{f}{\mu}
  = 
  \frac{f - \fM}{\mfpe}
  + \frac{1}{2 \mfpei}
  \pdv{}{\mu}(1 - \mu^2)\pdv{f}{\mu} ,
  \label{eq:OOE_AWBS_model_1D}
\end{equation}

\begin{eqnarray}
  f^0 &=& \fM + \frac{1}{\vmag}f^1 \Zbar\mfpei^2 ,
  \label{eq:BGK_f0} \\
  f^1 &=& - \frac{\Zbar}{\Zbar+1}
  \left[ \pdv{f^0}{z} + \frac{\tilde{E}_z}{\vmag}\pdv{f^0}{\vmag} \right] , 
  \label{eq:BGK_f1}
\end{eqnarray}
\begin{equation}
  f = \fM - \frac{\Zbar}{\Zbar+1}
  \left[\frac{1}{\rho}\pdv{\rho}{z} + 
  \left( \frac{\vmag^2}{2 \vth^2} - \frac{3}{2}\right)
  \frac{1}{T}\pdv{T}{z} - \frac{\tilde{E}_z}{\vth^2} \right]\fM \mfpei \mu , 
  \nonumber
\end{equation}
and when holds 
%\begin{equation}
$\vect{j} \equiv \qe \int \vv f \, \dI\vv = \vect{0}  \rightarrow
\tE = \vth^2\left(\frac{\nabla\rho}{\rho} + \frac{5}{2}\frac{\nabla T}{T} 
\right)$
%\end{equation}
, i.e. the~electric field $\tEz$ obeying the~zero current condition leads to
\begin{equation}
  f = \fM - \frac{\Zbar}{\Zbar+1}
  \left( \frac{\vmag^2}{2 \vth^2} - 4\right)
  \frac{1}{T}\pdv{T}{z}\fM \mfpei \mu , 
  \nonumber
\end{equation}

\begin{comment} % BGK appendix
\begin{multline}
  \vmag\vn\cdot\nabla f + \tE\cdot\vn \pdv{f}{\vmag} 
  + \frac{\tE\cdot\vect{e}_\theta}{\vmag}\pdv{f}{\theta}
  + \frac{\tE\cdot\vect{e}_\phi}
  {\vmag\sin(\theta)}\pdv{f}{\phi}
  =\\
  \vmag\frac{\left(\fM - f\right)}{\lambda^e} 
  + \frac{\vmag}{2 \lambda^{ei}} 
  \left(\pdv{}{\mu}\left((1 - \mu^2)\pdv{f}{\mu}\right)
  + \frac{1}{\sin^2(\theta)}\frac{\partial^2f}{\partial\phi^2} \right) ,
  \label{eq:BGK_spherical}
\end{multline}
where $\mu = \cos(\theta)$, $\mfpe$ is the~electron-electron mean free path, and
$\mfpei$ is the~electron-ion mean free path. We also approximate
$\mfpe = \Zbar \mfpei$.

%We can try to find an~approximate solution while using the~first term of
%expansion in $\mfp$, $\cos(\phi)$, $\sin(\phi)$, $\sin(\theta)$, and $\fM$ as
%\begin{equation}
%  f = f^0 + f^1 \mfp\fM\cos(\phi) + f^2 \mfp\fM\sin(\phi)\sin(\theta),
%  \label{eq:f_approximation}
%\end{equation}
Clearly, $\pdv{\tilde{f}}{\theta} = 0$, and if $\tB = \tilde{B}_z\vect{e}_z$, 
there is no effect of magnetic field. We also assume, that 
$\nabla f = \pdv{f}{z}\vect{e}_z$ and appropriately 
$\tE = \tilde{E}_z\vect{e}_z$.
From the~orientation of the~Cartesian basis vectors and spherical 
basis vectors, one can find $\tE\cdot\vn = \tilde{E}_z \cos(\phi) = \mu$ and
$\tE\cdot\vect{e}_\phi = -\tilde{E}_z\sin(\phi)$. As a~result, the~analyzed
BGK equation reads
\begin{multline}
  \mu\pdv{}{z}\left( f^0 + f^1 \mfpei\mu \right) 
  + \frac{1}{\vmag} \left[ \tilde{E}_z\mu \pdv{}{\vmag} 
  \left( f^0 + f^1 \mfpei\mu \right) 
  - \frac{\tilde{E}_z\sin(\phi)}{\vmag}\pdv{}{\phi} 
  \left( f^0 + f^1 \mfpei\mu \right)
  \right] 
  =\\
  \frac{\left(\fM - \left( f^0 + f^1 \mfpei\mu \right) \right)}{\mfpe} 
  + \frac{1}{2 \mfpei}\pdv{}{\mu}\left((1 - \mu^2)
  \pdv{}{\mu}\left( f^0 + f^1 \mfpei\mu \right) \right) ,
  \label{eq:BGK_spherical}
\end{multline}

\begin{multline}
  \mu\pdv{f^0}{z} + \mu^2 \pdv{}{z}\left(f^1 \mfpei \right)
  + \frac{\tilde{E}_z}{\vmag} \left[ \mu \pdv{f^0}{\vmag} 
  + \mu^2 \pdv{}{\vmag} \left( f^1 \mfpei \right) 
  + \frac{1 - \mu^2}{\vmag}f^1 \mfpei
  \right] 
  =\\
  \frac{\fM - f^0}{\Zbar\mfpei} - \mu \frac{1}{\Zbar}f^1
  - \mu f^1 ,
  \label{eq:BGK_spherical}
\end{multline}
consequently, we have the~following anisotropy expansion 
$\mu^0, \mu^1, \mu^2, ...$ equations
\begin{eqnarray}
  \frac{\fM - f^0}{\Zbar\mfpei} &=& \frac{\tEz}{\vmag^2}f^1 \mfpei , 
  \nonumber \\
  \pdv{f^0}{z} + \frac{\tilde{E}_z}{\vmag}\pdv{f^0}{\vmag} &=& 
  - \frac{1}{\Zbar}f^1 - f^1 , 
  \nonumber \\ 
  \pdv{}{z}\left(f^1 \mfpei \right) 
  + \frac{\tilde{E}_z}{\vmag} \left[\pdv{}{\vmag} \left( f^1 \mfpei \right)
  - \frac{1}{\vmag}f^1 \mfpei \right] &=& 0 , \nonumber
\end{eqnarray}
which lead to the~definitions
\begin{eqnarray}
  f^0 &=& \fM + \frac{1}{\vmag}f^1 \Zbar\mfpei^2 ,
  \label{eq:BGK_f0} \\
  f^1 &=& - \frac{\Zbar}{\Zbar+1}
  \left[ \pdv{f^0}{z} + \frac{\tilde{E}_z}{\vmag}\pdv{f^0}{\vmag} \right] 
  \nonumber \\
  &=& - \frac{\Zbar}{\Zbar+1}
  \left[\frac{1}{\rho}\pdv{\rho}{z} + 
  \left( \frac{\vmag^2}{2 \vth^2} - \frac{3}{2}\right)
  \frac{1}{T}\pdv{T}{z} - \frac{\tilde{E}_z}{\vth^2} \right]\fM 
  %\nonumber \\
  %&=& - \frac{\Zbar}{\Zbar+1}
  %\left( \frac{\vmag^2}{2 \vth^2} - \frac{3}{2} - \alpha\right)
  %\frac{1}{T}\pdv{T}{z}\fM ,
  \label{eq:BGK_f1}
\end{eqnarray}

In order to ensure the~plasma to be quasi-neutral, the~zero-current condition
\begin{equation}
  \vect{j} = \int_0^{\infty}\int_{4\pi} \qe \vmag \vn f 
  \, \dI\vn~\vmag^2~\dI\vmag 
  = \vect{0} ,
  \label{eq:zero_current}
\end{equation}
can be achieved by providing a~consistent electric field in 
\refeq{eq:f_approximation}, i.e.
\begin{equation}
  \tE = \frac{\vth^2~\int_{4\pi} \vn\otimes\vn\cdot \int_0^{\infty} \vmag  
  \fM \frac{\mfp}{\alpha}\left(\frac{\nabla\rho}{\rho} + 
  \left( \frac{\vmag^2}{2 \vth^2} - \frac{3}{2}\right) 
  \frac{\nabla T}{T}\right)
  \vmag^2\, \dI\vmag\, \dI\vn}
  {\int_{4\pi} \vn\otimes\vn\cdot \int_0^{\infty} \vmag  
  \fM \frac{\mfp}{\alpha}\vmag^2\, \dI\vmag\, \dI\vn} ,
\end{equation}
which may be further simplified as
\begin{equation}
  \tE = \frac{\int_0^{\infty} \fM
  \frac{1}{2}\frac{\nabla T}{T}\vmag^9\, \dI\vmag}
  {\int_0^{\infty} \fM \vmag^7\, \dI\vmag} + 
  \vth^2\left(\frac{\nabla\rho}{\rho} - \frac{3}{2}\frac{\nabla T}{T} \right)
  = \vth^2\left(\frac{\nabla\rho}{\rho} + \frac{5}{2}\frac{\nabla T}{T} 
  \right) ,
\end{equation}
where it is worth mentioning, that the~part 
$\fM + \frac{\vmag\mfp}{\alpha} \pdv{f_1}{\vmag}$ of the~distribution
does not contribute to the~current since it is isotropic.
One can write the~quasi-neutral distribution function explicitly 
distinguishing between original part (blue color) and E field correction
(red color) as
\begin{equation}
  f \approx \fM \left(1 - \frac{\mfp}{\alpha}\vn\cdot\left( 
  \textcolor{blue}{
  \frac{\vmag^2}{2 \vth^2} - \frac{3}{2}
  }
  \textcolor{red}{
  - \frac{5}{2}
  }
  \right) \frac{\nabla T}{T} \right) 
  + \frac{\vmag\mfp}{\alpha} \pdv{f_1}{\vmag} .
  \label{eq:f_localized_quasineutral}
\end{equation}
which leads to the~resulting heat flux
\begin{equation}
  \vect{q}_H = \int_{4\pi}\int_0^{\infty} \frac{\me \vmag^2}{2} \vmag \vn f 
  \vmag^2\, \dI\vmag\, \dI\vn = \frac{4\pi}{3}\frac{\me}{2}
  \frac{1}{\alpha \crs\rho}
  \int_0^{\infty} \left( 
  \textcolor{blue}{
  \frac{\vmag^2}{2 \vth^2} - \frac{3}{2}
  }
  \textcolor{red}{
  - \frac{5}{2}
  }
  \right) \vmag^9 \fM\, \dI\vmag \frac{\nabla T}{T} .
  \nonumber
\end{equation} 
Based on the~Gauss integral formula
\begin{equation}
  \int \vmag^{2s+1} \exp\left(-\frac{\vmag^2}{2\vth^2}\right)\, \dI\vmag = 
  \frac{s!~(2 \vth^2)^{s+1}}{2}
  \nonumber 
\end{equation}
and Maxwell-Boltzmann distribution \refeq{eq:MBdistribution} the~heat flux can 
be written as
\begin{equation}
  \vect{q}_H = \frac{4\pi}{3}\frac{\me}{2}\frac{1}{\alpha \crs\rho}
  \frac{\rho}{\vth^3 \left( 2 \pi \right)^{3/2}}
  \frac{4!~2^4 \vth^{10}}{T} \left( 
  \textcolor{blue}{
  5 - \frac{3}{2}
  }
  \textcolor{red}{
  - \frac{5}{2}
  }
  \right) \nabla T 
  = \frac{\me}{\alpha \crs}\frac{128}{\sqrt{2 \pi}}
  \left(\frac{\kB}{\me}\right)^{\frac{7}{2}}T^{\frac{5}{2}} \nabla T .
  \label{eq:Lorentz_flux}
\end{equation} 
In conclusion, equation \refeq{eq:Lorentz_flux} provides nothing else than
the~well known Lorentz approximation heat flux and its nonlinearity $2.5$
in temperature. What is worth mentioning is the~effect of E field 
(quasi-neutrality), which reduces the~flux of about	71.4$\%$ 
(also assuming constant density).


Finally, one can find the~approximate solution
\begin{equation}
  \tilde{f} = \fM - \mfpei\frac{\Zbar}{\Zbar+1}
  \left( \frac{\vmag^2}{2 \vth^2} - \frac{3}{2} - \alpha\right)
  \frac{\vn\cdot\nabla T}{T}\fM .
  \label{eq:BGK_approximate_solution}
\end{equation}
\end{comment} % BGK appendix

\subsection{The~AWBS diffusive electron transport}
\label{sec:AWBSDiffusiveRegime}

\begin{multline}
  \mu\left(\pdv{f}{z} + \frac{\tEz}{\vmag}\pdv{f}{\vmag}\right) 
  + \frac{\tEz(1-\mu^2)}{\vmag^2}\pdv{f}{\mu}
  = \\ 
  \frac{\vmag}{2\mfpe} \pdv{}{\vmag}\left(f - \fM\right) 
  + \frac{1}{2}\left(\frac{1}{\mfpei} + \frac{1}{2\mfpe}\right)
  \pdv{}{\mu}(1 - \mu^2)\pdv{f}{\mu} ,
  \label{eq:OOE_AWBS_model_1D}
\end{multline}

\begin{eqnarray}
  \pdv{}{\vmag}\left( f^0 -\fM\right) &=& \frac{1}{\vmag^2}f^1 \Zbar\mfpei^2 ,
  \label{eq:AWBS_f0} \\
  \frac{\vmag}{\Zbar\mfpei}\pdv{\left(f^1 \mfpei\right)}{\vmag}  
  - \frac{\Zbar + 1}{\Zbar} f^1 &=&
  \pdv{f^0}{z} + \frac{\tilde{E}_z}{\vmag}\pdv{f^0}{\vmag}
  \nonumber 
  %\\  
  %\frac{\vmag}{\Zbar}\pdv{f^1}{\vmag} + \frac{4}{\Zbar}f^1 
  %- \frac{\Zbar + 1}{\Zbar} f^1 &=&
  %\pdv{f^0}{z} + \frac{\tilde{E}_z}{\vmag}\pdv{f^0}{\vmag}
  %\nonumber
\end{eqnarray}

\begin{equation}
  \pdv{f^1}{\vmag} + \frac{1}{\vmag}(3-\Zbar)f^1
  =
  \frac{\Zbar}{\vmag}\left(\frac{1}{\rho}\pdv{\rho}{z} + 
  \left( \frac{\vmag^2}{2 \vth^2} - \frac{3}{2}\right)
  \frac{1}{T}\pdv{T}{z} - \frac{\tilde{E}_z}{\vth^2}\right)\fM .
  \label{eq:AWBS_f1}
\end{equation}

\begin{comment} % AWBS appendix
The~AWBS electron transport equation in 6D reads
\begin{multline}
  \vn\cdot\nabla f + \frac{1}{\vmag} \left[ \tE\cdot\vn \pdv{f}{\vmag} 
  + \frac{\tE\cdot\vect{e}_\phi 
  - \vmag\tB\cdot\vect{e}_\theta}{\vmag}\pdv{f}{\phi}
  + \frac{\tE\cdot\vect{e}_\theta + \vmag\tB\cdot\vect{e}_\phi}
  {\vmag\sin(\phi)}\pdv{f}{\theta} \right] 
  =\\
  \frac{\vmag}{\lambda^e}\pdv{}{\vmag}\left(f - f_M\right) 
  + \left(\frac{1}{\mfpei} + \frac{1}{\mfpe} \right) \frac{1}{2}
  \left(\pdv{}{\mu}\left((1 - \mu^2)\pdv{f}{\mu}\right)
  + \frac{1}{\sin^2(\phi)}\frac{\partial^2f}{\partial\theta^2} \right) ,
  \label{eq:AWBS_spherical}
\end{multline}
where $\mu = \cos(\phi)$, $\mfpe$ is the~electron-electron mean free path, and
$\mfpei$ is the~electron-ion mean free path, and $\mfpe = \Zbar \mfpei$.

We can try to find an~approximate solution while using the~first term of
expansion in $\mfpe$ and $\mu$ as
\begin{equation}
  \tilde{f}(z, \vmag, \mu) = f^0(z, \vmag) + f^1(z, \vmag) \mfpei\mu .
  \label{eq:f_approximation}
\end{equation}
Clearly, $\pdv{\tilde{f}}{\theta} = 0$, and if $\tB = \tilde{B}_z\vect{e}_z$, 
there is no effect of magnetic field. We also assume, that 
$\nabla f = \pdv{f}{z}\vect{e}_z$ and appropriately 
$\tE = \tilde{E}_z\vect{e}_z$.
From the~orientation of the~Cartesian basis vectors and spherical 
basis vectors, one can find $\tE\cdot\vn = \tilde{E}_z \cos(\phi) = \mu$ and
$\tE\cdot\vect{e}_\phi = -\tilde{E}_z\sin(\phi)$. As a~result, the~analyzed
AWBS equation reads
\begin{multline}
  \mu\pdv{}{z}\left( f^0 + f^1 \mfpei\mu \right) 
  + \frac{1}{\vmag} \left[ \tilde{E}_z\mu \pdv{}{\vmag} 
  \left( f^0 + f^1 \mfpei\mu \right) 
  - \frac{\tilde{E}_z\sin(\phi)}{\vmag}\pdv{}{\phi} 
  \left( f^0 + f^1 \mfpei\mu \right)
  \right] 
  =\\
  \frac{\vmag}{\mfpe}\pdv{}{\vmag}
  \left(\left( f^0 + f^1 \mfpei\mu \right) -\fM\right)
  + \frac{\Zbar +1}{2 \mfpei \Zbar}\pdv{}{\mu}\left((1 - \mu^2)
  \pdv{}{\mu}\left( f^0 + f^1 \mfpei\mu \right) \right) ,
  \label{eq:AWBS_spherical}
\end{multline}

\begin{multline}
  \mu\pdv{f^0}{z} + \mu^2 \pdv{}{z}\left(f^1 \mfpei \right)
  + \frac{\tilde{E}_z}{\vmag} \left[ \mu \pdv{f^0}{\vmag} 
  + \mu^2 \pdv{}{\vmag} \left( f^1 \mfpei \right) 
  + \frac{1 - \mu^2}{\vmag}f^1 \mfpei
  \right] 
  =\\
  \frac{\vmag}{\Zbar\mfpei}\pdv{}{\vmag}
  \left( f^0 -\fM\right)
  + \mu\frac{\vmag}{\Zbar\mfpei}\pdv{\left(f^1 \mfpei\right)}{\vmag} 
  - \mu \frac{\Zbar + 1}{\Zbar}f^1 ,
  \label{eq:AWBS_spherical}
\end{multline}
consequently, we have the~following anisotropy expansion 
$\mu^0, \mu^1, \mu^2, ...$ equations
\begin{eqnarray}
  \frac{\vmag}{\Zbar\mfpei}\pdv{}{\vmag}
  \left( f^0 -\fM\right) &=& \frac{\tEz}{\vmag^2}f^1 \mfpei , 
  \nonumber \\
  \pdv{f^0}{z} + \frac{\tilde{E}_z}{\vmag}\pdv{f^0}{\vmag} &=& 
  \frac{\vmag}{\Zbar\mfpei}\pdv{\left(f^1 \mfpei\right)}{\vmag}  
  - \frac{\Zbar + 1}{\Zbar} f^1 , 
  \nonumber \\ 
  \pdv{}{z}\left(f^1 \mfpei \right) 
  + \frac{\tilde{E}_z}{\vmag} \left[\pdv{}{\vmag} \left( f^1 \mfpei \right)
  - \frac{1}{\vmag}f^1 \mfpei \right] &=& 0 , \nonumber
\end{eqnarray}
which lead to the~definitions
\begin{eqnarray}
  \pdv{}{\vmag}\left( f^0 -\fM\right) &=& \frac{1}{\vmag^2}f^1 \Zbar\mfpei^2 ,
  \label{eq:AWBS_f0} \\
  \frac{\vmag}{\Zbar\mfpei}\pdv{\left(f^1 \mfpei\right)}{\vmag}  
  - \frac{\Zbar + 1}{\Zbar} f^1 &=&
  \pdv{f^0}{z} + \frac{\tilde{E}_z}{\vmag}\pdv{f^0}{\vmag}
  \nonumber \\  
  \frac{\vmag}{\Zbar}\pdv{f^1}{\vmag} + \frac{4}{\Zbar}f^1 
  - \frac{\Zbar + 1}{\Zbar} f^1 &=&
  \pdv{f^0}{z} + \frac{\tilde{E}_z}{\vmag}\pdv{f^0}{\vmag}
  \nonumber \\
  \pdv{f^1}{\vmag} + \frac{1}{\vmag}(3-\Zbar)f^1
  &=&
  \frac{\Zbar}{\vmag}\left(\frac{1}{\rho}\pdv{\rho}{z} + 
  \left( \frac{\vmag^2}{2 \vth^2} - \frac{3}{2}\right)
  \frac{1}{T}\pdv{T}{z} - \frac{\tilde{E}_z}{\vth^2}\right)\fM .
  \label{eq:AWBS_f1}
\end{eqnarray}
\end{comment} % AWBS appendix

\subsection{The~Fokker-Planck diffusive electron transport}
\label{sec:FPDiffusiveRegime}
\newcommand{\gt}{g}
\newcommand{\gM}{\gt_M}

\begin{multline}
  \mu\left(\pdv{f}{z} + \frac{\tEz}{\vmag}\pdv{f}{\vmag}\right) 
  + \frac{\tEz(1-\mu^2)}{\vmag^2}\pdv{f}{\mu}
  = \\ 
  \frac{\Gamma^{ee}}{\vmag} \left(4\pi f^2 
  + \frac{\gv\gv \ft : \gv\gv \gt}{2}\right)
  + \frac{1}{\mfpei}\pdv{}{\mu}(1 - \mu^2)\pdv{f}{\mu} ,
  \nonumber
\end{multline}
where $g(\vv) = \int |\vv - \vv^*| f(\vv^*)\, \dI\vv^*$ is 
the~Rosenbluth potential \cite{Rosenbluth_PR1957}. Since we are interested in 
the~approximate solution in the~diffusive regime, it is convenient to
use a~low anisotropy approximation 
$\tilde{\gt} = \gt^0(f^0) + \gt^1(f^1) \mfpei(\vmag) \mu$, which arises from 
Eq. 45 in \cite{Rosenbluth_PR1957}.

\begin{multline}
  \Gamma^{ee} \left(4\pi \tilde{f}^2 
  + \frac{\gv\gv \tilde{\ft} : \gv\gv \tilde{\gt}}{2}\right) = 
  \Gamma^{ee}\left(4\pi{\ft^0}^2 + 
  \frac{1}{2}\frac{\partial^2 \ft^0}{\partial \vmag^2}
  \frac{\partial^2 \gt^0}{\partial \vmag^2}
  + \frac{1}{\vmag^2}\pdv{\ft^0}{\vmag}\pdv{\gt^0}{\vmag} \right)
  \\
  + \frac{\mu}{n_e}
  \Bigg[8\pi \ft^0 \ft^1\vmag^4 - \vmag\left(\pdv{\ft^0}{\vmag}\gt^1 
  + \pdv{\gt^0}{\vmag}\ft^1\right) 
  + \frac{1}{\vmag^2}\left(\pdv{\ft^0}{\vmag}
  \pdv{(\gt^1\vmag^4)}{\vmag}
  + \pdv{\gt^0}{\vmag}\pdv{(\ft^1\vmag^4)}{\vmag}\right) \\
  + \frac{1}{2}\left(\frac{\partial^2 \ft^0}{\partial \vmag^2}
  \frac{\partial^2 (\gt^1\vmag^4)}{\partial \vmag^2}
  + \frac{\partial^2 \gt^0}{\partial \vmag^2} 
  \frac{\partial^2 (\ft^1\vmag^4)}{\partial \vmag^2}
  \right) \Bigg] + O(\mfpei^2, \mu^2)
  \nonumber
\end{multline}

\begin{equation}
  \Gamma^{ee}\left(4\pi{\ft^0}^2 + 
  \frac{1}{2}\frac{\partial^2 \ft^0}{\partial \vmag^2}
  \frac{\partial^2 \gt^0}{\partial \vmag^2}
  + \frac{1}{\vmag^2}\pdv{\ft^0}{\vmag}\pdv{\gt^0}{\vmag} \right) = \frac{1}{\vmag^2}f^1 \Zbar\mfpei^2 ,
  \label{eq:FP_f0} 
\end{equation}
%\begin{multline}
%  \frac{1}{2}\left(\frac{\partial^2 \ft^0}{\partial \vmag^2}
%  \frac{\partial^2 (\gt^1\vmag^4)}{\partial \vmag^2}
%  + \frac{\partial^2 \gt^0}{\partial \vmag^2} 
%  \frac{\partial^2 (\ft^1\vmag^4)}{\partial \vmag^2}
%  \right) 
%  + \frac{1}{\vmag^2}\left(\pdv{\ft^0}{\vmag} \pdv{(\gt^1\vmag^4)}{\vmag} 
%  + \pdv{\gt^0}{\vmag}\pdv{(\ft^1\vmag^4)}{\vmag}\right) \\
%  - \vmag\left(\pdv{\ft^0}{\vmag}\gt^1 
%  + \pdv{\gt^0}{\vmag}\ft^1\right)
%  + 8\pi \ft^0 \ft^1\vmag^4 
%  - \vmag \ft^1
%  = 
%  \vmag\pdv{f^0}{z} + \tilde{E}_z\pdv{f^0}{\vmag}
%  ,
%  \label{eq:FP_f1}
%\end{multline}

\begin{multline}
  \frac{1}{2}\left(\frac{\partial^2 \fM}{\partial \vmag^2}
  \frac{\partial^2 (\gt^1\vmag^4)}{\partial \vmag^2}
  + \frac{\partial^2 \gM}{\partial \vmag^2} 
  \frac{\partial^2 (\ft^1\vmag^4)}{\partial \vmag^2}
  \right) 
  + \frac{1}{\vmag^2}\left(\pdv{\fM}{\vmag} \pdv{(\gt^1\vmag^4)}{\vmag} 
  + \pdv{\gM}{\vmag}\pdv{(\ft^1\vmag^4)}{\vmag}\right) \\
  - \vmag\left(\pdv{\fM}{\vmag}\gt^1 
  + \pdv{\gM}{\vmag}\ft^1\right)
  + 8\pi \fM \ft^1\vmag^4 
  - n_e \vmag \ft^1
  = 
  %\vmag\left(\frac{1}{\rho}\pdv{\rho}{z} + 
  %\left( \frac{\vmag^2}{2 \vth^2} - \frac{3}{2}\right)
  %\frac{1}{T}\pdv{T}{z} - \frac{\tilde{E}_z}{\vth^2}\right)\fM
  n_e \vmag\pdv{\fM}{z} + n_e \tilde{E}_z\pdv{\fM}{\vmag}
  ,
  \label{eq:FP_f1_equation}
\end{multline}

%The~Rosenbluth equation \refeq{eq:FP_Rosenbluth} can be further rewritten
%according to $[$Longmire, Conrad L. : Elementary Plasma Physics. Intersci. Pub., 1963$]$ as
%\begin{equation}
%  \left(\pdv{\ft}{t}\right)_c = \Gamma^{ee} 4\pi f^2 
%  + \frac{\gv\gv\Rgb : \gv\gv \ft}{2} ,
%  \label{eq:FP_Longmire}
%\end{equation}
%which was also published in Shkarofsky 1966 and used in Tzoufras 2011.

\begin{comment} % FP appendix
\begin{equation}
  \vtwoh = \sqrt{\frac{2 \kB T}{\me}} = 1 / j,
  \nonumber
\end{equation}
\begin{eqnarray}
  A &=& -\frac{\me^2 \vtwoh^2 \tE}{2 \pi e^4 n_e \lnc} = - \frac{m E}{2\pi j^2 e^3 n_e \lnc}
  , \nonumber \\
  B &=& \frac{\me^2 \vtwoh^4 |\nabla T|}{2 \pi e^4 n_e \lnc T} = \frac{2 \kB^2 T |\nabla T|}{\pi e^4 n_e \lnc}
  , \nonumber
\end{eqnarray}
\begin{equation}
  \frac{A}{B} = - \frac{|\tE| T}{\vtwoh^2 |\nabla T|} ,
  \nonumber
\end{equation}
\begin{equation}
  \tE = -\frac{3}{2}\frac{\vtwoh^2}{2}\frac{\gamma_T}{\gamma_E}
  \frac{\nabla T}{T} ,
  \nonumber
\end{equation}
From Eq. (24) CSR, we can write the~form of $f_1$ including both $\nabla T$ 
and $\tE$ effects as
\begin{equation}
  f_1(\vmag, \theta) = \cos(\theta) \frac{B}{\Zbar}\left( d_T(\vmag/\vtwoh) 
  + \frac{A}{B} d_E(\vmag/\vtwoh) \right) \fM(\vmag)  ,
  \nonumber
\end{equation}
where in the~case of vanishing current one gets
\begin{equation}
  \frac{A}{B} = \frac{3}{2}\frac{\gamma_T}{2 \gamma_E} ,
  \nonumber
\end{equation}
i.e.
\end{comment} % FP appendix
\begin{equation}
  f_1(\vmag, \mu) = \mu \frac{\me^2}{4 \pi e^4\lnc} 
  \frac{\vtwoh^4}{\Zbar}\left( 2 d_T(\vmag/\vtwoh) 
  + \frac{3}{2}\frac{\gamma_T}{\gamma_E} d_E(\vmag/\vtwoh) \right) 
  \frac{\fM(\vmag)}{n_e}\frac{\nabla T}{T}  ,
  \label{eq:f1_SH}
\end{equation}
where $d_T(x) = \Zbar D_{T}(x) / B$ and $d_E(x) = \Zbar D_{E}(x) / A$ 
are represented by numerical values in TABLE I and TABLE II in 
\cite{SH_PR1953}, respectively. 

%In the~case of high $\Zbar$ limit, $\gamma_T \rightarrow 1$,
%$\gamma_E \rightarrow 1$, $d_E(x) = x^4$, and $d_T(x) = x^4 (2.5 - x^2)/2$
%\cite{SH_PR1953}, which leads to the~standard Lorentz gas model
%\begin{equation}
%   f_1(\vmag, \theta) = \cos(\theta) \frac{\me^2}{4 \pi e^4\lnc} 
%  \frac{\vmag^4}{\Zbar}\left( 4 - \frac{\vmag^2}{\vtwoh^2} \right) 
%  \frac{\fM(\vmag)}{n_e}\frac{\nabla T}{T}  ,
%  \label{eq:f1_Lorentz} 
%\end{equation}

\cite{Rosenbluth_PR1957}, \cite{Longmire_1963}, \cite{Shkarofsky_1966}

\subsection{Summary of BGK, AWBS, and Fokker-Planck diffusion}
\label{sec:SummaryDiffusiveKinetics}

\begin{figure}[tbh]
  \begin{center}
    \begin{tabular}{c}
      \includegraphics[width=1.0\textwidth]{q1s.png}
    \end{tabular}
  \caption{  
  The~flux velocity moment of the~anisotropic part of the~electron distribution 
  function in low $Z=1$ and high $Z=116$ plasmas in diffusive regime.}
  \end{center}
  \label{fig:q1s_summary}
\end{figure}

\begin{table}
\begin{center}
  \begin{tabular}{c|ccccc}
    \hline\hline\\
    %$\Zbar$ & $1$ & $2$ & $4$ & $16$ & $\infty$ \\\\
    & $\,\Zbar=1\,$ & $\,\Zbar=2\,$ & $\,\Zbar=4\,$ & $\,\Zbar=16\,$ & $\,\Zbar=116\,$ \\\\
    \hline\\
    $\bar{\Delta}\vect{q}_{AWBS}$ & 0.057 & 0.004 & 0.038 & 0.049 & 0.004 \\\\
    \hline\hline
  \end{tabular}
  \caption{
  Relative error $\bar{\Delta}\vect{q}_{AWBS} = 
  |\vect{q}_{AWBS} - \vect{q}_{SH}| / \vect{q}_{SH}$ of the~AWBS
  kinetic model equation \refeq{eq:AWBS_model} showing the~discrepancy 
  (maximum around 5$\%$) with respect to the~original solution of 
  the~heat flux given by Spitzer and Harm \cite{SH_PR1953}.
  }
\end{center}
\label{tab:qAWBS}
\end{table}



%\section{Benchmarking the~AWBS nonlocal transport model}
%\label{sec:BenchmarkingAWBS}
%After having shown several encouraging properties of the~AWBS transport 
%equation defined by \eqref{eq:AWBS_model} under local diffusive conditions
%in \secref{sec:DiffusiveKinetics}, this section provides a~broader analysis
%of the~electron transport and focuses on analysis its behavior under variety of
%conditions in plasmas. In principle, this is characterized by allowing that
%electron mean free path can be arbitrarily long, which leads to so-called 
%nonlocal electron transport extensively investigated in numerous publications 
%\cite{Malone_1975_15, Colombant_PoP2005, Bell_1981_83, LMV_1983_7, Brantov_Nonlocal_electron_transport_1998, schurtz2000, Sorbo_2015}, where the~Fokker-Planck
%modeling of electrons in plasma represents the~essential tool. Being so, 
%we introduce our implementation of the~AWBS transport equation called AP1,
%where its results are further benchmarked against simulation results
%provided by Aladin, Impact FP codes, and Calder a~collisional Particle-In-Cell
%code. Their description follows in the~next section.
\section{The AWBS nonlocal transport model of electrons}
\label{sec:C7cod}

Here, we use the~collision operator \eqref{eq:AWBS_model} with 
\eqref{eq:qAWBS_approximation} and the~P1 angular discretization 
referred to as AP1. It adopts the~so-called angular 
moments method with the~electron distribution function having the~form
\begin{equation}
  \tilde{\ft} = \frac{\fzero}{4\pi} + \frac{3}{4\pi}\vn\cdot\fone , 
  \label{eq:P1approximation}
\end{equation}
which consists of the~isotropic part represented by the zeroth angular moment 
$\fzero = \int_{4\pi} \tilde{\ft} \dI\vn$ 
and the~directional part represented by the~first angular moment 
$\fone = \int_{4\pi} \vn
\tilde{\ft} \dI\vn$, where $\vn$ is the~transport direction (the~solid angle).
It should noticed, that \eqref{eq:P1approximation} represents 
a~multi-dimensional equivalent to \eqref{eq:f_approximation}, where 
the~following relations between the~spherical harmonics method
and the~moments method hold $\ft^0 = \frac{\fzero}{4\pi}$ and 
$\ft^1 = \frac{3}{4\pi}|\fone|$.

The~first two angular moments applied to the~steady form of 
\eqref{eq:kinetic_equation} with collision operator \eqref{eq:AWBS_model} 
using \eqref{eq:qAWBS_approximation} lead to the~AP1 model equations
\begin{eqnarray}
  \vmag\frac{\nue}{2}\pdv{}{\vmag}\left(\fzero - 4\pi\fM \right) &=&
  \vmag\nabla\cdot\fone + \frac{\qe}{\me}\E\cdot\left(
  \pdv{\fone}{\vmag} + \frac{2}{\vmag}\fone\right) , 
  \nonumber \\
  \label{eq:AP1f0}\\
  \vmag\frac{\nue}{2}\pdv{\fone}{\vmag}
  - \nuscat\fone &=& 
  \frac{\vmag}{3}\nabla\fzero + 
  \frac{\qe}{\me}\frac{\E}{3}\pdv{\fzero}{\vmag} ,
  \label{eq:AP1f1}
\end{eqnarray}
where $\nuscat = \nuei + \frac{\nue}{2}$. 

The~strategy of solving 
\eqref{eq:AP1f0} and \eqref{eq:AP1f1} resides in integrating 
$\pdv{\fzero}{\vmag}$
and $\pdv{\fone}{\vmag}$ along the~velocity magnitude. 
This is done by starting the~integration
from infinite velocity ($\vmag = 7 \vth^{max}$) to zero velocity. The~value
$\vth^{max}$ equals the~electron thermal velocity corresponding to the~maximum 
electron temperature in the~current profile of plasma.
It should be noted, that the~backward integration concept is crucial for 
the~model, since it corresponds to the~deceleration of electrons due to 
collisions \cite{Touati_2014}. 

\subsection{Nonlocal electric field treatment}
\label{sec:Efield}

%\begin{multline}
%  %\frac{\vect{j}}{\qe} = 
%  \vect{q}_c \equiv
%  \qe \intv \Bigg[\frac{\frac{\nue}{2}\vmag^2}{\nuei + \frac{\nue}{2}}
%  \pdv{\fone}{\vmag} 
%  - \frac{\vmag^2}{3\left(\nuei + \frac{\nue}{2}\right)}
%  \nabla\fzero \\
%  - \frac{\vmag}{3\left(\nuei + \frac{\nue}{2}\right)}\pdv{\fzero}{\vmag}\tE
%  \Bigg] \vmag^2\, \dI\vmag = 0
%  , \nonumber
%\end{multline}
Similarly to the~quasi-neutrality condition \eqref{app_eq:BGK_Efield}, 
one can obtain the~model equation of the~electric field $\E$ by evaluating 
the~zero current condition (a~first moment velocity integration of 
\eqref{eq:AP1f1})
\begin{equation}
  \intv \left(\frac{\frac{\nue}{2}\vmag^2}{\nuscat}
  \pdv{\fone}{\vmag} 
  - \frac{\vmag^2}{3\nuscat}
  \nabla\fzero 
  - \frac{\vmag}{3\nuscat}\pdv{\fzero}{\vmag}\frac{\qe}{\me}\E
  \right) \vmag^2\, \dI\vmag = 0 ,
  \label{eq:AP1_Efield}
\end{equation}
from which it is easy to express $\E$ once $\fzero$ and $\fone$ are known.

In other words, the~integral-differential model equations 
\eqref{eq:AP1f0}, \eqref{eq:AP1f1}, and \eqref{eq:AP1_Efield}
need to be solved simultaneously. This is achieved by $k$-iteration of 
$\fzero^k(\E^k), \fone^k(\E^k)$, i.e. \eqref{eq:AP1f0}, \eqref{eq:AP1f1}, and 
$\E^{k+1}(\fzero^k, \fone^k)$, i.e.  \eqref{eq:AP1_Efield}, until 
the~current evaluation \eqref{eq:AP1_Efield} converges to zero. In principle,
our concept of $k$-iteration resembles to the~embedded nonlinear iteration
of the~implicit E field introduced in \cite{Kingham_JCP2004}.
The~first iteration starts with $\E=\vect{0}$ in \eqref{eq:AP1f0} and 
\eqref{eq:AP1f1} and usually less than 10 iterations is sufficient to obey
the~quasi-neutrality constraint.
%\begin{eqnarray}
%  \pdv{\fzero}{\vmag} &=&
%  \frac{2}{\nue}\pdv{\fonez}{z} + \frac{2\Ez}{\vmag\nue} \pdv{\fonez}{\vmag} 
%  + \frac{4}{\vmag^2\nue}\Ez\fonez 
%  + 4\pi\pdv{\fM}{\vmag}
%  , \nonumber\\
%  \vmag\frac{\nue}{2}\pdv{\fonez}{\vmag} 
%  &=&  
%  \frac{\Ez}{3}\pdv{\fzero}{\vmag} + \frac{\vmag}{3}\pdv{\fzero}{z} 
%  + \left(\nuei + \frac{\nue}{2}\right)\fonez
%  , \nonumber \label{eq:OOE_P1f1}
%\end{eqnarray}


%\begin{table}
%\begin{center}
%  \begin{tabular}{c|ccccc}
%    \hline\hline\\
%    Kn$^e$ & $\,\,10^{-3}\,\,$ & $\,\,5\times10^{-3}\,\,$ & $\,\,10^{-2}\,\,$ & $\,\,5\times10^{-2}\,\,$ & $\,\,10^{-1}\,\,$ \\\\
%    \hline\\
%    $\vmag_{lim}^{Z=1} / \vth$ & 21.6 & 9.8 & 7.0 & 3.8 & 3.1 \\\\
%    \hline\\
%    $\vmag_{lim}^{Z=2} / \vth$ & 14.8 & 6.8 & 5.0 & 3.1 & 2.6 \\\\
%    \hline\\
%    $\vmag_{lim}^{Z=10} / \vth$ & 6.7 & 3.4 & 2.6 & 1.6 & 1.3 \\\\
%    \hline\hline
%  \end{tabular}
%  \caption{
%  $\sqrt{3}\vmag\frac{\nue}{2} > |\tE|$.
%  }
%\end{center}
%\label{tab:vlim}
%\end{table}

\section{Benchmarking the~AWBS nonlocal transport model}
\label{sec:BenchmarkingAWBS}
After having shown several encouraging properties of the~AWBS transport 
equation defined by \eqref{eq:AWBS_model} under local diffusive conditions
in \secref{sec:DiffusiveKinetics}, this section provides a~broader analysis
of the~electron transport and focuses on analysis its behavior under variety of
conditions in plasmas. In principle, this is characterized by allowing that
electron mean free path can be arbitrarily long, which leads to so-called 
nonlocal electron transport extensively investigated in numerous publications 
\cite{Malone_1975_15, Colombant_PoP2005, Bell_1981_83, LMV_1983_7, Brantov_Nonlocal_electron_transport_1998, schurtz2000, Sorbo_2015}, where the~Fokker-Planck
modeling of electrons in plasma represents the~essential tool. Being so, 
we introduce our implementation of the~AWBS transport equation called AP1,
where its results are further benchmarked against simulation results
provided by Aladin, Impact VFP codes, and Calder a~collisional Particle-In-Cell
code. Their description follows in the~next section.

%\begin{figure}[tbh]
%  \begin{center}
%    \begin{tabular}{c}
%      \includegraphics[width=\figscale\textwidth]{../VFPdata/C7_Impact_case3_heatflux.png} \\
%      \includegraphics[width=\figscale\textwidth]{../VFPdata/C7_Impact_case3_kinetics.png}
%    \end{tabular}
%  \caption{  
%  Snapshot 12 ps. Left: correct steady solution of heat flux. 
%  Right: correct comparison to kinetic profiles at point 437 $\mu$m by Impact.
%  Velocity limit 4.0 $\vth$.
%  }
%  \label{fig:C7_Impact_case3}
%  \end{center} 
%\end{figure}

%\subsection{Aladin, Impact, and Calder kinetic codes}
%\label{sec:AladinImpactCaldercodes}

%%% CEA contribution starts

\subsection{Calder PIC code}
In the limit of classical physics, a fluid description of the particle phase-space, including small angle binary collisions, can be described with the Maxwell equations:

\begin{eqnarray}
&&\nabla\cdot \mathbf{E}=\sum_\alpha \frac{q_\alpha}{\varepsilon_0} n_{\alpha},\\
&&\nabla\cdot \mathbf{B}=0,\\
&&\nabla\wedge \mathbf{E}+\frac{\partial \mathbf{B}}{\partial t}=0,\\
&&\nabla\wedge \mathbf{B}-\mu_0\varepsilon_0\frac{\partial \mathbf{E}}{\partial t}=\mu_0\sum_\alpha\mathbf{j}_{\alpha},\label{maxw}
\end{eqnarray}
coupled with the ion and electron Vlasov equations with the Landau-Beliaev-Budker collisions integral [14, and you can cite beliaev and budker 1956]. The latter is the relativistic version of the collision operator introduced in Eq. (2) and is written (it would be better to write Eq. (2) in its conservative forme and its non relativistic kernel U):
\begin{widetext}
\begin{eqnarray}
&&\frac{\partial f_\alpha}{\partial t}+\mathbf{v}\cdot\nabla_{\mathbf{x}}f_\alpha+q_\alpha\left(\mathbf{E}+\mathbf{v}\wedge\mathbf{B}\right)\nabla_{\mathbf{p}}f_\alpha=C_{LBB}(f_\alpha,f_\alpha)+\sum_\beta C_{LBB}(f_\alpha,f_\beta),\\
&&C_{LBB}(f_\alpha,f_\beta)=-\frac{\partial}{\partial \mathbf{p}}\cdot\frac{\Gamma_{\alpha\beta}}{2}\left[\int \mathbf{U}(\mathbf{p},\mathbf{p}^\prime)\cdot(f_\alpha\nabla_{\mathbf{p}^\prime}f_\beta^\prime-f_\beta^\prime\nabla_{\mathbf{p}}f_\alpha)\right]d^3\mathbf{p}^\prime,\\
&&\mathbf{U}(\mathbf{p},\mathbf{p}^\prime)=\frac{r^2/\gamma\gamma^\prime}{(r^2-1)^{3/2}}\left[(r^2-1)\mathbf{I}-\mathbf{p}\otimes\mathbf{p}-\mathbf{p}^\prime\otimes\mathbf{p}^\prime+r(\mathbf{p}\otimes\mathbf{p}^\prime+\mathbf{p}^\prime\otimes\mathbf{p})\right],
\end{eqnarray}
\end{widetext}
with $\gamma=\sqrt{1+\mathbf{p}^2}$, $\gamma^\prime=\sqrt{1+\mathbf{p}^{\prime 2}}$ and $r=\gamma\gamma^\prime-\mathbf{p}\cdot\mathbf{p}^\prime$. The momemtum $\mathbf{p}_\alpha$ ($\mathbf{p}_\beta$) is normalized to $m_\alpha c$ (resp. $m_\beta c$). Obviously, this collision operator tends to Eq. (2) in the non-relativistic limit.

The aforementioned set of Eqs. in 3D is solved by the PIC code CALDER. (cite lefebvre et al., nucl. fus. 43, 629, 2003 and Perez et al. pop 19,083104, 2012).

\subsection{P$_1$ model: Impact and Aladin}

The PIC code is extremely expensive as the collisions require the description of the velocity space in 3 dimensions. Yet, a reduction of dimensions can be done by developing the distribution function in a cartesian tensor series, equivalent to a serie along the spherical harmonics (cite johnston 1960) as follows:
\begin{equation}
f(t,\mathbf{x},\mathbf{v})=f_0(t,\mathbf{x},v)+\Omega\cdot \mathbf{f}_1(t,\mathbf{x},v)+\Omega\otimes\Omega:\mathbf{f}_2(t,\mathbf{x},v)+... \label{Pn},
\end{equation}
where $v=|\mathbf{v}|$, $\Omega=\mathbf{v}/v$. A P$_n$ model refers to neglecting orders higher than $n$.
The distribution function approximation $f(t,\mathbf{x},\mathbf{v})\approx f_0(t,\mathbf{x},v)+\Omega\cdot \mathbf{f}_1(t,\mathbf{x},v)$ coupled with the Vlasov-Landau Eq. (cite Eq. (2)), leads to the P$_1$ model (cite johnston 1960 and kingham 2004):
\begin{widetext}
\begin{eqnarray}
&&\frac{\partial f_0}{\partial t}+\frac{v}{3}\nabla_{\mathbf{x}}\cdot f_1-\frac{e}{3m_ev^2}\frac{\partial}{\partial v}(v^2\mathbf{E}\cdot \mathbf{f}_1)=\frac{\Gamma_{ee}}{v^2}\frac{\partial}{\partial v}\left[C(f_0)f_0+D(f_0)\frac{\partial f_0}{\partial v}\right],\\
&&\frac{\partial \mathbf{f}_1}{\partial t}+v\nabla_{\mathbf{x}}f_0-\frac{e\mathbf{E}}{m_e}\frac{\partial f_0}{\partial v}-\frac{e\mathbf{B}}{m_e}\wedge \mathbf{f}_1=-\Gamma_{ei}\frac{n_i}{v^3}\mathbf{f}_1.\\ 
&&C[f_0(v)]=4\pi\int_0^vf_0(u)u^2du,\\
&&D[f_0(v)]=\frac{4\pi}{v}\int_0^vu^2\int_u^\infty wf_0(w)dwdu.\label{P1}
\end{eqnarray}
\end{widetext}



The electron and energy densities are obtained from the isotropic part of the distribution function $f_0$:
\begin{eqnarray}
n_e=4\pi\int_{\mathbf{R}^+} f_0v^2dv,\\
U_e=2\pi m_e\int_{\mathbf{R}^+} f_0v^4dv,
\end{eqnarray}
and the electron current and heat flux are obtained from the anisotropic part of the distribution function $\mathbf{f}_1$:
\begin{eqnarray}
\mathbf{j}_e=-\frac{4\pi e}{3}\int_{\mathbf{R}^+} \mathbf{f}_1v^3dv,\\
\mathbf{Q}_e=\frac{2\pi m_e}{3}\int_{\mathbf{R}^+} \mathbf{f}_1v^5dv.
\end{eqnarray}

Impact and Aladin solve the system of Eqs. \eqref{P1} with the Maxwell equations  \eqref{maxw} in two dimensions, assuming motionless ions.

%%% CEA contribution ends

\begin{itemize}
  \item Brief description of the Aladin code \figref{fig:C7_Aladin_case5}, \figref{fig:C7_Aladin_case3}. %\figref{fig:C7_Aladin_case6}
\end{itemize}

%\begin{itemize}
%  \item Brief description of the Impact code \figref{fig:C7_Impact_case3}.
%\end{itemize}

\begin{figure}[tbh]
  \begin{center}
    \begin{tabular}{c}
      \includegraphics[width=\figscale\textwidth]{../VFPdata/C7_Aladin_case3_heatflux.png} \\
      \includegraphics[width=\figscale\textwidth]{../VFPdata/C7_Aladin_case3_kinetics.png} \\
      \includegraphics[width=\figscale\textwidth]{../VFPdata/C7_Aladin_case3_nonlocal_kinetics.png}  
    \end{tabular}
  \caption{  
  Snapshot 12 ps. Top: correct steady solution of heat flux.  
  Middle: correct comparison to kinetic profiles at point 442 $\mu$m by Aladin. 
  Velocity limit 3.4 $\vth$.
  Bottom: correct comparison to kinetic profiles at point 550 $\mu$m by Aladin.
  Velocity limit 8.8 $\vth$
  }
  \label{fig:C7_Aladin_case3}
  \end{center} 
\end{figure}

\begin{figure}[tbh]
  \begin{center}
    \begin{tabular}{c}
      \includegraphics[width=\figscale\textwidth]{../VFPdata/C7_Calder_case1_heatflux.png} 
	  \\ 
	  \includegraphics[width=\figscale\textwidth]{../VFPdata/C7_Calder_case1_kinetics.png}
	  \\ 
	  \includegraphics[width=\figscale\textwidth]{../VFPdata/C7_Calder_case1_nonlocal_kinetics.png}
    \end{tabular}
  \caption{  
  Snapshot 11 ps. Left: correct steady solution of heat flux. 
  %Right: AP1 kinetic profiles at point 750~$\mu$m corresponding to 
  %a~highly nonlocal nature of the~heat flux and is in a~good agreement with
  %\cite{Sherlock_PoP2017}. Velocity $max(q_1)$ = 5.8 $\vth$. 
  Velocity limit 6.4 $\vth$.
  }
  \label{fig:C7_Calder_case1}
  \end{center} 
\end{figure}

\begin{itemize}
  \item Brief description of the Calder code \figref{fig:C7_Calder_case1}.
\end{itemize}

%\subsection{Tests relevant to laser-heated plasmas}
%\label{sec:SimulationResults}

Among a~variety of test suitable for benchmarking the~nonlocal electron 
transport models published 
\cite{Epperlein_PoFB1991, marocchino2013, Sorbo_2015, 
Sorbo_2016, Sherlock_PoP2017, Brodrick_PoP2017}, we decided to focus on 
conditions relevant to inertial confinement fusion plasmas generated by lasers.

\subsection{Heat-bath problem}  
\label{sec:heatbath_test}

%\begin{figure}[tbh]
%  \begin{center}
%    \begin{tabular}{c}
%      \includegraphics[width=\figscale\textwidth]{../VFPdata/C7_Aladin_case5_heatflux.png} \\
%      \includegraphics[width=\figscale\textwidth]{../VFPdata/C7_Aladin_case5_kinetics.png}
%    \end{tabular}
%  \caption{  
%  Snapshot 20 ps. Left: correct steady solution of heat flux. 
%  Right: Aladins results are correct. Velocity limit 4.4 $\vth$..
%  }
%  \label{fig:C7_Aladin_case5}
%  \end{center} 
%\end{figure}
\begin{figure}[tbh]
  \begin{center}
    \begin{tabular}{c}
      \includegraphics[width=\figscale\textwidth]{../VFPdata/C7_Aladin_case5_heatflux.png} \\
      \includegraphics[width=\figscale\textwidth]{../VFPdata/C7_Aladin_case5_kinetics.png} \\
	  \includegraphics[width=\figscale\textwidth]{../VFPdata/C7_Aladin_case5_nonlocal_kinetics.png}
    \end{tabular}
  \caption{  
  Snapshot 20 ps. Top: correct steady solution of heat flux. 
  Right: Aladins results are correct. Velocity limit 4.4 $\vth$.
  Snapshot 20 ps. AP1 kinetic profiles at point 580~$\mu$m corresponding to 
  a~highly nonlocal nature of the~heat flux %\figref{fig:C7_Aladin_case5} 
  and is in a~good agreement with
  \cite{Sherlock_PoP2017}. Velocity $max(q_1)$ = 6.0 $\vth$. 
  Velocity limit 9.0 $\vth$.
  }
  \label{fig:C7_Aladin_case5}
  %\label{fig:C7_Aladin_case5_nonlocal}
  \end{center} 
\end{figure}

The accuracy of the AP1 implementation is compared to Aladin, Impact and Calder
codes by calculating the heat flow in the case
where a~large relative temperature variation
\begin{equation}
  T_e(z) = 0.575 - 0.425 \tanh\left((z-450) s\right) ,
  \label{eq:T_init}
\end{equation}
which exhibits a~smooth steep gradient at point 450~$\mu$m 
connecting a~hot bath ($T_e = 1$~keV) 
and cold bath ($T_e = 0.17$~keV) and $s$ is the~parameter of steepness. 
This test is referred to as a~simple non-linear heat-bath problem and
originally was introduced in \cite{marocchino2013} and further investigated
in  \cite{Sorbo_2015, Sorbo_2016, Sherlock_PoP2017, Brodrick_PoP2017}.

%Aladin and Impact simulations showed an evolution of the heat flow
%from the local (due to initialising as a Maxwellian) to the
%nonlocal, with a reduced peak, over an initial transient
%phase (over which the temperature ramp flattened somewhat). 
%The transient phase was considered over when the
%ratio of the VFP heat flow to the expected local heat
%flow stopped decreasing. After the transient phase this
%ratio begins to slowly increase as the thermal conduction flattens 
%the temperature ramp and the ratio of the scalelength to mfp increases 
%(i.e. the thermal transport slowly becomes more local). 

The~total computational box size is 700 $\mu$m in the~case
of Aladin and Impact and 1000 $\mu$m in the~case of Calder.
We performed Aladin, Impact, and Calder simulations showing an~evolution of
temperature starting from the~initial profile \eqref{eq:T_init}. 
Due to an~inexact initial distribution function (approximated by Maxwellian),
the~first phase of the~simulation exhibits a~transient behavior of the~heat
flux. After several ps the~distribution adjusts properly to its nonlocal nature
and the~actual heat flux profile can be compared. 
We then take the temperature profile from Aladin/Impact/Calder and used 
our AP1 implementation to calculate the heat flow
it would predict given this profile. In~particular, the~AP1 results 
corresponding to the~evolved temperature profile by Aladin can be found
in \figref{fig:C7_Aladin_case5} and \figref{fig:C7_Aladin_case3} for 
$\Zbar = 1$ and $\Zbar = 10$ respectively. The~AP1 results computed on
the~evolved temperature profile for $\Zbar = 2$ 
%by Impact are shown in \figref{fig:C7_Impact_case3} and 
by Calder can be found in \figref{fig:C7_Calder_case1}.

For all heat-bath simulations the electron density, Coulomb logarithm and 
ionisation were kept constant and uniform.
The~coulomb logarithm was held fixed throughout, 
$\lnc = 7.09$.
Nevertheless, the~Knudsen number Kn$^e$ has been varied among the~simulation 
runs in order to address a~broad range of nonlocality of 
the~electron transport corresponding 
to the~laser-heated plasma conditions, i.e. Kn$^e \in (0.0001, 1)$. 
The~variation of Kn$^e$ arises from the~variation
of the~uniform electron density $n_e \in (10^{19}, 10^{23})$ cm$^{-3}$ or 
the~length scale given by the~slope of the~temperature profile 
$s \in (1/2500, 1/25) \mu$m. Results of an~extensive set of simulations of
varying Kn$^e$ is shown in \figref{fig:Kn_results}.
 
\begin{figure}[tbh]
  \begin{center}
    \begin{tabular}{c}
      \includegraphics[width=\figscale\textwidth]{Kn_results.png}
    \end{tabular}
  \caption{  
  Simulation results for the case $Z=2$ computed by AP1/Aladin/Impact/Calder.
  Every point corresponds to the maximum heat flux in a "tanh" temperature 
  simulation, which can be characterized by Kn. The range of 
  $\log_{10}(\text{Kn})\in (0, -4)$ can be expressed as equivalent 
  to the~electron density approximate range n$_e \in (1e19, 3.5e22)$ of 
  the~50 $\mu$m slope tanh case. In the case of Kn = 0.56, 
  $q_{Aladin} / q_{AP1}\approx 7.9$.}
  \label{fig:Kn_results}
  \end{center} 
\end{figure}

Apart from the~distribution function details related to the~point of 
the~heat flux maximum, in \figref{fig:C7_Aladin_case5}
we also present the~detail of the~kinetic profile at point 580~$\mu$m 
corresponding to a~highly nonlocal nature of the~heat flux profile. 
Here a~good agreement with \cite{Sherlock_PoP2017} can be found.

When analyzing the~results of \figref{fig:Kn_results} we have found 
an~interesting observation related to stopping effect of electrons.
It turns out, that the~force acting on electrons is dominated by electric field
above some velocity limit $\vmag_{lim}$ and this limit drops down significantly
when the~plasma conditions are more nonlocal, i.e. with increasing Knudsen 
number as can be seen in \tabref{tab:vlim}.

\begin{table}
\begin{center}
  \begin{tabular}{c|ccccc}
    \hline\hline\\
    %Kn$^e$ & $10^{-4}$ & $10^{-3}$ & $10^{-2}$ & $10^{-1}$ & $1$ \\\\
    Kn$^e$ & $\,\,10^{-4}\,\,$ & $\,\,10^{-3}\,\,$ & $\,\,10^{-2}\,\,$ & $\,\,10^{-1}\,\,$ & $\,\,1\,\,$ \\\\
    \hline\\
    $\vmag_{lim} / \vth$ & 70.8 & 22.4 & 7.3 & 3.1 & 1.8\\\\
    \hline\hline
  \end{tabular}
  \caption{
  Scan over varying nonlocality (Kn$^e$) showing the~limit of 
  the~collision friction dominance over the~acceleration of electrons 
  due to the~electric field force. The~electric field effect is dominant
  for electrons with higher velocity than $\vmag_{lim}$ defined in 
  \eqref{eq:v_limit}. Kn$^e$ and $\vth$ are evaluated from the~same 
  plasma profiles.
  %$\sqrt{3}\vmag\frac{\me}{2\qe}\nue > |\E|$.
  }
\label{tab:vlim}
\end{center}
\end{table}

For practical reasons we present in \tabref{tab:vlim} 
some explicit values of velocity limit corresponding to varying transport 
conditions expressed in terms of "$\Zbar$" Knudsen number 
$\text{Kn}^e = \frac{\mfpe}{\sqrt{\Zbar + 1}L_{T_e}}$, 
where $\sqrt{\Zbar + 1}$ provides a~proper scaling of nonlocality with respect
to ionization, i.e. the~effect of scattering of electrons on ions 
\cite{LMV_1983_7}.

In every simulation run of AP1 we used 250~velocity groups in order to avoid
numerical errors in modeling of the~electron kinetics. However, a~smaller 
number of groups, e.g. 50, provides a~very similar results 
(an~error around 10$\%$ in the~heat flux).

\subsection{Hohlraum problem}
%While comparisons between the AP1 model and VFP
%codes have previously been performed 8,45 , none have included 
%spatially-inhomogeneous ionisation. 
Additionally to the~steep temperature gradients, the~laser-heated plasma 
experiments also involve steep density gradients and variation in ionization,
which is even more dominant in multi-material targets as in inertial
fusion experiments, e.g. at the interface between the helium gas-fill and 
the ablated high $\Zbar$ plasma.
%, it is critical that the AP1 model be tested 
%in such an environment.

In~\cite{Brodrick_PoP2017}, a~kinetic simulation of such a~test was introduced.
Plasma profiles provided by a~HYDRA simulation in 1D spherical
geometry of a~laser-heated gadolinium hohlraum containing a~typical helium 
gas-fill were used as input for the~IMPACT \cite{Kingham_JCP2004} VFP code. 
Electron temperature $T_e$, electron density $n_e$ and ionisation $\Zbar$ 
profiles shown in \figref{fig:Gd_VFP_10ps_heatflux} at 20 nanoseconds of 
the~HYDRA simulation were used (after spline smoothing) as 
the~initial conditions for the~IMPACT run (in planar geometry). 
For simplicity, the Coulomb logarithm was treated as a
constant $\lnc_{ei}$ = $\lnc_{ee}$ = 2.1484. In reality, in the~low-density 
corona $\lnc$ reaches 8, which, however, does not affect the~heat flux profile 
significantly. 
%Note that in reality
%5the plasma is only strongly coupled in the colder region of
%the gadolinium bubble beyond $\sim$1.7 mm and $\lnc_{ei}\approx$ 8
%up to $\sim$1.6 mm in the hotter corona.

It is worth mentioning that in the~surroundings of the~heat flux maximum 
($\sim 1662 \mu$m) the~profiles of all plasma variables exhibit steep gradients 
with a change from $T_e$ = 2.5 keV, $n_e$ = 5$\times$10$^{20}$ cm$^{−3}$, 
$\Zbar$ = 2 to $T_e$ = 0.3 keV, $n_e$ = 6$\times$10$^{21}$ cm$^{−3}$ , 
$\Zbar$ = 44 across approximately 100 $\mu$m 
(between 1600~$\mu$m and 1700~$\mu$m), starting at the~helium-gadolinium 
interface.  

%Reflective boundary
%conditions were used here as in the previous section and
%IMPACT used a timestep of 1.334 fs. The $n_e$ and $\Zbar$ profiles did not 
%evolve in the IMPACT simulation due to the treatment of the electric field 
%discussed in section II that ensures quasineutrality and the neglection of 
%ion motion and ionisation models.

%As with the VFP simulations in the previous section,
%there is an initial transient phase where the IMPACT
%heat flux gradually reduces from the Braginskii prediction
%as the distribution function rapidly moves away from
%Maxwellian. Once again this transient phase is considered
%to be over when the ratio of the peak heat flow to the
%Braginskii prediction stops reducing. This ratio is not
%observed to change by more than 5\% after the first 5 ps
%of our 15.7 ps simulation. Therefore, we conclude that
%it safe to assume the transient phase is over after 5 ps,
%at which point the temperature front has advanced by
%approximately 8 $\mu$m leading to a maximum temperature
%change of 41\% as shown in \figref{fig:Gd_VFP_10ps_heatflux}.

%In comparing the IMPACT, AP1, and SNB heat flow profiles
%we encountered an important subtlety concerning the implementation of 
%the model. A~recent SNB model with separate electron
%ion and electron-electron collision frequencies provides a
%very good prediction of the preheat into the hohlraum, the
%peak heat flow to within 16\% and an improved estimate
%of the thermal conduction in the gas-fill region, the latter
%is still too large by a factor of $\sim$2. This discrepancy could
%potentially lead to an overestimate of hohlraum temperatures and thus cause 
%issues similar to those arising with
%using an overly restrictive flux limiter.

\begin{figure}[tbh]
  \begin{center}
    \begin{tabular}{c}
      \includegraphics[width=\figscale\textwidth]{../VFPdata/GD_Hohlraum/fluxes_10ps.png} 
    \end{tabular}
  \caption{
  }
  \label{fig:Gd_VFP_10ps_heatflux}
  \end{center} 
\end{figure}

%\begin{itemize}
%  \item Multiple runs analyzing the~performance of AP1 with respect to 
%    Aladin/Impact/Calder along wide range of Kn$^e$ shown in 
%    \figref{fig:Kn_results}.
%  \item Realistic hydro simulation setting provided by HYDRA, a~comparison
%    between AP1, Impact, and SNB shown in \figref{fig:Gd_VFP_10ps_heatflux}.
%  \item Comment on and summarize the~velocity limits for all figs.
%\end{itemize}

%


%\subsection{AP1 implementation}
\label{sec:C7code}

AP1 represents the~abbreviation AWBS-P1, i.e. the~use of collision operator 
\eqref{eq:AWBS_model} and the~P1 angular discretization, i.e. the~lowest order 
anisotropy approximation. AP1 in general belongs to the~so-called angular 
moments method and the~electron distribution function takes the~form
\begin{equation}
  \tilde{\ft} = \frac{\fzero}{4\pi} + \frac{3}{4\pi}\vn\cdot\fone , 
  \nonumber \label{eq:P1approximation}
\end{equation}
which consists of the~isotropic part $\fzero = \int_{4\pi} \tilde{\ft} \dI\vn$ 
and the~directional part $\fone = \int_{4\pi} \vn
\tilde{\ft} \dI\vn$, where $\vn$ is the~transport direction (the~solid angle).

The~first two angular moments applied to \eqref{eq:kinetic_equation}
with collision operator \eqref{eq:AWBS_model} lead to the~AP1 model equations
\begin{eqnarray}
  \vmag\frac{\nue}{2}\pdv{}{\vmag}\left(\fzero - 4\pi\fM \right) &=&
  \vmag\nabla\cdot\fone + \tE\cdot
  \pdv{\fone}{\vmag} + \frac{2}{\vmag}\tE\cdot\fone , 
  \label{eq:P1f0}\\
  \vmag\frac{\nue}{2}\pdv{\fone}{\vmag}
  - \nuscat\fone &=& 
  \frac{\vmag}{3}\nabla\fzero + 
  \frac{\tE}{3}\pdv{\fzero}{\vmag} ,
  \label{eq:P1f1}
\end{eqnarray}
where $\nuscat = \nuei + \frac{\nue}{2}$. The~strategy of solving 
\eqref{eq:P1f0} and \eqref{eq:P1f1} resides in integrating $\pdv{\fzero}{\vmag}$
and $\pdv{\fone}{\vmag}$ in velocity magnitude while starting the~integration
from infinite velocity to zero velocity, which corresponds to decelerating 
electrons. It should be noted, that in practice we start the~integration from
$\vmag = 7 \vth$, which represents a~sufficiently high velocity.

\subsubsection{Nonlocal electric field treatment}
\label{sec:Efield}

%\begin{multline}
%  %\frac{\vect{j}}{\qe} = 
%  \vect{q}_c \equiv
%  \qe \intv \Bigg[\frac{\frac{\nue}{2}\vmag^2}{\nuei + \frac{\nue}{2}}
%  \pdv{\fone}{\vmag} 
%  - \frac{\vmag^2}{3\left(\nuei + \frac{\nue}{2}\right)}
%  \nabla\fzero \\
%  - \frac{\vmag}{3\left(\nuei + \frac{\nue}{2}\right)}\pdv{\fzero}{\vmag}\tE
%  \Bigg] \vmag^2\, \dI\vmag = 0
%  , \nonumber
%\end{multline}
Similarly to \eqref{eq:BGK_Efield}, one can obtain the~model equation of 
the~electric field $\tE$ by evaluating the~zero current condition 
(a~velocity integration of \eqref{eq:P1f1})
\begin{equation}
  \intv \left(\frac{\frac{\nue}{2}\vmag^2}{\nuscat}
  \pdv{\fone}{\vmag} 
  - \frac{\vmag^2}{3\nuscat}
  \nabla\fzero 
  - \frac{\vmag}{3\nuscat}\pdv{\fzero}{\vmag}\tE
  \right) \vmag^2\, \dI\vmag = 0 ,
  \label{eq:AP1_Efield}
\end{equation}
from which it is easy to express $\tE$ once $\fzero$ and $\fone$ are known, or
in other words, the~integral-differential model equations need to be solved 
simultaneously, which is achieved by $k$-iteration of 
$\fzero^k(\tE^k), \fone^k(\tE^k)$, i.e. \eqref{eq:P1f0}, \eqref{eq:P1f1}, and 
$\tE^{k+1}(\fzero^k, \fone^k)$, i.e.  \eqref{eq:AP1_Efield}, until 
the~current evaluation \eqref{eq:AP1_Efield} converges to zero. In particular,
the~first iteration starts with $\tE=\vect{0}$ in \eqref{eq:P1f0} and 
\eqref{eq:P1f1}.
%\begin{eqnarray}
%  \pdv{\fzero}{\vmag} &=&
%  \frac{2}{\nue}\pdv{\fonez}{z} + \frac{2\tEz}{\vmag\nue} \pdv{\fonez}{\vmag} 
%  + \frac{4}{\vmag^2\nue}\tEz\fonez 
%  + 4\pi\pdv{\fM}{\vmag}
%  , \nonumber\\
%  \vmag\frac{\nue}{2}\pdv{\fonez}{\vmag} 
%  &=&  
%  \frac{\tEz}{3}\pdv{\fzero}{\vmag} + \frac{\vmag}{3}\pdv{\fzero}{z} 
%  + \left(\nuei + \frac{\nue}{2}\right)\fonez
%  , \nonumber \label{eq:OOE_P1f1}
%\end{eqnarray}

\begin{multline}
  %\frac{2}{3\vmag\nue} 
  %\left(\left(\sqrt{3}\vmag\frac{\nue}{2}\right)^2 - \tEz^2\right)  
  \left(\vmag\frac{\nue}{2} - \frac{2\tEz^2}{3\vmag\nue}\right) 
  \pdv{\fonez}{\vmag} 
  =\\
  \frac{2\tEz}{3\nue}\pdv{\fonez}{z}  
  + \frac{4\pi\tEz}{3}\pdv{\fM}{\vmag}
  + \frac{\vmag}{3}\pdv{\fzero}{z} 
  + \left(\frac{4\tEz^2}{3\vmag^2\nue}
  + \left(\nuei + \frac{\nue}{2}\right) \right)\fonez
  , \nonumber
\end{multline}

%\begin{multline}
%  \frac{2}{3\vmag\nue} 
%  \left(\left(\sqrt{3}\vmag\frac{\nue}{2}\right)^2 - \tEz^2\right)
%  \frac{\fonez^{n+1} - \fonez^n}{\Delta\vmag} 
%  =\\
%  \frac{2\tEz}{3\nue}\pdv{\fonez^{n+1}}{z}  
%  + \frac{4\pi\tEz}{3}\pdv{\fM^{n+1}}{\vmag}
%  + \frac{\vmag}{3}\pdv{\fzero^{n+1}}{z} 
%  + \left(\frac{4\tEz^2}{3\vmag^2\nue}
%  + \left(\nuei + \frac{\nue}{2}\right) \right)\fonez^{n+1}
%  , \nonumber
%\end{multline}

%\begin{table}
%\begin{center}
%  \begin{tabular}{c|ccccc}
%    \hline\hline\\
%    Kn$^e$ & $\,\,10^{-3}\,\,$ & $\,\,5\times10^{-3}\,\,$ & $\,\,10^{-2}\,\,$ & $\,\,5\times10^{-2}\,\,$ & $\,\,10^{-1}\,\,$ \\\\
%    \hline\\
%    $\vmag_{lim}^{Z=1} / \vth$ & 21.6 & 9.8 & 7.0 & 3.8 & 3.1 \\\\
%    \hline\\
%    $\vmag_{lim}^{Z=2} / \vth$ & 14.8 & 6.8 & 5.0 & 3.1 & 2.6 \\\\
%    \hline\\
%    $\vmag_{lim}^{Z=10} / \vth$ & 6.7 & 3.4 & 2.6 & 1.6 & 1.3 \\\\
%    \hline\hline
%  \end{tabular}
%  \caption{
%  $\sqrt{3}\vmag\frac{\nue}{2} > |\tE|$.
%  }
%\end{center}
%\label{tab:vlim}
%\end{table}

\begin{table}
\begin{center}
  \begin{tabular}{c|ccccc}
    \hline\hline\\
    %Kn$^e$ & $10^{-4}$ & $10^{-3}$ & $10^{-2}$ & $10^{-1}$ & $1$ \\\\
    Kn$^e$ & $\,\,10^{-4}\,\,$ & $\,\,10^{-3}\,\,$ & $\,\,10^{-2}\,\,$ & $\,\,10^{-1}\,\,$ & $\,\,1\,\,$ \\\\
    \hline\\
    $\vmag_{lim} / \vth$ & 70.8 & 22.4 & 7.3 & 3.1 & 1.8\\\\
    \hline\hline
  \end{tabular}
  \caption{
  $\sqrt{3}\vmag\frac{\nue}{2} > |\tE|$.
  }
\end{center}
\label{tab:vlim}
\end{table}

\begin{comment} % too complicated
\begin{equation}
  |\tE_{red}| = (1 + \alpha^E) \vmag \frac{\nue}{2} ,
  \label{eq:reducedEfield}
\end{equation}

\begin{eqnarray}
  |\tE_{red}| &=& |\tE_{d}| + |\tE_{iso}| ,
  \nonumber \\
  \vmag \frac{\nue}{2} + |\tE_{iso}| &=& |\tE_{d}| ,
  \nonumber 
\end{eqnarray}
where we define the~isotropic effect of E field as
$|\vect{E}_{iso}| = \vmag \nue^E$ by introducing the~effective collisional 
frequency $\nue^E$.

Since the~effect of the~original E field $\tE$ has been reduced in 
\eqref{eq:reducedEfield}, an~additional collision term
\begin{equation}
  \vmag \nuei^E = |\tE| - |\tE_{red}| ,
  \nonumber
\end{equation}
is added to scattering on ions. The~improved collision AWBS operator then takes 
the~following form
\begin{equation}
  C_{AWBS} = \vmag \left(\frac{1}{2} + \alpha^E\right)\nue \pdv{}{\vmag}
  \left(f - \fM \right) 
  + \left(\nuei + \nuei^E + \frac{\nue}{2}\right) (\fzero - f) ,
  \nonumber
\end{equation}
where both $\alpha^E$ and $\nuei^E$ apply only if 
$|\tE| > \sqrt{3} \vmag \frac{\nue}{2}$ 
and are set to zero otherwise.
\end{comment} % too comlicated

\begin{comment} % directional E field condition
Since in the~laser heated plasmas the~Knudsen number 
Kn$ = \frac{\vth}{\nu_t(\vth) L} \in (0, 1)$, i.e. the~collisionality in 
the~kinetics of electrons plays always an~important effect for thermal-like 
particles, the~electron distribution 
function can be treated as out-of-equilibrium approximation 
\begin{equation}
  f = \fM + \daf ,
  \label{eq:OOE_outofeq}
\end{equation}  
where the~consequent AWBS model reads
\begin{multline}
  \vmag\vn\cdot\nabla (\fM + \daf) + \tE\cdot\vn \left(\pdv{\fM}{\vmag} 
  + \pdv{\daf}{\vmag}\right) 
  + \frac{\tE\cdot\vect{e}_\theta}{\vmag}\pdv{\daf}{\theta}
  = \\
  \vmag \frac{\nue}{2} \pdv{\daf}{\vmag} 
  + \left(\nuei + \frac{\nue}{2}\right) (\fzero - (\fM + \daf)) ,
  \label{eq:OOE_AWBS_model}
\end{multline}
or its 1D version
\begin{multline}
  \vmag\mu\pdv{}{z}(\fM + \daf) + \tEz\mu\left(\pdv{\fM}{\vmag} 
  + \pdv{\daf}{\vmag}\right) 
  + \frac{\tEz(1-\mu^2)}{\vmag}\pdv{\daf}{\mu}
  = \\
  \vmag \frac{\nue}{2} \pdv{\daf}{\vmag} 
  + \left(\nuei + \frac{\nue}{2}\right) (\fzero - (\fM + \daf)) ,
  \label{eq:OOE_AWBS_model_1D}
\end{multline}
where $\tE\cdot\vect{e}_\theta = \tEz\sin(\theta)$ and $\pdv{}{\theta} = \sin(\theta)\pdv{}{\mu}$, $\mu = \cos(\theta)$.
\begin{multline}
  \left(\vmag \frac{\nue}{2} - \tEz\mu\right) \pdv{\daf}{\vmag} = 
  \vmag\mu\pdv{\daf}{z} + \vmag\mu\pdv{\fM}{z} + \tEz\mu\pdv{\fM}{\vmag} \\ 
  + \frac{\tEz(1-\mu^2)}{\vmag}\pdv{\daf}{\mu}
  - \left(\nuei + \frac{\nue}{2}\right) (\fzero - (\fM + \daf))
  ,
  \nonumber
\end{multline}
we adopt $\daf(\vmag, \mu) = \dafzero(\vmag) + \mu\dafone(\vmag)$, 
which leads to
\begin{multline}
  \left(\vmag \frac{\nue}{2} - \tEz\mu\right) \pdv{}{\vmag}
  (\dafzero + \mu\dafone) = 
  \vmag\mu\pdv{}{z}(\dafzero + \mu\dafone) + \vmag\mu\pdv{\fM}{z} 
  + \tEz\mu\pdv{\fM}{\vmag} \\ 
  + \frac{\tEz(1-\mu^2)}{\vmag}\dafone
  + \left(\nuei + \frac{\nue}{2}\right) \mu\dafone
  ,
  \nonumber
\end{multline}
\begin{eqnarray}
  \vmag \frac{\nue}{2}\pdv{\dafzero}{\vmag}
  - \tEz\mu^2 \pdv{\dafone}{\vmag}
  &=& 
  \vmag\mu^2\pdv{\dafone}{z}
  + \frac{\tEz(1-\mu^2)}{\vmag}\dafone
  , \nonumber \\
  \mu\vmag \frac{\nue}{2}\pdv{\dafone}{\vmag} - \tEz\mu \pdv{\dafzero}{\vmag}
  &=& 
  \vmag\mu\pdv{\dafzero}{z}
  + \vmag\mu\pdv{\fM}{z} + \tEz\mu\pdv{\fM}{\vmag}  
  + \left(\nuei + \frac{\nue}{2}\right) \mu\dafone
  ,
  \nonumber
\end{eqnarray}
\end{comment} % directional E field condition

\begin{equation}
  |\tE_{red}| = \sqrt{3} \vmag\frac{\nue}{2} ,
  \label{eq:Elimit}
\end{equation}
\begin{equation}
  \omega_{red} = |\tE_{red}| / |\tE| ,\quad 
  \nuscat^E = \frac{|\tE| - |\tE_{red}|}{\vmag} .
  \nonumber
\end{equation}

P1 approximation equivalent
\begin{equation}
  \tilde{f} = \frac{4\pi \fM + \dafzero}{4\pi} + \frac{3}{4\pi}\vn\cdot\fone .
  \label{eq:OOE_P1outofeq}
\end{equation}
where the~moment model reads
\begin{eqnarray}
  \vmag \frac{\nue}{2}\pdv{\dafzero}{\vmag} &=&
  \vmag\nabla\cdot\fone + \tE\cdot\left(\omega_{red} \pdv{\fone}{\vmag} 
  + \frac{2}{\vmag}\fone\right) , 
  \nonumber\\
  \vmag\frac{\nue}{2}\pdv{\fone}{\vmag} 
  &=& \tnuscat\fone 
  + \frac{\vmag}{3}\nabla\left(4\pi\fM + \dafzero\right)
  \nonumber \\
  && 
  + \frac{\tE}{3}\left(4\pi \pdv{\fM}{\vmag} 
  + \omega_{red} \pdv{\dafzero}{\vmag} 
  \right) ,
  \nonumber
\end{eqnarray}
where $\tnuscat = \nuei + \nuscat^E + \frac{\nue}{2}$.
\begin{equation}
  \tE =
  \frac{\intv \left(\frac{\nue}{2\tnuscat}\vmag^2\pdv{\fone}{\vmag}
  - \frac{\vmag^2}{3\tnuscat}
  \nabla\left(4\pi\fM + \dafzero\right)\right) \vmag^2\, \dI\vmag}
  {\intv \frac{\vmag}
  {3\tnuscat}
  \left(4\pi\pdv{\fM}{\vmag} + \omega_{red} \pdv{\dafzero}{\vmag}\right)
  \vmag^2\, \dI\vmag} ,
  \nonumber
\end{equation}


%\subsection{Aladin, Impact, and Calder kinetic codes}
\label{sec:AladinImpactCaldercodes}

\begin{figure}[tbh]
  \begin{center}
    \begin{tabular}{c}
      \includegraphics[width=\figscale\textwidth]{../VFPdata/C7_Aladin_case6_heatflux.png} \\
      \includegraphics[width=\figscale\textwidth]{../VFPdata/C7_Aladin_case6_kinetics.png}
    \end{tabular}
  \caption{  
  Snapshot 20 ps. Left: correct steady solution of heat flux. 
  Right: Aladins results are need to added.
  }
  \end{center}
  \label{fig:C7_Aladin_case3}
\end{figure}


\begin{figure}[tbh]
  \begin{center}
    \begin{tabular}{c}
      \includegraphics[width=\figscale\textwidth]{../VFPdata/C7_Aladin_case3_heatflux.png} \\
      \includegraphics[width=\figscale\textwidth]{../VFPdata/C7_Aladin_case3_kinetics.png}
    \end{tabular}
  \caption{  
  Snapshot 12 ps. Left: correct steady solution of heat flux. Right: correct comparison to kinetic profiles at point 442 $\mu$m by Aladin.
  }
  \end{center}
  \label{fig:C7_Aladin_case3}
\end{figure}

\begin{figure}[tbh]
  \begin{center}
    \begin{tabular}{c}
      \includegraphics[width=\figscale\textwidth]{../VFPdata/C7_Impact_case3_heatflux.png} \\
      \includegraphics[width=\figscale\textwidth]{../VFPdata/C7_Impact_case3_kinetics.png}
    \end{tabular}
  \caption{  
  Snapshot 12 ps. Left: correct steady solution of heat flux. Right: correct comparison to kinetic profiles at point 437 $\mu$m by Impact.}
  \end{center}
  \label{fig:C7_Impact_case3}
\end{figure}

\begin{figure}[tbh]
  \begin{center}
    \begin{tabular}{c}
      \includegraphics[width=\figscale\textwidth]{../VFPdata/C7_Calder_case1_heatflux.png} \\ 
      \includegraphics[width=\figscale\textwidth]{../VFPdata/C7_Calder_case1_kinetics.png}
    \end{tabular}
  \caption{  
  Snapshot 11 ps. Left: correct steady solution of heat flux. Right: Kinetic profiles at point of maximum flux by AP1. Kinetics profiles by CALDER should be added.
  }
  \end{center}
  \label{fig:C7_CALDER_case1}
\end{figure}


%\subsection{Simulation results}
\label{sec:SimulationResults}

\begin{figure}[tbh]
  \begin{center}
    \begin{tabular}{c}
      \includegraphics[width=0.5\textwidth]{Kn_results.png}
    \end{tabular}
  \caption{  
  Simulation results for the case $Z=2$ computed by AP1/Aladin/Impact/Calder.
  Every point corresponds to the maximum heat flux in a "tanh" temperature 
  simulation, which can be characterized by Kn. The range of 
  $\log_{10}(\text{Kn})\in (0, -4)$ can be expressed as equivalent 
  to the~electron density approximate range n$_e \in (1e19, 3.5e22)$ of 
  the~50 $\mu$m slope tanh case. In the case of Kn = 0.56, 
  $q_{Aladin} / q_{AP1}\approx 7.9$.}
  \end{center}
  \label{fig:Kn_results}
\end{figure}

\begin{figure}[tbh]
  \begin{center}
    \begin{tabular}{c}
      \includegraphics[width=0.5\textwidth]{../VFPdata/GD_Hohlraum/fluxes_10ps.png} 
    \end{tabular}
  \caption{
  }
  \end{center}
  \label{fig:Gd_VFP_10ps_heatflux}
\end{figure}
%\clearpage

\section{Conclusions}
\label{sec:Conclusions}

\begin{itemize}
  \item The~most important point is that we introduce a~collision operator, 
    which is coherent with the full FP, i.e. no extra dependence on $\Zbar$.
  \item Touch pros/contras of linearized FP in Aladin and Impact vs AWBS
  \item Raise discussion about what is the weakest point of AP1 for high Kns: 
    the~velocity limit or phenomenological Maxwellization?
  \item Summarize useful outcomes related to plasma physics as 
    the~tendency of the~velocity maximum in $q_1$ with respect to $\Zbar$ and
	Kn$^e$.
  \item Emphasize the~good results of Aladin (compared to Impact) and also
    outstanding results of Calder while being PIC. 
\end{itemize}


\begin{acknowledgments}
\end{acknowledgments}

\appendix
\section{Background of the local diffusive regime theory}
\label{app:DiffusiveKinetics}

The~left hand side of \eqref{eq:1D_kinetic_equation} acts on 
\eqref{eq:f_approximation} as
\begin{multline}
  \mu\left(\pdv{\tilde{\ft}}{z} 
  + \frac{\qe\Ez}{\me\vmag}\pdv{\tilde{\ft}}{\vmag}\right) 
  + \frac{\qe\Ez}{\me}\frac{1-\mu^2}{\vmag^2}\pdv{\tilde{\ft}}{\mu} = \\
  \mu\left(\pdv{\ft^0}{z} + \frac{\qe\Ez}{\me\vmag}\pdv{\ft^0}{\vmag}\right) 
  + \frac{\qe\Ez}{\me\vmag^2} \ft^1 + O(\mu^2) .
  \label{app_eq:LHS_kinetic_equation}
\end{multline}
The~action on \eqref{eq:f_approximation} of the~BGK operator 
\eqref{eq:BGK_model_1D} as used in \eqref{eq:1D_kinetic_equation} reads
\begin{eqnarray}
  \frac{1}{\vmag}C_{BGK}(\tilde{\ft})
  &=&
  \frac{\tilde{\ft} - \fM}{\mfpe}
  + \frac{1}{2}\left(\frac{\Zbar}{\mfpe} + \frac{1}{\mfpe}\right)
  \pdv{}{\mu}(1 - \mu^2)\pdv{\tilde{\ft}}{\mu} ,\nonumber\\
  &=&  \frac{\ft^0 - \fM}{\mfpe}
  - \mu \frac{\Zbar}{\mfpe}\ft^1 .
  \label{app_eq:BGK_model_1D}
\end{eqnarray}
Consequently, if the~isotropic and anisotropic parts of 
\eqref{app_eq:LHS_kinetic_equation} and \eqref{app_eq:BGK_model_1D} are 
compared, one finds the~following equations 
\begin{eqnarray}
  \ft^0 &=& \fM + \frac{\mfpe\qe\Ez}{\me\vmag^2}f^1 ,
  \label{app_eq:BGK_f0} \\
  \ft^1 &=& - \frac{\mfpe}{\Zbar}
  \left( \pdv{\ft^0}{z} + \frac{\qe\Ez}{\me\vmag}\pdv{\ft^0}{\vmag} \right) . 
  \label{app_eq:BGK_f1}
\end{eqnarray}
It is valid to assume that $\ft^0 = \fM$ from \eqref{app_eq:BGK_f0}. Then,
\begin{equation}
  \ft^1_{BGK} = - \frac{\mfpe}{\Zbar}
  \left( \pdv{\fM}{z} + \frac{\qe\Ez}{\me\vmag}\pdv{\fM}{\vmag} \right) . 
  \label{app_eq:BGK_f1_fM}
\end{equation}
The~\textit{quasi-neutrality} constraint, corresponding to a~zero current 
imposed by the~electric field reads
\begin{equation}
\vect{j} \equiv \qe \int \vv \tilde{\ft} \, \dI\vv = \vect{0} .
\end{equation}
In the~case of the BGK EDF, in particular its~anisotropic part 
\eqref{app_eq:BGK_f1_fM}, the~zero current condition takes the~form
\begin{equation}
  2\pi\int_{-1}^{1} \int_{\vmag} \vmag \mu^2 \ft^1_{BGK}\, \dI\vmag\dI\mu = 0 ,
  \nonumber
\end{equation}
which leads to the~electric field (same as the~classical Lorentz electric field
$\E_L$ \cite{Lorentz_1905})
\begin{equation}
  \Ez = \frac{\me\vth^2}{\qe}\left(\frac{1}{L_{n_e}} 
  + \frac{5}{2}\frac{1}{L_{T_e}} \right) .
  \label{app_eq:BGK_Efield}
\end{equation}
It is worth mentioning, that the~deviation of $\ft^0$ from $\fM$ in 
\eqref{app_eq:BGK_f0} can be written as $\left(\frac{\mfpe}{L_{n_e}} 
  + \frac{5}{2}\frac{\mfpe}{L_{T_e}} \right)\frac{\vth^2}{\vmag^2}f^1$,
where naturally arises the~Knudsen number 
$Kn = \frac{\mfpe}{L_{n_e}} + \frac{5}{2}\frac{\mfpe}{L_{T_e}}$ comprising
both contributions of electron density and temperature gradients.

In the~case of the~AWBS operator \eqref{eq:AWBS_model} used in 
\eqref{eq:1D_kinetic_equation}, its action on \eqref{eq:f_approximation} reads
\begin{eqnarray}
  \frac{1}{\vmag}C_{AWBS}(\tilde{\ft})
  &=& 
  \frac{\vmag\zeta}{\mfpe} \pdv{}{\vmag}\left(\tilde{\ft} - \fM\right) 
  \nonumber \\
  && + \frac{1}{2}\left(\frac{\Zbar}{\mfpe} + \frac{\zeta}{\mfpe}\right)
  \pdv{}{\mu}(1 - \mu^2)\pdv{\tilde{\ft}}{\mu}  \nonumber \\
  &=& \frac{\vmag\zeta}{\mfpe} \pdv{}{\vmag}\left(\ft^0 - \fM\right) \nonumber \\ 
  &&\, + \mu\left(\frac{\vmag\zeta}{\mfpe} \pdv{\ft^1}{\vmag} 
  - \frac{\Zbar+\zeta}{\mfpe}\ft^1\right) ,
  \label{app_eq:AWBS_model_1D}
\end{eqnarray}
where $\nue^* = \zeta \nue = \frac{\vmag\zeta}{\mfpe}$ with $\zeta$ being a~scaling
parameter of the~standard e-e collision frequency. Its purpose is to
match AWBS heat flux to results obtained by Spitzer and Harm 
\cite{SpitzerHarm_PR1953} obtained for any $\Zbar$. 
\secref{sec:FPDiffusiveRegime} shows that this match can be found with 
a~constant $\zeta=0.5$.

One finds the~following equations if the~isotropic and anisotropic parts of 
\eqref{app_eq:LHS_kinetic_equation} and \eqref{app_eq:AWBS_model_1D} are 
compared 
\begin{eqnarray}
  \pdv{}{\vmag}\left( \ft^0 -\fM\right) &=& 
  \frac{\mfpe\qe\Ez}{\zeta\me\vmag^2}\frac{\ft^1}{\vmag} ,
  \label{app_eq:AWBS_f0} \\
  \frac{\vmag\zeta}{\mfpe} \pdv{\ft^1}{\vmag} 
  - \frac{\Zbar+\zeta}{\mfpe}\ft^1 &=&
  \pdv{\ft^0}{z} + \frac{\qe\Ez}{\me\vmag}\pdv{\ft^0}{\vmag} .
  \label{app_eq:AWBS_f1} 
  %\\  
  %\frac{\vmag}{\Zbar}\pdv{f^1}{\vmag} + \frac{4}{\Zbar}f^1 
  %- \frac{\Zbar + 1}{\Zbar} f^1 &=&
  %\pdv{f^0}{z} + \frac{\tilde{E}_z}{\vmag}\pdv{f^0}{\vmag}
  %\nonumber
\end{eqnarray}
If we assume that $\pdv{\ft^0}{\vmag} = \pdv{\fM}{\vmag}$, i.e. $\ft^0 = \fM$,
the~anisotropic part of the~AWBS operator is governed by the~equation
\begin{equation}
  \pdv{\ft^1_{AWBS}}{\vmag} 
  - \frac{\Zbar+\zeta}{\vmag\zeta}\ft^1_{AWBS} =
  \frac{\mfpe}{\vmag\zeta} 
  \left(\pdv{\fM}{z} + \frac{\qe\Ez}{\me\vmag}\pdv{\fM}{\vmag}\right) .
  \label{app_eq:AWBS_f1_fM}
\end{equation}
Even though it is not straightforward, the~electric field in 
\eqref{app_eq:AWBS_f1_fM} (solved numerically) providing a~zero current 
exactly matches \eqref{app_eq:BGK_Efield}. Consequently, the~deviation of
$\pdv{\ft^0}{\vmag}$ from $\pdv{\fM}{\vmag}$ in 
\eqref{app_eq:AWBS_f0} can be written as 
$Kn\frac{\vth^2}{\zeta\vmag^2}\frac{f^1}{\vmag}$.

Finally, it should be stressed, that the~concept of locality expressed as 
$Kn\ll1$ is crucial for our \textit{local diffusive regime} analysis, 
because it provides sufficient Maxwellization, i.e.  \eqref{app_eq:BGK_f0} and
\eqref{app_eq:AWBS_f0}, and correspondingly, \eqref{app_eq:BGK_f1_fM} 
and \eqref{app_eq:AWBS_f1_fM} are valid models.

\section{AP1 electric field limit}
\label{app:AP1limit}

Interestingly, we have encountered a~very specific property of the~AP1 model
with respect to the~electric field magnitude. The~easiest way how to 
demonstrate this is to write the~model equations \eqref{eq:AP1f0} and 
\eqref{eq:AP1f1} in 1D (z-axis). Then, due to its linear nature, it is easy 
to eliminate one of the~partial derivatives with respect to $\vmag$, i.e. 
$\pdv{\fzero}{\vmag}$ or $\pdv{\fonez}{\vmag}$. 
In the~case of elimination of $\pdv{\fzero}{\vmag}$ 
one obtains the~following equation
\begin{multline}
  %\frac{2}{3\vmag\nue} 
  %\left(\left(\sqrt{3}\vmag\frac{\nue}{2}\right)^2 - \Ez^2\right)  
  \left(\vmag\frac{\nue}{2} - \frac{2\qe^2\Ez^2}{3\me^2\vmag\nue}\right) 
  \pdv{\fonez}{\vmag} 
  =
  \frac{2\qe\Ez}{3\me\nue}\pdv{\fonez}{z}  
  + \frac{4\pi\qe\Ez}{3\me}\pdv{\fM}{\vmag} \\
  + \frac{\vmag}{3}\pdv{\fzero}{z} 
  + \left(\frac{4\qe^2\Ez^2}{3\me^2\vmag^2\nue}
  + \left(\nuei + \frac{\nue}{2}\right) \right)\fonez .
  \label{eq:AP1_model_1D}
\end{multline}
It is convenient to write the~bracket on the~left hand side of 
\eqref{eq:AP1_model_1D} as
$\frac{2}{3\vmag\nue} 
\left(\left(\sqrt{3}\vmag\frac{\nue}{2}\right)^2 
- \frac{\qe^2}{\me^2}\Ez^2\right)$
from where it is clear that the~bracket is negative if 
$\sqrt{3}\vmag\frac{\nue}{2} < \frac{\qe}{\me}|\E|$, 
i.e. there is a~velocity limit for a~given magnitude $|\E|$, 
when the~collisions are no more fully dominant and the~electric field 
introduces a~comparable effect to the~collision friction in 
the~electron transport.

It can be shown, that the~last term on the~right hand side of 
\eqref{eq:AP1_model_1D} is dominant and the~solution behaves as 
\begin{equation}
  \Delta \fone \sim \exp\left(\frac{\frac{4\qe^2\Ez^2}{3\me^2\vmag^2\nue}
  + \left(\nuei + \frac{\nue}{2}\right)}
  {\vmag\frac{\nue}{2} - \frac{2\qe^2\Ez^2}{3\me^2\vmag\nue}}\, 
  \Delta\vmag\right) ,
  \label{eq:f1z_behavior}
\end{equation}
where $\Delta \vmag < 0$ represents a~velocity step of the~implicit Euler
numerical integration of decelerating electrons.
However, \eqref{eq:f1z_behavior} exhibits an~exponential growth 
for velocities above the~friction limit (bracket on the~left hand side of 
\eqref{eq:AP1_model_1D})
\begin{equation}
  \vmag_{lim}  = \sqrt{\frac{\sqrt{3}\Gamma\me}{2\qe}\frac{n_e}{|\E|}} ,
  \label{eq:v_limit}
\end{equation}
which makes the~problem to be ill-posed.

In order to provide a~stable model, we introduce a~reduced electric field
to be acting as the~accelerating force of electrons
\begin{equation}
  |\E_{red}| = \sqrt{3} \vmag\frac{\me}{\qe}\frac{\nue}{2} ,
  \label{eq:Elimit}
\end{equation}
ensuring that the~bracket on the~left hand side of \eqref{eq:AP1_model_1D}
remains positive. Further more we define two quantities
\begin{equation}
  \omega_{red} = \frac{|\E_{red}|}{|\E|} ,\quad 
  \nuscat^E = \frac{\qe}{\me\vmag} \left(|\E| - |\E_{red}|\right),
  \nonumber
\end{equation}
introducing the~reduction factor of the~electric field
$\omega_{red}$ and the~compensation of the~electric field effect in terms of
scattering $\nuscat^E$. Consequently, the~AP1 model \eqref{eq:AP1f0}, 
\eqref{eq:AP1f1}, and \eqref{eq:AmpereKinetic} can be formulated as well posed 
with the~help of $\omega_{red}$ and $\nuscat^E$. 

\begin{comment} % delta f0.
Nevertheless, before doing so,
we introduce a~slightly different approximation to the~electron distribution 
function as
\begin{equation}
  \tilde{f} = \frac{4\pi \fM + \dafzero}{4\pi} + \frac{3}{4\pi}\vn\cdot\fone .
  \label{eq:P1_OOE}
\end{equation}
where $\dafzero$ represents the~departure of isotropic part from 
the~Maxwell-Boltzmann equilibrium distribution $\fM$. 
%which we keep 
%intentionally in the~distribution function approximation.
Then, the~stable AP1 model reads
\begin{eqnarray}
  \vmag \frac{\nue}{2}\pdv{\dafzero}{\vmag} &=&
  \vmag\nabla\cdot\fone 
  + \frac{\qe}{\me}\E\cdot\left(\omega_{red} \pdv{\fone}{\vmag} 
  + \frac{2}{\vmag}\fone\right) , 
  \label{eq:AP1f0_stable}\\
  \vmag\frac{\nue}{2}\pdv{\fone}{\vmag} 
  &=& \tnuscat\fone 
  + \frac{\vmag}{3}\nabla\left(4\pi\fM + \dafzero\right)
  \nonumber \\
  && 
  + \frac{\qe\E}{3\me}\left(4\pi \pdv{\fM}{\vmag} 
  + \omega_{red} \pdv{\dafzero}{\vmag} 
  \right) ,
  \label{eq:AP1f1_stable}
\end{eqnarray}
where $\tnuscat = \nuei + \nuscat^E + \frac{\nue}{2}$.

The~reason for keeping $\fM$ in the~distribution function approximation
\eqref{eq:P1_OOE} can be seen in the~last term on the~right hand side of 
\eqref{eq:AP1f1_stable}, which provides the~effect of electric field on
directional quantities as current or heat flux. In principle, the~explicit use
of $\fM$ ensures the~proper effect of $\E$ if $\dafzero \ll \fM$, i.e.
no matter what the~reduction $\omega_{red}$ is. Apart from its stability,
it also exhibits much better convergence of the~electric field, which is now
given by the~zero current condition of \eqref{eq:AP1f1_stable} as
\begin{equation}
  \E =
  \frac{\intv \left(\frac{\nue}{2\tnuscat}\vmag^2\pdv{\fone}{\vmag}
  - \frac{\vmag^2}{3\tnuscat}
  \nabla\left(4\pi\fM + \dafzero\right)\right) \vmag^2\, \dI\vmag}
  {\frac{\qe}{\me}\intv \frac{\vmag}
  {3\tnuscat}
  \left(4\pi\pdv{\fM}{\vmag} + \omega_{red} \pdv{\dafzero}{\vmag}\right)
  \vmag^2\, \dI\vmag} .
  \label{eq:AP1_Efield_stable}
\end{equation}
\end{comment} % delta f0.

\begin{comment} % CALDER kinetics.
\section{Calder kinetics}
\label{app:CalderKinetics}
and is written (it would be better to write Eq. (2) in its conservative forme and its non relativistic kernel U):
\begin{widetext}
\begin{eqnarray}
&&\frac{\partial f_\alpha}{\partial t}+\mathbf{v}\cdot\nabla_{\mathbf{x}}f_\alpha+q_\alpha\left(\mathbf{E}+\mathbf{v}\vect{\times}\mathbf{B}\right)\nabla_{\mathbf{p}}f_\alpha=C_{LBB}(f_\alpha,f_\alpha)+\sum_\beta C_{LBB}(f_\alpha,f_\beta),\\
&&C_{LBB}(f_\alpha,f_\beta)=-\frac{\partial}{\partial \mathbf{p}}\cdot\frac{\Gamma_{\alpha\beta}}{2}\left[\int \mathbf{U}(\mathbf{p},\mathbf{p}^\prime)\cdot(f_\alpha\nabla_{\mathbf{p}^\prime}f_\beta^\prime-f_\beta^\prime\nabla_{\mathbf{p}}f_\alpha)\right]d^3\mathbf{p}^\prime,\\
&&\mathbf{U}(\mathbf{p},\mathbf{p}^\prime)=\frac{r^2/\gamma\gamma^\prime}{(r^2-1)^{3/2}}\left[(r^2-1)\mathbf{I}-\mathbf{p}\otimes\mathbf{p}-\mathbf{p}^\prime\otimes\mathbf{p}^\prime+r(\mathbf{p}\otimes\mathbf{p}^\prime+\mathbf{p}^\prime\otimes\mathbf{p})\right],
\end{eqnarray}
\end{widetext}
with $\gamma=\sqrt{1+\mathbf{p}^2}$, $\gamma^\prime=\sqrt{1+\mathbf{p}^{\prime 2}}$ and $r=\gamma\gamma^\prime-\mathbf{p}\cdot\mathbf{p}^\prime$. The momemtum $\mathbf{p}_\alpha$ ($\mathbf{p}_\beta$) is normalized to $m_\alpha c$ (resp. $m_\beta c$). Obviously, this collision operator tends to Eq. (2) in the non-relativistic limit.
\end{comment} % CALDER kinetics.



\bibliographystyle{elsarticle-num}
\bibliography{NTH}

%% Authors are advised to submit their bibtex database files. They are
%% requested to list a bibtex style file in the manuscript if they do
%% not want to use elsarticle-num.bst.

%% References without bibTeX database:

%\begin{thebibliography}{00}

%% \bibitem must have the following form:
%\bibitem{Fish_RMP1987}{Nathaniel J. Fisch. Theory of current drive in plasmas. Rev. Mod. Phys., 59(1):175–234, Jan 1987.}
%\bibitem{Rosenbluth_PR1957}{Marshall N. Rosenbluth, William M. MacDonald, and David L. Judd. Fokker-planck equation for an inverse-square force. Phys.
%Rev., 107(1):1–6, Jul 1957.}
%\bibitem{Longmire_1963}{Longmire, Conrad L. : Elementary Plasma Physics. Intersci. Pub., 1963.}
%\bibitem{Shkarofsky_1966}{I.P. Shkarofsky, T.W. Johnston, T.W. Bachynski, The Particle Kinetics of Plasmas, Addison-Wesley, Reading, MA, 1966.}
%\bibitem{SpitzerHarm_PR1953}{SH 1953.}

%\end{thebibliography}

\clearpage

\end{document}

%% Old stuff goes to appendix 
\appendix
\section{The~Fokker-Planck equation}
\label{sed:FP}
%Use equation Hu (1, 2, 3) in text, (10), which puts together (4, 5, 7),
%and	write definitions (11, 12) using (5, 7, 13) and after define (6, 8)
%and to close write (31) from Tzoufras.
\begin{equation}
  \pdv{\ft}{t} + \vv\cdot\gx \ft + (\tE + \vv\times\tB) \cdot\gv \ft =
  - \gv \cdot \sum_{\pb}\vect{S}_c^{\tob} ,
  \nonumber
\end{equation}
where the~collision flux of test particles (labeled $\ft$) colliding on field
particles (labeled $\fb$) takes the~Landau-Fokker-Planck (LFP) form
\begin{equation}
  \vect{S}_c^{\tob} = \Gamma^{\tob}\int \gv\gv\vs \cdot \left[ 
  \ft(\vv)\frac{m_\pt}{m_\pb}\gvb \fb(\vvb) - \fb(\vvb)\gv \ft(\vv)
  \right]\, \dI\vvb , \nonumber 
  %&=& \vect{F}^{\tob}(\fb)\, \ft - \matr{D}^{\tob}(\fb)\cdot\gv\ft ,
  \nonumber
\end{equation} 
where $\Gamma^{\tob} = \frac{4\pi\Zbar_\pt^2\Zbar_\pb^2 q^4 \lnc}{m_\pt^2}$, $\vs = \vv - \vvb$, and 
$\frac{\MI}{\vsmag} - \frac{\vs\vs}{\vsmag^3} = \gv\gv\vs$ was used.
%The friction and diffusion coefficients of collision flux are then defined as
%\begin{eqnarray}
%  \vect{F}^{\tob}(\fb) &=& \frac{1}{m_\pb} 
%  \int \gv\gv\vs\cdot\gvb \fb(\vvb)\,\dI\vvb = -\Gamma^{\tob}\gv 
%  \Rh(\vv) ,
%  \nonumber \\
%  \matr{D}^{\tob}(\fb) &=& \frac{1}{m_\pt} 
%  \gv\gv\int \vs \fb(\vvb)\,\dI\vvb = -\Gamma^{\tob}\gv\gv \Rg(\vv) ,
%  \nonumber
%\end{eqnarray}
The~LFP integral collision model can written in the~form introduced by 
Rosenbluth 1957
\begin{equation}
  \left(\pdv{\ft}{t}\right)_{\pb} = - \gv \cdot \vect{S}_c^{\tob} = 
  - \Gamma^{\tob} \left[ \gv\cdot\left(\ft \gv\Rhb\right)
  - \frac{\gv\gv:\left(\ft \gv\gv\Rgb \right)}{2}\right] ,
  \label{eq:FP_Rosenbluth}
\end{equation}
where $\left(\pdv{\ft}{t}\right)_{\pb}$ expresses the~rate of change in 
the~distribution function of test particles $\ft$ due to collisions with
background field particles (distribution function $\fb$) and
where the~complicated nature of collisions is modeled by the~Rosenbluth 
potentials 
\begin{equation}
  \Rhb(\vv) = \frac{m_\pt + m_\pb}{m_\pb}
  \int \frac{\fb(\vvb)}{|\vv - \vvb|}\, \dI\vvb ,
  \quad \Rgb(\vv) = \int \fb(\vvb)|\vv - \vvb|\, \dI\vvb ,
  \nonumber
\end{equation}
which have the following properties
\begin{equation}
  \gv\cdot\gv \Rhb = 
  -4\pi \frac{m_\pt + m_\pb}{m_\pb} \Gamma^{\tob}\fb ,\quad
  \gv\cdot\gv \Rgb = 
  2 \frac{m_\pb}{m_\pt + m_\pb} \Rhb .
  \nonumber
\end{equation}
The~Rosenbluth equation \refeq{eq:FP_Rosenbluth} can be further rewritten
according to $[$Longmire, Conrad L. : Elementary Plasma Physics. Intersci. Pub., 1963$]$ as
\begin{equation}
  \left(\pdv{\ft}{t}\right)_c = \sum_{\pb} \Gamma^{\tob} 
  \left[ 4\pi \frac{m_{\pt}}{m_\pb} \fb \ft 
  + \frac{m_\pb - m_\pt}{m_\pt + m_\pb} \gv\Rhb\cdot\gv \ft 
  + \frac{\gv\gv\Rgb : \gv\gv \ft}{2} \right] ,
  \label{eq:FP_Longmire}
\end{equation}
which was also published in Shkarofsky 1966 and used in Tzoufras 2011.

%\subsection{Spherical harmonics expansion}
%\label{sec:FP_spherical_harmonics}
%Write Hu (14, 15) and full (1) in spherical coordinates. Then explicitly write
%(17, 18, 23, 24, 25). Further define spherical harmonics expansion and 
%from Tzoufras (34, 35, 36, 37).

\subsection{The linearized Fokker-Planck equation for low anisotropy}
\label{sec:FP_linear}
Define the anisotropic perturbation as in Tzoufras and then use the equations
(32, 33) and the harmonic expansions (38, 39, 40) and the most importantly (41)
for one-kind particles. Finally, write explicitly (41) for the case $f_1^0$
and write set of integrals $I, J$ and constants $C_1, .., C_6$,
which will be used to calculate FP equation solution for diffusive conditions.

If we write the~distribution function as its isotropic and anisotropic parts, 
i.e. $\ft = \ft_0 + \delta\ft$ and $\fb = \fb_0 + \delta\fb$, 
then the~linearized LFP operator for low anisotropy of order 
$O(\delta\ft^2, \delta\fb^2)$ reads 
\begin{eqnarray}
  \frac{1}{\Gamma^{\tob}} \left(\pdv{\ft_0}{t}\right)_{\pb} &=&  
  4\pi \frac{m_{\pt}}{m_\pb} \fb_0 \ft_0 
  + \frac{m_\pb - m_\pt}{m_\pt + m_\pb} \gv\Rh(\fb_0)\cdot\gv \ft \nonumber \\ 
  &&+ \frac{\gv\gv\Rg(\fb_0) : \gv\gv \ft_0}{2} ,
  \label{eq:FP_f0} \\ 
  \frac{1}{\Gamma^{\tob}}\left(\pdv{\delta\ft}{t}\right)_{\pb} &=&  
  4\pi \frac{m_{\pt}}{m_\pb} 
  \left(\fb_0\delta\ft + \ft_0\delta\fb\right) \nonumber \\ 
  &&+ \frac{m_\pb - m_\pt}{m_\pt + m_\pb} 
  \left(\gv\Rh(\fb_0)\cdot\gv \delta\ft + \gv\ft_0\cdot\gv\Rh(\delta\fb)\right)
  \nonumber \\ 
  &&+ \frac{\gv\gv\Rg(\fb_0) : \gv\gv \delta\ft}{2} 
  + \gv\gv \ft_0 : \frac{\gv\gv\Rg(\delta\fb)}{2} .
  \label{eq:FP_deltaf} 
\end{eqnarray}

\begin{eqnarray}
  \ft &=& \ft_0 + \sum_{l=1}^\infty\sum_{m=-l}^l \ft^m_l(\vmag) 
  P_l^{|m|}(\cos \theta) \exp^{i m \phi} ,
  \nonumber \\ 
  \fb &=& \fb_0 + \sum_{l=1}^\infty\sum_{m=-l}^l \fb^m_l(\vmag) 
  P_l^{|m|}(\cos \theta) \exp^{i m \phi} ,
  \nonumber
\end{eqnarray}

\begin{eqnarray}
  I_j(\fb_l^m) &=& \frac{4\pi}{\vmag^j} \int_0^\vmag \fb_l^m(u) u^{j+2}
  \, \dI u ,
  \nonumber \\
  J_j(\fb_l^m) &=& \frac{4\pi}{\vmag^j} \int_\vmag^\infty 
  \fb_l^m(u) u^{j+2}\, \dI u ,
  \nonumber
\end{eqnarray}

\begin{equation}
  \frac{1}{\Gamma^{\tob}}\left(\pdv{\ft_0}{t}\right)_{\pb} =
  \frac{1}{3\vmag^2}\pdv{}{\vmag}\left[\frac{3 m_\pt}{m_\pb} 
  \ft_0 I_0(\fb_0) + \vmag \left( I_2(\fb_0) + J_{-1}(\fb_0)\right)
  \pdv{\ft_0}{\vmag} \right] ,
  \nonumber
\end{equation}

\begin{multline}
  \frac{1}{\Gamma^{\tob}}\left(\pdv{\ft_l^m}{t}\right)_{\pb} = 
  4\pi\frac{m_\pt}{m_\pb} \left[\fb_0 \ft_l^m + \ft_0 \fb_l^m \right] 
  \\
  + \frac{m_\pt-m_\pb}{m_\pb \vmag^2} \left[ \pdv{\ft_0}{\vmag}
  \left(\frac{l+1}{2 l+1} I_l(\fb_l^m) 
  - \frac{l}{2 l +1} J_{-1-l}(\fb_l^m)\right) 
  + I_0(\fb_0)\pdv{\ft_l^m}{\vmag}\right] 
  \\
  + \frac{I_2(\fb_0) + J_{-1}(\fb_0)}{3\vmag}
  \frac{\partial^2 \ft_l^m}{\partial\vmag^2} + 
  \frac{-I_2(\fb_0) + 2 J_{-1}(\fb_0) + 3 I_0(\fb_0)}{3\vmag^2}
  \pdv{\ft_l^m}{\vmag}
  \\
  - \frac{l(l+1)}{2}
  \frac{-I_2(\fb_0) + 2 J_{-1}(\fb_0) + 3 I_0(\fb_0)}{3\vmag^3} \ft_l^m 
  \\
  \frac{1}{2 \vmag} \frac{\partial^2 \ft_0}{\partial\vmag^2}
  \left[C_1 I_{l+2}(\fb_l^m) + C_1 J_{-l-1}(\fb_l^m) + C_2 I_l(\fb_l^m) 
  + C_2 J_{1-l}(\fb_l^m) \right]
  \\
  \frac{1}{\vmag^2} \pdv{\ft_0}{\vmag}
  \left[C_3 I_{l+2}(\fb_l^m) + C_4 J_{1-l}(\fb_l^m) + C_5 J_{-l-1}(\fb_l^m) 
  + C_6 I_l(\fb_l^m) \right]
  \label{eq:FP_flm}
\end{multline}
\begin{eqnarray}
  C_1 = \frac{(l+1)(l+2)}{(2l+1)(2l+3)}, 
  C_2 = -\frac{(l-1)l}{(2l +1)(2l-1)},
  C_3 = -\frac{l(l+1)/2 + (l+1)}{(2l+1)(2l+3)},
  \nonumber \\
  C_4 = \frac{l(l+1)/2 - l}{(2l+1)(2l-1)},
  C_5 = \frac{(l+2) - l(l+1)/2}{(2l+1)(2l+3)}, 
  C_6 = \frac{l(l+1)/2 + (l-1)}{(2l+1)(2l-1)}, 
  \nonumber
\end{eqnarray}

In the~case of massive background particles $m_\pt / m_\pb << 1$ in equilibrium
and comparable temperatures $T_\pb\approx T_\pb$, 
i.e. slow-non-moving background, 
the~isotropic distribution function can be approximated by 
$\fb_0^{slow} = n_{slow} \delta(\vmag) / (4\pi\vmag^2)$, 
and since all integrals 
$I_j(\fb_0^{slow}), J_j(\fb_0^{slow})$ vanish except 
$I_0(\fb_0^{slow}) = n_{slow}$, equation \refeq{eq:FP_flm} reduces to 
\begin{equation}
  \frac{1}{\Gamma^{\pt/slow}}\left(\pdv{\ft_l^m}{t}\right)_{slow} = 
  - \frac{l(l+1)}{2}
  \frac{n_{slow}}{\vmag^3} \ft_l^m 
  \label{eq:FP_flm_slow}
\end{equation} 
where $n_{slow}$ is the~density of slow massive particles. Consequently, 
the~effect of collisions on slow massive particles leads to scattering but no
change in velocity, i.e. energy, of test particles.

\subsection{Plasma Fokker-Planck equation in diffusive regime}
\begin{equation}
  \left(\pdv{\ft_0}{t}\right)_{e} =
  \frac{\Gamma^{e/e}}{3\vmag^2}\pdv{}{\vmag}\left[3 
  \ft_0 I_0(\ft_0) + \vmag \left( I_2(\ft_0) + J_{-1}(\ft_0)\right)
  \pdv{\ft_0}{\vmag} \right] ,
  \nonumber
\end{equation}

\begin{multline}
  \left(\pdv{\ft_l^m}{t}\right)_{e} = \Gamma^{e/e} \Bigg[
  8\pi \ft_0 \ft_l^m  
  - \frac{l(l+1)}{2}
  \frac{3 I_0(\ft_0) - I_2(\ft_0) + 2 J_{-1}(\ft_0)}{3\vmag^3} \ft_l^m
  \\
  + \frac{I_2(\ft_0) + J_{-1}(\ft_0)}{3\vmag}
  \frac{\partial^2 \ft_l^m}{\partial\vmag^2} + 
  \frac{3 I_0(\ft_0) - I_2(\ft_0) + 2 J_{-1}(\ft_0)}{3\vmag^2}
  \pdv{\ft_l^m}{\vmag} 
  \\
  + \frac{1}{2 \vmag} \frac{\partial^2 \ft_0}{\partial\vmag^2}
  \left[C_1 I_{l+2}(\ft_l^m) + C_1 J_{-l-1}(\ft_l^m) + C_2 I_l(\ft_l^m) 
  + C_2 J_{1-l}(\ft_l^m) \right]
  \\
  + \frac{1}{\vmag^2} \pdv{\ft_0}{\vmag}
  \left[C_3 I_{l+2}(\ft_l^m) + C_4 J_{1-l}(\ft_l^m) + C_5 J_{-l-1}(\ft_l^m) 
  + C_6 I_l(\ft_l^m) \right] \Bigg]
\end{multline}

\begin{equation}
  C_1 = \frac{2}{5}, 
  C_2 = 0,
  C_3 = -\frac{1}{5},
  C_4 = 0,
  C_5 = \frac{2}{15}, 
  C_6 = \frac{1}{3}, 
  \nonumber
\end{equation}

%\begin{multline}
%  \frac{1}{\Gamma^{e/e}}\left(\pdv{\ft_1}{t}\right)_{e} = 
%  8\pi \ft_0 \ft_1  
%  - \frac{-I_2(\ft_0) + 2 J_{-1}(\ft_0) + 3 I_0(\ft_0)}{3\vmag^3} \ft_1
%  \\
%  + \frac{I_2(\ft_0) + J_{-1}(\ft_0)}{3\vmag}
%  \frac{\partial^2 \ft_1}{\partial\vmag^2} + 
%  \frac{-I_2(\ft_0) + 2 J_{-1}(\ft_0) + 3 I_0(\ft_0)}{3\vmag^2}
%  \pdv{\ft_1}{\vmag} 
%  \\
%  + \frac{1}{5 \vmag} \frac{\partial^2 \ft_0}{\partial\vmag^2}
%  \left[I_{3}(\ft_1) + J_{-2}(\ft_1)\right]
%  + \frac{1}{15 \vmag^2} \pdv{\ft_0}{\vmag}
%  \left[5 I_1(\ft_1) - 3 I_{3}(\ft_1) + 2 J_{-2}(\ft_1) 
%  \right]
%\end{multline}

\begin{multline}
  \left(\pdv{\ft_1}{t}\right)_{e+i} = \Gamma^{e/e}\Bigg[
  8\pi \ft_0 \ft_1  
  - \left[\frac{3 I_0(\ft_0) - I_2(\ft_0) + 2 J_{-1}(\ft_0)}{3\vmag^3} +
  \frac{\Zbar n_e}{\vmag^3}\right] \ft_1
  \\
  +   \frac{3 I_0(\ft_0) -I_2(\ft_0) + 2 J_{-1}(\ft_0)}{3\vmag^2}
  \pdv{\ft_1}{\vmag}
  + \frac{I_2(\ft_0) + J_{-1}(\ft_0)}{3\vmag}
  \frac{\partial^2 \ft_1}{\partial\vmag^2} 
  \\
  + \frac{1}{5 \vmag} \frac{\partial^2 \ft_0}{\partial\vmag^2}
  \left[I_{3}(\ft_1) + J_{-2}(\ft_1)\right]
  + \frac{1}{15 \vmag^2} \pdv{\ft_0}{\vmag}
  \left[5 I_1(\ft_1) - 3 I_{3}(\ft_1) + 2 J_{-2}(\ft_1) 
  \right] \Bigg]
\end{multline}

\begin{equation}
  \ft \approx \fM + \cos(\theta) \mfpei(\vmag) 
  \left[\frac{\vmag^2}{2\vth^2} - 4 + D\left(\vmag\right) \right] \fM ,
  \nonumber
\end{equation}

\begin{multline}
  \vmag\pdv{\fM}{z} - \frac{\vmag\tEz}{\vth^2}\fM = \Gamma^{e/e}\Bigg[
  8\pi \fM \ft_1  
  - \frac{3 I_0(\fM) - I_2(\fM) + 2 J_{-1}(\fM) + 3 \Zbar n_e}{3\vmag^3} \ft_1
  \\
  +   \frac{3 I_0(\fM) -I_2(\fM) + 2 J_{-1}(\fM)}{3\vmag^2}
  \pdv{\ft_1}{\vmag}
  + \frac{I_2(\fM) + J_{-1}(\fM)}{3\vmag}
  \frac{\partial^2 \ft_1}{\partial\vmag^2} 
  \\
  + \frac{\fM}{15 \vmag\vth^2} \left[\frac{3\vmag^2}{\vth^2}
  \left[I_{3}(\ft_1) + J_{-2}(\ft_1)\right]
  - 5 \left[I_1(\ft_1) + J_{-2}(\ft_1)\right]
  \right] \Bigg]
  \nonumber
\end{multline}

\begin{multline} 
  \left[\frac{I_2(\fM) + J_{-1}(\fM)}{3\vmag}\right]
  \frac{\partial^2 \ft_1}{\partial\vmag^2}
  + \left[\frac{3 I_0(\fM) -I_2(\fM) + 2 J_{-1}(\fM)}{3\vmag^2}\right]
  \pdv{\ft_1}{\vmag}
  \\ 
  + \left[8\pi \fM  - \frac{3 I_0(\fM) - I_2(\fM) + 2 J_{-1}(\fM)}{3\vmag^3}
  - \frac{\Zbar n_e}{\vmag^3} \right] \ft_1 =
  \\
  \frac{1}{\Gamma^{e/e}}
  \left[\vmag\pdv{\fM}{z} - \frac{\vmag\tEz}{\vth^2}\fM\right]
  - \frac{\fM}{15 \vmag\vth^2} \left[\frac{3\vmag^2}{\vth^2}
  \left[I_{3}(\ft_1) + J_{-2}(\ft_1)\right]
  - 5 \left[I_1(\ft_1) + J_{-2}(\ft_1)\right]
  \right]
  \label{eq:FP_f1_diffusive}
\end{multline}

\begin{equation} 
  a \frac{\partial^2 \ft_1}{\partial\vmag^2} + b \pdv{\ft_1}{\vmag} + c \ft_1 
  = d
  \nonumber
\end{equation}

Integration of \refeq{eq:FP_f1_diffusive} from $\infty \rightarrow 0$
\begin{eqnarray}
  a^{n-0.5} \frac{df^n - df^{n-1}}{ - \Delta \vmag} &=& b^{n-0.5} df^{n-1} 
  + c^{n-0.5} \ft_1^{n-1} + d^{n-0.5} ,
  \nonumber \\
  \frac{\ft_1^n - \ft_1^{n-1}}{- \Delta \vmag} &=& df^{n-1} ,
  \nonumber
\end{eqnarray}
\begin{eqnarray}
  \left(\frac{a^{n-0.5}}{\Delta \vmag} - b^{n-0.5}\right) df^{n-1} 
  - c^{n-0.5} \ft_1^{n-1}&=&  d^{n-0.5} + \frac{a^{n-0.5}}{\Delta \vmag} df^n,
  \nonumber \\
  - df^{n-1} + \frac{1}{\Delta \vmag} \ft_1^{n-1} &=&  
  \frac{1}{\Delta \vmag} \ft_1^n ,
  \nonumber
\end{eqnarray}

\begin{equation}
  \begin{bmatrix}
    - c^{n-0.5}  &
	\frac{a^{n-0.5}}{\Delta \vmag} - b^{n-0.5}
    \\
	\frac{c^{n-0.5}}{\Delta \vmag} & -c^{n-0.5} 
  \end{bmatrix}
  \begin{bmatrix}
    \ft_1^{n-1} \\
    df^{n-1}
  \end{bmatrix}
  =  
  \begin{bmatrix}
    d^{n-0.5} + \frac{a^{n-0.5}}{\Delta \vmag} df^n \\
    \frac{c^{n-0.5}}{\Delta \vmag} \ft_1^n
  \end{bmatrix}   ,
  \nonumber
\end{equation}

\begin{multline}
  \begin{bmatrix}
    - c^{n-0.5}  &
	\frac{a^{n-0.5}}{\Delta \vmag} - b^{n-0.5} 
    \\
	0 & 
	\frac{1}{\Delta \vmag}
	\left(\frac{a^{n-0.5}}{\Delta \vmag} - b^{n-0.5}\right) - c^{n-0.5} 
  \end{bmatrix}
  \begin{bmatrix}
    \ft_1^{n-1} \\
    df^{n-1}
  \end{bmatrix}
  =  \\
  \begin{bmatrix}
    d^{n-0.5} + \frac{a^{n-0.5}}{\Delta \vmag} df^n \\
    \frac{1}{\Delta \vmag} \left(c^{n-0.5}\ft_1^n + 
	d^{n-0.5} + \frac{a^{n-0.5}}{\Delta \vmag} df^n \right) 
  \end{bmatrix}   ,
  \nonumber
\end{multline}

\begin{eqnarray}
  a^n &=& \frac{I^n_2(\fM) + J^n_{-1}(\fM)}{3\vmag^n} ,
  \nonumber \\
  b^n &=& \frac{3 I^n_0(\fM) - I^n_2(\fM) + 2 J^n_{-1}(\fM)}{3{\vmag^n}^2} ,
  \nonumber \\
  c^n &=& 8\pi \fM^n  - \frac{3 I^n_0(\fM) - I^n_2(\fM) + 2 J^n_{-1}(\fM)}
  {3{\vmag^n}^3} - \frac{\Zbar n_e}{{\vmag^n}^3},
  \nonumber \\
  d^n &=& \frac{1}{\Gamma^{e/e}}
  \left[\vmag^n\pdv{\fM^n}{z} - \frac{\vmag^n\tEz}{\vth^2}\fM^n\right]
  \nonumber \\
  &&- \frac{\fM^n}{15 \vmag^n\vth^2} \left[\frac{3{\vmag^n}^2}{\vth^2}
  \left[I^n_{3}(\ft_1) + J^n_{-2}(\ft_1)\right]
  - 5 \left[I^n_1(\ft_1) + J^n_{-2}(\ft_1)\right] \right]
  \nonumber
\end{eqnarray}

\begin{eqnarray}
  I^n_0(g) = 4\pi \int_0^{\vmag^n} g(u) u^{2}
  \, \dI u ,\quad
  I^n_1(g) = \frac{4\pi}{\vmag^n} \int_0^{\vmag^n} g(u) u^{3}
  \, \dI u ,
  \nonumber \\
  I^n_2(g) = \frac{4\pi}{{\vmag^n}^2} \int_0^{\vmag^n} g(u) u^{4}
  \, \dI u ,\quad 
  I^n_3(g) = \frac{4\pi}{{\vmag^n}^3} \int_0^{\vmag^n} g(u) u^{5}
  \, \dI u ,
  \nonumber \\
  J^n_{-1}(g) = 4\pi\vmag^n \int_{\vmag^n}^\infty 
  g(u) u\, \dI u ,\quad
  J^n_{-2}(g) = 4\pi{\vmag^n}^2 \int_{\vmag^n}^\infty 
  g(u) \, \dI u ,
  \nonumber
\end{eqnarray}

\section{AWBS-P1 modeling of laser heated plasmas}\label{sec:OOE_AWBSP1}
\subsection{Model equations}
The~AWBS electron transport equation reads
\begin{multline}
  \vmag\vn\cdot\nabla f + \tE\cdot\vn \pdv{f}{\vmag} 
  + \frac{\tE\cdot\vect{e}_\phi 
  - \vmag\tB\cdot\vect{e}_\theta}{\vmag}\pdv{f}{\phi}
  =
  \vmag \nue \pdv{}{\vmag}\left(f - f_M\right) 
  + (\nuei + \nue) (\fzero - f) ,
  \nonumber \label{eq:AWBS_model}
\end{multline}
where $\nue$ is the~electron-electron collision frequency, 
$\nuei$ is the~electron-ion collision frequency, and $\nuei = \Zbar \nue$.

In order to eliminate the~dimensions of the~above transport problem 
the~first-two-moment model based on approximation 
\begin{equation}
  f = \frac{\fzero}{4\pi} + \frac{3}{4\pi}\vn\cdot\fone , 
  \nonumber \label{eq:OOE_P1approximation}
\end{equation}
can be adopted and reads
\begin{eqnarray}
  \nue\vmag\pdv{}{\vmag}\left(\fzero - \tfM \right) &=&
  \vmag\nabla\cdot\fone + \tE\cdot
  \pdv{\fone}{\vmag} + \frac{2}{\vmag}\tE\cdot\fone , 
  \nonumber \label{eq:OOE_P1f0}\\
  \nue\vmag\pdv{}{\vmag}\fone - \nutot\fone &=& 
  \vmag\nabla\cdot\left(\MA\fzero\right) + 
  \tE\cdot\pdv{\left(\MA\fzero\right)}{\vmag} + \tB\times\fone ,
  \nonumber \label{eq:OOE_P1f1}
\end{eqnarray}
where $\tfM = 4\pi \fM$ and the~closure matrix takes the~form
\begin{equation}
  \MA = \frac{1}{3}\MI .
  \nonumber \label{eq:OOE_P1closure}
\end{equation}

Since in the~laser heated plasmas the~Knudsen number 
Kn$ = \frac{\vth}{\nu_t(\vth) L} \in (0, 1)$, i.e. the~collisionality in 
the~kinetics of electrons plays always an~important effect for thermal-like 
particles, the~electron distribution 
function can be treated as out-of-equilibrium approximation 
\begin{equation}
  f = \fM + \daf ,
  \label{eq:OOE_outofeq}
\end{equation}  
where the~consequent AWBS model reads
\begin{multline}
  \vmag\vn\cdot\nabla (\fM + \daf) + \tE\cdot\vn \pdv{\fM}{\vmag} 
  + \tE\cdot\vn \pdv{\daf}{\vmag} 
  + \frac{\tE\cdot\vect{e}_\phi 
  - \vmag\tB\cdot\vect{e}_\theta}{\vmag}\pdv{\daf}{\phi}
  = \\
  \vmag \nue \pdv{\daf}{\vmag} 
  + (\nuei + \nue) (\fzero - \fM - \daf) ,
  \label{eq:OOE_AWBS_model}
\end{multline}
or its P1 approximation equivalent
\begin{equation}
  f = \frac{4\pi \fM + \dafzero}{4\pi} + \frac{3}{4\pi}\vn\cdot\fone .
  \label{eq:OOE_P1outofeq}
\end{equation}
where the~moment model reads
\begin{eqnarray}
  \nue\vmag\pdv{\dafzero}{\vmag} &=&
  \vmag\nabla\cdot\fone + \tE\cdot
  \pdv{\fone}{\vmag} + \frac{2}{\vmag}\tE\cdot\fone , 
  \label{eq:OOE_P1f0}\\
  \nue\vmag\pdv{\fone}{\vmag} - \nutot\fone &=& 
  \frac{\vmag}{3}\nabla\dafzero + 
  \frac{\tE}{3}\pdv{\dafzero}{\vmag} + \tB\times\fone 
  \nonumber \\
  & & + \frac{\vmag}{3}\nabla\tfM + \frac{\tE}{3}\pdv{\tfM}{\vmag} .
  \label{eq:OOE_P1f1}
\end{eqnarray}

\subsection{A~consistent treatment of $\tE$ field}
\label{sec:OOE_E_treatment}
The~plasma conditions providing an~appropriate electric field
are the~best expressed via the~definition of current
\begin{equation}
  \vect{q}_c(\vect{x}) = \intv
  \vmag\fone(\vect{x}) \vmag^2\, \dI\vmag , 
  \nonumber 
\end{equation}
which can be directly expressed from \refeq{eq:OOE_P1f1} as
\begin{equation}
  %\frac{\vect{j}}{\qe} = 
  \vect{q}_c =
  \intv \left(\frac{\nue\vmag^2}{\nutot}\pdv{\fone}{\vmag}
  - \frac{\vmag^2}{3\nutot}\nabla\left(\tfM + \dafzero\right) - 
  \frac{\vmag}{3\nutot}\pdv{\left(\tfM + \dafzero\right)}{\vmag}\tE
  \right) \vmag^2\, \dI\vmag ,
  \label{eq:OOE_P1current}
\end{equation}
where the $\B$ field and $\E$ field scattering effect (angular) have been 
omitted.  

Then, the~current can be easily evaluated based on
\begin{eqnarray}
  a_0(\vect{x}) &=& \intv\frac{\vmag}{3 \nutot} \pdv{\left(\tfM + \dafzero\right)}{\vmag}(\vect{x})
  \vmag^2\, \dI\vmag , \nonumber \\
  \vect{b}_{0}(\vect{x}) &=& \intv\left(  
  \frac{\vmag^2}{3 \nutot}\nabla\left(\tfM(\vect{x}) + \dafzero(\vect{x})\right)
  - \frac{\nue\vmag^2}{\nutot}\pdv{\fone}{\vmag}(\vect{x})
  \right)\vmag^2\, \dI\vmag , \nonumber 
\end{eqnarray}
%Alors, treat $a_0, \vect{b}_{0}, \vect{b}_1$ as grid functions, 
%where it is obvious to express 
%$\pdv{a_0}{\vmag}, \pdv{\vect{b}_{0}}{\vmag}, \pdv{\vect{b}_1}{\vmag}$ in 
%ImplicitSolve().
as the~following generalization of the~Ohm's law
\begin{equation}
  \vect{q}_c(\vect{x}) 
  = -\vect{b}_{0}(\vect{x}) 
  - a_{0}(\vect{x})\tE(\vect{x}) ,
  \nonumber \label{eq:OOE_HOFcurrent}
\end{equation}
where one needs the~actual distribution function $f$ values.

%\begin{equation}
%  \tE(\vect{x}) = -\frac{\vect{b}_{0}(\vect{x})}{a_{0}(\vect{x})} 
%  - \frac{\vect{q}_c(\vect{x})}{a_{0}(\vect{x})}
%   + \frac{\vect{b}_{1}(\vect{x})}{a_{0}(\vect{x})}\times\tB(\vect{x}) ,
%  \nonumber \label{eq:OOE_HOFcurrent}
%\end{equation}

It is straightforward to find the~\textit{zero current} formula for 
the~electric field
\begin{equation}
  \tE(\vect{x}) = -\frac{\vect{b}_{0}(\vect{x})}{a_{0}(\vect{x})} .
  \label{eq:OOE_E_j0}
\end{equation}

\subsection{AWBS model analysis}
\label{sec:OOE_AWBS_model_analysis}
The~AWBS transport equation can be written as the~following
\begin{multline}
  \left(\vmag \nue - \tE\cdot\vn\right)\pdv{\daf}{\vmag} 
  %+ \frac{\tE\cdot\vect{e}_\phi 
  %- \vmag\tB\cdot\vect{e}_\theta}{\vmag}\pdv{f}{\phi}
  =
  \vmag\vn\cdot\nabla (\fM + \daf) + \tE\cdot\vn\pdv{\fM}{\vmag} 
  + (\nuei + \nue) (\fM + \daf - \fzero) ,
  \label{eq:OOE_AWBS_model_analysis}
\end{multline}
in order to stress the~effect of force applied to electrons, i.e. the~effect
of friction described by $\nue$ and the~Lorentz force effect via $\tE$, 
and their competition.

The~same reformulation can be written for the~moment AWBS model
\begin{eqnarray}
  \pdv{\dafzero}{\vmag} &=&
  \frac{1}{\nue\vmag}\left(\vmag\nabla\cdot\fone + \tE\cdot
  \pdv{\fone}{\vmag}\right) , 
  \nonumber\\
  \nue\vmag\pdv{\fone}{\vmag} - \nutot\fone &=& 
  \frac{\vmag}{3}\nabla\dafzero + 
  \frac{\tE}{3}\pdv{\dafzero}{\vmag}
  + \frac{\vmag}{3}\nabla\tfM + \frac{\tE}{3}\pdv{\tfM}{\vmag} ,
  \nonumber
\end{eqnarray}
and it takes the~following form
\begin{equation}
  \left(\nue\vmag\MI - \frac{\tE\tE}{3 \nue\vmag}\right)\cdot
  \pdv{\fone}{\vmag} = 
  \frac{\vmag}{3}\nabla\dafzero + \frac{\tE}{3 \nue}\nabla\cdot\fone
  + \frac{\vmag}{3}\nabla\tfM + \frac{\tE}{3}\pdv{\tfM}{\vmag} 
  + \nutot\fone ,
  \nonumber \label{eq:OOE_P1AWBS_model_analysis}
\end{equation}
which is especially instructive in 1D
\begin{equation}
  \left(\nue\vmag - \frac{\tEz^2}{3 \nue\vmag}\right)
  \pdv{\fonez}{\vmag} = 
  \frac{\vmag}{3}\pdv{\dafzero}{z} + \frac{\tEz}{3 \nue}\pdv{\fonez}{z}
  + \frac{\vmag}{3}\pdv{\tfM}{z} + \frac{\tEz}{3}\pdv{\tfM}{\vmag} 
  + \nutot\fonez ,
  \label{eq:OOE_P1AWBS_model_1Danalysis}
\end{equation}
because it gives a \textit{"reverse-time-like evolution"} condition
\begin{equation}
  \sqrt{3} \nue > \frac{|\tEz|}{\vmag} .
  \label{eq:OOE_reverse_time_condition}
\end{equation}

\begin{equation}
  \pdv{\fonez}{\vmag} = 
  \frac{3 \nue\vmag}{3 (\nue\vmag)^2 - \tEz^2}
  \left(
  \frac{\vmag}{3}\pdv{\dafzero}{z} + \frac{\tEz}{3 \nue}\pdv{\fonez}{z}
  + \frac{\vmag}{3}\pdv{\tfM}{z} + \frac{\tEz}{3}\pdv{\tfM}{\vmag} 
  + \nutot\fonez\right) ,
  \label{eq:OOE_P1AWBS_model_1Danalysis}
\end{equation}
because it gives a \textit{"reverse-time-like evolution"} condition
\begin{equation}
  3(\nue\vmag)^2 - \tEz^2 \neq 0 ,
  \nonumber
\end{equation}
or a~numerical stability formulation
because it gives a \textit{"reverse-time-like evolution"} condition
\begin{equation}
  |3(\nue\vmag)^2 - \tEz^2| > \epsilon .
  \label{eq:OOE_reverse_time_condition}
\end{equation}
which can be obtained from
\begin{equation}
  \left(3(\nue\vmag)^2 - \tEz^2\right)^2 - \epsilon^2 = 0 .
  \label{eq:OOE_reverse_time_condition}
\end{equation}

\subsection{\textit{"Reverse-time-like evolution"} model by splitting}
\label{sec:OOE_stable_splitting}
Full separation of advection and E field
\begin{eqnarray}
  \nue\vmag \pdv{\fone^{\nue}}{\vmag} &=& 
  \frac{\vmag}{3}\nabla\dafzero^{\nue}
  + \frac{\vmag}{3}\nabla\tfM + \nutot\fone^{\nue} ,
  \nonumber \\
  \frac{\tE\tE}{3 \nue\vmag}\cdot
  \pdv{\fone^{\tE}}{\vmag} &=& -\frac{\tE}{3 \nue}\nabla\cdot\fone^{\tE} 
  - \frac{\tE}{3}\pdv{\tfM}{\vmag},
  \nonumber
\end{eqnarray}
or separation of stable "bulk" E field effect and implicit E field effect
\begin{eqnarray}
  \nue\vmag \pdv{\fone^{\nue}}{\vmag} &=& 
  \frac{\vmag}{3}\nabla\dafzero^{\nue}
  + \frac{\vmag}{3}\nabla\tfM + \nutot\fone^{\nue} 
  + \frac{\tE}{3}\pdv{\tfM}{\vmag},
  \nonumber \\
  \frac{\tE\tE}{3 \nue\vmag}\cdot
  \pdv{\fone^{\tE}}{\vmag} &=& -\frac{\tE}{3 \nue}\nabla\cdot\fone^{\tE} ,
  \nonumber
\end{eqnarray}
and the~complete effect of diffusion in velocity space reads
\begin{equation}
  \pdv{\fone}{\vmag} = \pdv{\fone^{\nue}}{\vmag} + \pdv{\fone^{\tE}}{\vmag} ,
  \nonumber
\end{equation}

\begin{comment}
\begin{multline}
  \IV_{\left(\MA \cdot \tE\right)} \cdot
  \left[
  \frac{1}{\vmag} \IM^{0^{-1}}_{(\nue)} \cdot \IV^T_{\left(\tE\right)} \cdot 
  \pdv{\fone}{\vmag} 
  + \IM^{0^{-1}}_{(\nue)} \cdot \left( \ID^T_{\left(\MI\right)} 
  + \frac{2}{\vmag^2}\IV^T_{\left(\tE\right)} \right)\cdot  
  \left(\fone^n 
  + \Delta\vmag\pdv{\fone}{\vmag}\right)\right] = \\
  -\IV_{\left(\MA \cdot \tE\right)} \cdot
  \pdv{\tvfM}{\vmag} .
  \nonumber
\end{multline}
\end{comment}

\subsection{\textit{"Friction"} model}
\label{sec:OOE_stable_splitting}
\newcommand{\nuE}{\nu_{\tE}}
In order to obey \refeq{eq:OOE_reverse_time_condition}, an~additional friction
$\nu_{\tE}$ can be introduced as
\begin{eqnarray}
  |\tE| &=& |\tE^*| + \nuE \vmag  ,
  \nonumber \\
  \nue + \nuE &=& \frac{|\tE^*|}{\vmag} ,
  \nonumber
\end{eqnarray}
which is then applied to perturbation $\dafzero$ as
\begin{eqnarray}
  \left(\nue + \nuE\right) \vmag \pdv{\dafzero}{\vmag} &=&
  \vmag\nabla\cdot\fone + \tE^*\cdot\pdv{\fone}{\vmag} , 
  \nonumber\\
  \left(\nue + \frac{\nuE}{3}\right) \vmag\pdv{\fone}{\vmag} - \nutot\fone &=& 
  \frac{\vmag}{3}\nabla\dafzero + 
  \frac{\tE^*}{3}\pdv{\dafzero}{\vmag}
  + \frac{\vmag}{3}\nabla\tfM + \frac{\tE}{3}\pdv{\tfM}{\vmag} .
  \nonumber
\end{eqnarray}

\begin{comment} % Implicit fully-discrete scheme
\subsection{Implicit fully-discrete scheme}
\label{sec:OOE_impl_fullydiscrete_scheme}
The moment P1 model \refeq{eq:OOE_P1f0}, \refeq{eq:OOE_P1f1} can be written 
in a~semi-discrete form
\begin{eqnarray}
  \IM^0_{(\nue)} \cdot \pdv{\davfzero}{\vmag}  
  &=& 
  \left(\ID^T_{\left(\MI\right)}
  + \frac{2}{\vmag^2}\IV^T_{\left(\tE\right)}\right) \cdot \fone
  + \frac{1}{\vmag}\IV^T_{\left(\tE\right)} \cdot 
  \pdv{\fone}{\vmag} ,  
  \nonumber \label{eq:OOE_semiM1hosf0} \\
  \IM^1_{(\nue)} \cdot \pdv{\fone}{\vmag}  
  &=& 
  - \ID_{\left(\MA\right)}\cdot\davfzero
  + \frac{1}{\vmag}\IV_{\left(\MA \cdot \tE\right)} \cdot
  \pdv{\davfzero}{\vmag}
  + \frac{1}{\vmag}\left(
  \IM^1_{\left( \nutot \right)} + \IB_{\left( \tB \right)}  
  \right) \cdot \fone
  \nonumber\\
  & & - \ID_{\left(\MA\right)}\cdot \tvfM
  + \frac{1}{\vmag}\IV_{\left(\MA \cdot \tE\right)} \cdot
  \pdv{\tvfM}{\vmag} .
  \label{eq:OOE_semiM1hosf1}
\end{eqnarray}
It is convenient to define
\begin{equation}
  \matr{DIVE} = \ID^T_{\left(\MI\right)} 
  + \frac{2}{\vmag^2}\IV^T_{\left(\tE\right)}.
  \nonumber
\end{equation}
The next step goes towards implicit Runge-Kutta method (DIRK).
Then, a~very effective inversion in the~L2 space can be formally performed
\begin{equation}
  \pdv{\davfzero}{\vmag}  
  = \frac{1}{\vmag}
  \IM^{0^{-1}}_{(\nue)} \cdot \IV^T_{\left(\tE\right)} \cdot 
  \pdv{\fone}{\vmag} 
  + \IM^{0^{-1}}_{(\nue)} \cdot \matr{DIVE}\cdot  
  \left(\fone^n 
  + \Delta\vmag\pdv{\fone}{\vmag}\right) .
  \label{eq:OOE_fullP1hosf0}
\end{equation}
The~first moment equation in the DIRK framework reads
\begin{multline}
  \IM^1_{(\nue)} \cdot \pdv{\fone}{\vmag}  
  = \frac{1}{\vmag}\IV_{\left(\MA \cdot \tE\right)} \cdot
  \pdv{\davfzero}{\vmag}
  - \ID_{\left(\MA\right)}\cdot \left(\tvfM^n + \davfzero^n 
  + \Delta\vmag\pdv{\davfzero}{\vmag}\right)\\ 
  + \frac{1}{\vmag}\left(\IB_{\left( \tB \right)} 
  + \IM^1_{\left( \nutot \right)}\right) 
  \cdot \left(\fone^n 
  + \Delta\vmag\pdv{\fone}{\vmag}\right)
  + \frac{1}{\vmag}\IV_{\left(\MA \cdot \tE\right)} \cdot
  \pdv{\tvfM^n}{\vmag} .
  \label{eq:OOE_semiM1hosf1}
\end{multline}
Finally, one can eliminate $\pdv{\davfzero}{\vmag}$, 
which leads to the~final form
\begin{multline}
  \IM^1_{(\nue)} \cdot \pdv{\fone}{\vmag}  
  = 
  \left(\frac{1}{\vmag}\IV_{\left(\MA \cdot \tE\right)}
  - \Delta\vmag \ID_{\left(\MA\right)} \right) \cdot
  \IM^{0^{-1}}_{(\nue)} \cdot 
  \left(\frac{1}{\vmag} \IV^T_{\left(\tE\right)} 
  + \Delta\vmag \matr{DIVE}
  \right)
  \cdot \pdv{\fone}{\vmag}
  \\
  +\frac{\Delta\vmag}{\vmag}\left(\IB_{\left( \tB \right)} 
  + \IM^1_{\left( \nutot \right)}\right) 
  \cdot \pdv{\fone}{\vmag} + 
  \frac{1}{\vmag}\left(\IB_{\left( \tB \right)} 
  + \IM^1_{\left( \nutot \right)}\right) 
  \cdot \fone^n
  + \frac{1}{\vmag}\IV_{\left(\MA \cdot \tE\right)} \cdot
  \pdv{\tvfM^n}{\vmag}
  \\ 
  +\left(\frac{1}{\vmag}\IV_{\left(\MA \cdot \tE\right)}
  - \Delta\vmag \ID_{\left(\MA\right)} \right) \cdot
  \IM^{0^{-1}}_{(\nue)} \cdot \matr{DIVE}\cdot\fone^n 
  - \ID_{\left(\MA\right)}\cdot\left(\tvfM^n + \davfzero^n\right),
  \label{eq:OOE_fullP1hosf1}
\end{multline}
where the~only unknown is $\pdv{\fone}{\vmag}$. Once \refeq{eq:OOE_fullP1hosf1}
is solved the~completion of the~solution 
$[\pdv{\davfzero}{\vmag}, \pdv{\fone}{\vmag}]$ is achieved by evaluating 
\refeq{eq:OOE_fullP1hosf0}.
\cite{Dobrev_Kolev_Rieben-High-order_curvilinear_finite_element_methods_for_Lagrangian_hydrodynamics}
\end{comment} % Implicit fully-discrete scheme

\section{Simulation results}\label{sec:results}
Three cases:
\begin{itemize}
  \item constant $n_e = 5\times10^{20}$ [1/cm$^3$], constant $\Zbar = 4$, 
  $T_e$ temperature profile taken from Impact simulation at 12 ps, see Figure 1
  %\figref{fig:Philippe_VFP_12ps}
  \item $n_e, T_e, \Zbar$ profiles taken from HYDRA simulation of Gadolinium
  hohlraum at 10 ps, see Figure 2 and Figure 3
  %\figref{fig:Gd_VFP_10ps_heatflux} and \figref{fig:Gd_VFP_10ps_kinetics} 
  \item Detail of distribution function under diffusive conditions of hydrogen 
  raising the~question of potentially outstanding properties of AWBS, since
  the uncorrected AWBS result is very close to KIPP full collision operator,
  see Figure 4 (consult Eq. (41) in Tzoufras OSHUN JCP 2011) 
\end{itemize}
\begin{figure}[tbh]
  \begin{center}
    \begin{tabular}{c}
      \includegraphics[width=1.0\textwidth]{../VFPdata/C7_heatflux_12ps.png} \\ 
      \includegraphics[width=1.0\textwidth]{../VFPdata/C7_kinetics_12ps.png}
    \end{tabular}
  \caption{
  Philippe's preferred test $\Zbar = 4$ at 12 ps.  
  }
  \end{center}
  \label{fig:Philippe_VFP_12ps}
\end{figure}

\begin{figure}[tbh]
  \begin{center}
    \begin{tabular}{c}
      \includegraphics[width=1.0\textwidth]{../VFPdata/GD_Hohlraum/fluxes_10ps.png} \\
      %\includegraphics[width=1.0\textwidth]{../VFPdata/GD_Hohlraum/diffusion_fluxes_10ps.png} 
      %\includegraphics[width=1.0\textwidth]{../VFPdata/GD_Hohlraum/fluxes_Efield_10ps.png} \\ 
      \includegraphics[width=1.0\textwidth]{../VFPdata/GD_Hohlraum/diffusion_fluxes_Efield_10ps.png} 
    \end{tabular}
  \caption{
  }
  \end{center}
  \label{fig:Gd_VFP_10ps_heatflux}
\end{figure}

\begin{figure}[tbh]
  \begin{center}
    \begin{tabular}{c}
       \includegraphics[width=1.0\textwidth]{../VFPdata/GD_Hohlraum/kinetics_10ps_xpointmax.png} \\ 
      \includegraphics[width=1.0\textwidth]{../VFPdata/GD_Hohlraum/kinetics_10ps_xpoint_1605microns.png}
    \end{tabular}
  \caption{
  Kinetics profiles for max(flux) point and 1605 microns point for the case of 10ps VFP temperature profile, ne and Z Hydra profiles. 
  }
  \end{center}
  \label{fig:Gd_VFP_10ps_kinetics}
\end{figure}

\begin{figure}[tbh]
  \begin{center}
    \begin{tabular}{c}
      \includegraphics[width=1.0\textwidth]{../VFPdata/KIPP_q_kinetics.png} \\
      \includegraphics[width=1.0\textwidth]{../VFPdata/KIPP_j_kinetics.png}
    \end{tabular}
  \caption{KIPP (by Johnathan) vs AWBS using 
  $\mfpei^* = \frac{\Zbar + 0.24}{\Zbar + 4.2}\mfpei, \Zbar=1, 
  \vth = \sqrt{\frac{\kB T_e}{\me}}$ 
  $f_1^{SH} = -\mfpei^*(v) \left(\frac{v^2}{2 \vth^2} - 4\right) 
  \frac{\vn\cdot\nabla T_e}{T_e} \fM,\quad 
  f_1^{KIPP} = -\mfpei^*(v) \left(\frac{3}{16}\frac{v^2}{\vth^2} - 1 
  - \frac{3}{2}\frac{\vth^2}{v^2} \right) 
  \frac{\vn\cdot\nabla T_e}{T_e} \fM$.
  }
  \end{center}
  \label{fig:}
\end{figure}

\begin{figure}[tbh]
  \begin{center}
    \begin{tabular}{c}
      \includegraphics[width=1.0\textwidth]{../results/fe_analysis/figs/P5_diffusive_heatflux_Z1_decelerating_Ezerothiter.png} \\
      \includegraphics[width=1.0\textwidth]{../results/fe_analysis/figs/P5_diffusive_heatflux_Z1_accelerating_Ezerothiter.png}
    \end{tabular}
  \caption{
  Decelerating (top) vs. accelerating (bottom) computations. 
  Zeroth E field iteration, i.e.
  no E field effect, of the diffusion regime conditions.
  }
  \end{center}
  \label{fig:}
\end{figure}

\clearpage

%\begin{figure}[tbh]
%  \begin{center}
%    \begin{tabular}{c}
%      \includegraphics[width=0.55\textwidth]{../results/fe_analysis/figs/AWBScorrection_fit.png} 
%    \end{tabular}
%  \caption{
%  Analytic fit to the~Z correction 
%  ($\nue^* = \frac{688.9 \Zbar + 114.4}{\Zbar^2 + 
%  1038 \Zbar + 474.1} \nue$, where $\nue = \nuei / \Zbar$) 
%  of the~diffusion asymptotic of AWBS with respect to SH.
%  }
%  \end{center}
%  \label{fig:}
%\end{figure}

%% The Appendices part is started with the command \appendix;
%% appendix sections are then done as normal sections
%% \appendix

%% \section{}
%% \label{}

%% References
%%
%% Following citation commands can be used in the body text:
%% Usage of \cite is as follows:
%%   \cite{key}         ==>>  [#]
%%   \cite[chap. 2]{key} ==>> [#, chap. 2]
%%

%% References with bibTeX database:

\bibliographystyle{elsarticle-num}
\bibliography{NTH}

%% Authors are advised to submit their bibtex database files. They are
%% requested to list a bibtex style file in the manuscript if they do
%% not want to use elsarticle-num.bst.

%% References without bibTeX database:

% \begin{thebibliography}{00}

%% \bibitem must have the following form:
%%   \bibitem{key}...
%%

% \bibitem{}

% \end{thebibliography}


\end{document}

%%
%% End of file 
