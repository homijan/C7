\section{Conclusions}
\label{sec:Conclusions}

\begin{itemize}
  \item The~most important point is that we introduce a~collision operator, 
    which is coherent with the full FP, i.e. no extra dependence on $\Zbar$.
  \item Touch pros/contras of linearized FP in Aladin and Impact vs AWBS
  \item Raise discussion about what is the weakest point of AP1 for high Kns: 
    the~velocity limit or phenomenological Maxwellization?
  \item Summarize useful outcomes related to plasma physics as
    the competition between collisions and electric field in the electron 
	stopping, then the knowledge about the nonlocal electron population 
	(preheat electrons can be tracked back to the point of source according 
	to their dominant velocity), and the last information about the~tendency 
	of the~velocity maximum in $q_1$ with respect to $\Zbar$ and Kn$^e$.
  \item Emphasize the~good results of Aladin (compared to Impact) and also
    outstanding results of Calder while being PIC. 
  \item Electric field plays an important role in nonlocal electron kinetics
    and nonlocal Ohm's law provides the necessary equation to treat it properly.
  \item In coherence with the latter, the dfdv must be treated properly in
    nolocal electron kinetics, and so, AWBS can be included without any extra
	effort thus making it way much suitable than BGK, which is outperformed
	by AWBS when compared to FP.

\end{itemize}

%LPI Kirkwood 2013 \cite{Kirkwood_NIFLPI_PPCF2013}
%Nernst Walsh 2017 \cite{Walsh_Nernst_PRL2017}
%Heat wave velocity measurement Schurtz 2007 \cite{Schurtz_2007}
%Nonlocal heat flux in laser-plasma corona Henchen 2018 \cite{Henchen_PRL2018}
%Return current instability Glenzer 2002 \cite{Glenzer_PRL2002}
%PETE code \cite{pete-code}
