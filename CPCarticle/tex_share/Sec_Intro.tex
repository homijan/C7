\section{Introduction}
\label{sec:Intro}
%In dealing with the nonequilibrium properties of
%systems whose particles obey an inverse-square law
%of interaction, it is convenient to make use of the fact
%that under most circumstances small-angle collisions
%are much more important than collisions resulting in
%large momentum changes. This leads to the method
%often used for treating such systems, in which one
%expands the collision integrand of the Boltzmann change per
%equation in powers of the momentum collision.

The~first modern attempts at kinetic modeling of plasma can be traced back 
to the~fifties, when Cohen, Spitzer, and Routly (CSR) \cite{CSR_1950} 
demonstrated that the~effect of Coulomb collisions between electrons and ions 
in the~ionized gas predominantly results 
from frequently occurring events of cumulative small deflections 
rather than occasional close encounters. This effect was originally described
by Jeans in \cite{Jeans_BOOK1929} and 
Chandrasekhar \cite{Chandrasekhar_RMP1943} 
proposed to use the~diffusion equation model of the~Vlasov-Fokker-Planck type 
(VFP)~\cite{Planck_1917}.
%CPR comment - do you need to start so far back in the history of plasma kinetic theory?


%A more generally valid approach to the problem of
%treating changes in a distribution function resulting
%each of which from frequently occurring "events",
%produces a small change in the momentum of a particle,
%is to use the Fokker-Planck equation \cite{Planck_1917}, 
%which has been discussed by Spitzer and collaborators 
%\cite{SpitzerHarm_PR1953} . 
A~classical paper by Spitzer and Harm (SH) 
\cite{SpitzerHarm_PR1953} provides the~computation of 
the~electron distribution function (EDF) in a~plasma (from low to high $\Zbar$)
with a~temperature gradient accounting for e-e and e-i collisions.
The~resulting expressions for current and heat flux are widely used in plasma 
hydrodynamic models.
%used the formalism of the VFP equation to evaluate the collision terms of 
%the~Boltzmann equation under the assumptions that 
%(a) the events producing changes in particle momenta
%are two-body interactions described by the associated
%differential scattering cross sections, and 
%(b) that 
The~distribution function based on the spherical harmonics method in 
its first approximation (P1) \cite{Jeans_MNRAS1917} is of the form 
$f^0+\mu f^1$, where $f^0$ and $f^1$ 
are isotropic and $\mu$, is the direction cosine between the particle 
trajectory and some preferred direction in space. It should be emphasized that
the~SH solution expresses a~small perturbation of equilibrium, i.e. that 
$f^0$ is the~Maxwell-Boltzmann distribution and $\mu f^1$ represents 
a~very small anisotropic deviation. 
This approximation holds for $L_T\gg\mfpe$, 
a~condition which is often invalid in laser plasmas, 
where $L_T$ is the~temperature length scale and $\mfpe$ 
the~mean free path of electrons. It is worth mentioning, that electrons having
3 to 4 times the~thermal velocity are dominantly responsible for heat-flow
and that those faster than 6 times the~thermal velocity can be completely 
neglected in this local theory.

%It is the purpose of this paper to present the mechanics 
%of two-body collisions in a somewhat simplified
%form, and to derive the Fokker-Planck equation for an function. 
%From this general arbitrary distribution equation such special cases 
%as those of Chandrasekhar and Spitzer may easily be obtained. 
%For more complex situations the equation is suitable for integration by an
%electronic computer \cite{Rosenbluth_PR1957}.

The~actual cornerstone of the~modern VFP simulations was set in place
by Rosenbluth \cite{Rosenbluth_PR1957}, when he derived a~simplified form 
of the~VFP equation for a~finite expansion of the~distribution function,
where all the~terms are computed according to plasma conditions, including
$f^0$, which of course needs to tend to the~Maxwell-Boltzmann distribution.
% CPR comment - again not sure you need to go back this far in history.
Consequently, the~pioneering work on numerical solution of the~VFP equation
\cite{Bell_1981_83, Matte_1982_86} revealed the~importance of the~nonlocal
electron transport in laser-heated plasmas. 
In particular, that the~heat flow down steep temperature gradients in 
unmagnetised plasma cannot be described by the classical, local fluid
description of transport \cite{SpitzerHarm_PR1953, Braginskii_1965_3}.
This is due to the~classical $f^1$ is not a~small deviation 
(especially for electrons having 3 to 4 times the~thermal velocity), 
i.e. $f^0\sim f^1$ characterized by $L_T\sim\mfpe$.
It was also shown that a~thermal transport inhibition \cite{Bell_1981_83} 
around the peak of the temperature gradient, and a~nonlocal preheat 
ahead of the main heat wave front, naturally appear. 
These effects are attributed to significant deviations 
of $f^0$ from Maxwellian distribution.


%Previously the Vlasov-Fokker-Planck (VFP) equation has been solved
%numerically ignoring magnetic fields in 1-D (1 spatial dimension) 
%\cite{Bell_1981_83, Matte_1982_86} to address heatflow down steep temperature 
%gradients in unmagnetised plasma. Under these conditions the classical, fluid
%description of transport \cite{SpitzerHarm_PR1953, Braginskii_1965_3}, 
%which makes the local approximation, 
%breaks down. They found that nonlocal effects are responsible for 
%thermal transport inhibition \cite{Bell_1981_83}.

%Particle-in-cell (PIC) codes \cite{PIC_Birdsall_Langdon_1985}
%%\cite{Dawson_PoF1962, PIC_Birdsall_Langdon_1985} 
%%and Vlasov codes $[23]$ 
%are fully kinetic and are ideally suited to collisionless plasmas. 
%Typically PIC codes tend to use explicit methods and consequently require a
%time step small enough to resolve the fastest frequency present in the problem.
%They also suffer from the so called "finite grid instability" 
%\cite{PIC_Birdsall_Langdon_1985} 
%whereby the plasma numerically heats up until the electron debye length is
%resolved by the grid. These limitations mean that multidimensional, explicit, 
%PIC cannot readily simulate the low temperature, high density plasma 
%created inlaser–solid interactions. 
%%PIC codes utilising implicit methods do exist 
%%$[24]$. Implicitness greatly relaxes the time step constraint and mitigates 
%%the effects of the finite grid instability, so that lower temperatures 
%%and higher densities can be dealt with. 
%One spatial dimension PIC codes with electron–ion collisions have successfully 
%been applied to fast electron transport through solid density plasma 
%\cite{Guerin_PPCF1999, Perez_PoP2012}. 
%%Two spatial dimension PIC codes with electron collisions also exist, both
%%explicit $[25]$ and implicit $[26]$. 

%Explicit 2-D PIC is unable to access conditions where collisions dominate for
%the bulk of the electrons, though. Even though implicit methods overcome 
%this particular problem, the PIC method in general struggles to adequately 
%resolve the distribution function in a given cell when a realistic sized, 
%2-D problem is addressed. 
%Statistical noise and under resolution of 
%the electron distribution lead to an inaccurate treatment of collisions and 
%can overwhelm real physical effects present.

Nevertheless, numerical solution of the~VFP equation even in the~Rosenbluth
formalism remains very challenging computationally, because the~e-e collision
integral is nonlinear. More simple linear forms of e-e collision operator
are needed. Although some VFP simulations on experimentally relevant timescales 
have been performed (for recent examples see 
\cite{Hawreliak04,Ridgers08,Willingale10,Bissell10,Joglekar14,Joglekar16,Henchen_PRL2018}, 
an extensive review has been conducted by Thomas et al. \cite{Thomas13}), 
their relative computational inefficiency severely limits the range of 
simulations that can be performed.

It is the~purpose of this paper to present an~efficient alternative 
to a full solution of the VFP equation to accurately calculate 
nonlocal transport, based on the~Albritton-Williams-Bernstein-Swartz 
collision operator (AWBS) \cite{AWBS_PRL1986}.
In Section \ref{sec:AWBSmodel} we propose a~modified form of 
the~AWBS collision operator, where its important properties are further
presented in Section \ref{sec:DiffusiveKinetics} with the~emphasis on its
comparison to the~full VFP solution in local diffusive regime. 
Section \ref{sec:BenchmarkingAWBS} focuses on the~performance of the~AWBS 
transport equation model compared to modern kinetic codes including VFP codes
Aladin and Impact \cite{Kingham_JCP2004}, and PIC code Calder 
\cite{Perez_PoP2012}, where the~cases related to real
laser generated plasma conditions are studied. Finally, the~most important
outcomes of our research are concluded in Section \ref{sec:Conclusions}. 
