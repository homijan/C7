\section{The~AWBS kinetic model}
\label{sec:AWBSmodel}

The~electrons in plasma can be modeled by the~deterministic Vlasov model 
of charged particles
\begin{equation}
  \pdv{\ft}{t} + \vv\cdot\gx \ft + 
  \frac{\qe}{\me}\left(\E + \frac{\vv}{c}\vect{\times}\B\right)\cdot\gv \ft = 
  C_{ee}(f) + C_{ei}(f) ,
  \label{eq:kinetic_equation}
\end{equation}
where $\ft(t, \vect{x}, \vect{v})$ represents the~density function of electrons
at time $t$, spatial point $\vect{x}$, and velocity $\vv$, $\E$ and $\B$ are 
the~electric and magnetic fields in plasma, $\qe$ and $\me$ being 
the~charge and mass of electron.

The~general form of the~e-e collision operator 
$C_{ee}$ is the~Fokker-Planck form published by Landau \cite{Landau_1936}
\begin{equation}
  C_{FP}(\ft) =
  \Gamma\int \gv\gv(\vv - \vvb) \cdot \left(
  \ft\, \gvb \ft - \ft\, \gv \ft \right)\, \dI\vvb ,
  \label{eq:LFP_model}
\end{equation}
where $\Gamma = \frac{4\pi\qe^4\lnc}{\me^2}$ and 
$\lnc$ is the~Coulomb logarithm.
The~e-i collision operator $C_{ei}$ could be expressed
in a~simpler form since massive ions are considered 
to be motionless compared to electrons \corrCPR{during a collision}. %CPR comment - sorry for the pedantic distinction just clarifying that it is possible to account for ion motion in the VFP equation while ignoring it in the collision operator
 The~scattering operator accounts
for the~change of electron velocity without change in the~velocity magnitude\corrCPR{, i.e. angular scattering}. 
It is expressed in spherical coordinates as
\begin{equation}
  C_{ei}(\ft) = \frac{\nuei}{2}
  \left(\pdv{}{\mu}\left((1 - \mu^2)\pdv{\ft}{\mu}\right)
  + \frac{1}{\sin^2\phi}\frac{\partial^2 \ft}{\partial\theta^2} \right) ,
  \label{eq:ei_scattering}
\end{equation}
where $\mu = \cos\phi$, $\phi$ and $\theta$ are the~polar and azimuthal 
angles, and $\nuei = \frac{\Zbar n_e \Gamma}{\vmag^3}$ is the~e-i
collision frequency.

The~e-e collision operator needs to be linearized for efficient computation\corrCPR{\st{s}}.
Fis\corrCPR{c}h introduced a~linear form of $C_{ee}$ in \cite{Fish_RMP1987} 
in the~high-velocity limit ($\vmag\gg\vth$) electron collision operator
\begin{multline}
  C_{H}(\ft) = \vmag \nue \pdv{}{\vmag}\left(\ft + 
  \frac{\vth^2}{\vmag}\pdv{f}{\vmag}\right) \\
  + \frac{\nue}{2}\left( 1 - \frac{\vth^2}{2\vmag^2}\right) 
  \left(\pdv{}{\mu}\left((1 - \mu^2)\pdv{f}{\mu}\right)
  + \frac{1}{\sin^2\phi}\frac{\partial^2f}{\partial\theta^2} \right)
  , \label{eq:HighVelocity_model}
\end{multline}
where $\nue = \frac{n_e \Gamma}{\vmag^3}$ is the~e-e collision 
frequency and $\vth = \sqrt{\frac{\kB T_e}{\me}}$ is the~electron thermal 
velocity.
The~linear form of $C_{H}$ arises from an~assumption that the~fast electrons 
predominantly interact with the~thermal (slow) electrons, 
which \corrCPR{ is an important simplification to} the~form \eqref{eq:LFP_model}.
However the~diffusion term in the~e-e collision operator 
\eqref{eq:HighVelocity_model} still presents numerical difficulties.

A~yet simpler form of the~collision operator of electrons was proposed in 
\cite{Sorbo_2015}
\begin{multline}
  C_{AWBS}(\ft) = \vmag \nue^*\pdv{}{\vmag}\left(\ft - \fM\right) \\
  + \frac{\nuei + \nue^*}{2} 
  \left(\pdv{}{\mu}\left((1 - \mu^2)\pdv{f}{\mu}\right)
  + \frac{1}{\sin^2\phi}\frac{\partial^2f}{\partial\theta^2} \right)
  , \label{eq:AWBS_model}
\end{multline}
where $\fM = \frac{n_e}{(2\pi)^{\frac{3}{2}}\vth^3}
\exp\left(-\frac{\vmag^2}{2\vth^2}\right)$ 
is the~Maxwell-Boltzmann equilibrium distribution.
Here, the~first term representing the AWBS operator \cite{AWBS_PRL1986}
accounts for relaxation to equilibrium due to the~e-e collisions, and 
the~second term accounts for the~e-i and e-e collisions contribution 
to scattering.

A~method of angular momenta for the~solution of the~electron kinetic equation
with the~collision operator \eqref{eq:AWBS_model} 
was introduced in \cite{Sorbo_2015, Sorbo_2016}. 

In \eqref{eq:AWBS_model} we have introduced a~modified e-e collision frequency
$\nue^*$ in order to address a~proper behavior with respect to $\Zbar$, which
is further analyzed in Section \ref{sec:DiffusiveKinetics} and promising 
results compared to the~full FP operator are presented.

%The~Maxwell-Boltzmann averaged e-e scattering in 
%\eqref{eq:HighVelocity_model} can be approximated as 
%$\nue \int\left(1 - \frac{\vth^2}{2\vmag^2}\right)\fM 4\pi\vmag^2\, \dI\vmag = 
%\frac{\nue}{2}$.
