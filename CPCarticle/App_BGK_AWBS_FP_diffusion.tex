\section{Background of the local diffusive regime theory}
\label{app:DiffusiveKinetics}

The~left hand side of \eqref{eq:1D_kinetic_equation} acts on 
\eqref{eq:f_approximation} as
\begin{multline}
  \mu\left(\pdv{\tilde{\ft}}{z} 
  + \frac{\qe\Ez}{\me\vmag}\pdv{\tilde{\ft}}{\vmag}\right) 
  + \frac{\qe\Ez}{\me}\frac{1-\mu^2}{\vmag^2}\pdv{\tilde{\ft}}{\mu} = \\
  \mu\left(\pdv{\ft^0}{z} + \frac{\qe\Ez}{\me\vmag}\pdv{\ft^0}{\vmag}\right) 
  + \frac{\qe\Ez}{\me\vmag^2} \ft^1 + O(\mu^2) .
  \label{app_eq:LHS_kinetic_equation}
\end{multline}
The~action on \eqref{eq:f_approximation} of the~BGK operator 
\eqref{eq:BGK_model_1D} as used in \eqref{eq:1D_kinetic_equation} reads
\begin{eqnarray}
  \frac{1}{\vmag}C_{BGK}(\tilde{\ft})
  &=&
  \frac{\tilde{\ft} - \fM}{\mfpe}
  + \frac{1}{2}\left(\frac{\Zbar}{\mfpe} + \frac{1}{\mfpe}\right)
  \pdv{}{\mu}(1 - \mu^2)\pdv{\tilde{\ft}}{\mu} ,\nonumber\\
  &=&  \frac{\ft^0 - \fM}{\mfpe}
  - \mu \frac{\Zbar}{\mfpe}\ft^1 .
  \label{app_eq:BGK_model_1D}
\end{eqnarray}
Consequently, if the~isotropic and anisotropic parts of 
\eqref{app_eq:LHS_kinetic_equation} and \eqref{app_eq:BGK_model_1D} are 
compared, one finds the~following equations 
\begin{eqnarray}
  \ft^0 &=& \fM + \frac{\mfpe\qe\Ez}{\me\vmag^2}f^1 ,
  \label{app_eq:BGK_f0} \\
  \ft^1 &=& - \frac{\mfpe}{\Zbar}
  \left( \pdv{\ft^0}{z} + \frac{\qe\Ez}{\me\vmag}\pdv{\ft^0}{\vmag} \right) . 
  \label{app_eq:BGK_f1}
\end{eqnarray}
It is valid to assume that $\ft^0 = \fM$ from \eqref{app_eq:BGK_f0}. Then,
\begin{equation}
  \ft^1_{BGK} = - \frac{\mfpe}{\Zbar}
  \left( \pdv{\fM}{z} + \frac{\qe\Ez}{\me\vmag}\pdv{\fM}{\vmag} \right) . 
  \label{app_eq:BGK_f1_fM}
\end{equation}
The~\textit{quasi-neutrality} constraint, corresponding to a~zero 
electric reads
\begin{equation}
\vect{j} \equiv \qe \int \vv \tilde{\ft} \, \dI\vv = \vect{0} .
\end{equation}
In the~case of the BGK EDF, in particular its~anisotropic part 
\eqref{app_eq:BGK_f1_fM}, the~zero current condition takes the~form
\begin{equation}
  2\pi\int_{-1}^{1} \int_{\vmag} \vmag \mu^2 \ft^1_{BGK}\, \dI\vmag\dI\mu = 0 ,
  \nonumber
\end{equation}
which leads to the~electric field (same as the~classical Lorentz electric field
\cite{Lorentz_1905})
\begin{equation}
  \Ez = \frac{\me\vth^2}{\qe}\left(\frac{1}{L_{n_e}} 
  + \frac{5}{2}\frac{1}{L_{T_e}} \right) .
  \label{app_eq:BGK_Efield}
\end{equation}
It is worth mentioning, that the~deviation of $\ft^0$ from $\fM$ in 
\eqref{app_eq:BGK_f0} can be written as $\left(\frac{\mfpe}{L_{n_e}} 
  + \frac{5}{2}\frac{\mfpe}{L_{T_e}} \right)\frac{\vth^2}{\vmag^2}f^1$,
where naturally arises the~Knudsen number 
$Kn = \frac{\mfpe}{L_{n_e}} + \frac{5}{2}\frac{\mfpe}{L_{T_e}}$ comprising
both contributions of electron density and temperature gradients.

In the~case of the~AWBS operator \eqref{eq:AWBS_model} used in 
\eqref{eq:1D_kinetic_equation}, its action on \eqref{eq:f_approximation} reads
\begin{eqnarray}
  \frac{1}{\vmag}C_{AWBS}(\tilde{\ft})
  &=& 
  \frac{\vmag}{\mfpe} \pdv{}{\vmag}\left(\tilde{\ft} - \fM\right) \nonumber \\
  && + \frac{1}{2}\left(\frac{\Zbar}{\mfpe} + \frac{1}{\mfpe}\right)
  \pdv{}{\mu}(1 - \mu^2)\pdv{\tilde{\ft}}{\mu}  \nonumber \\
  &=& \frac{\vmag}{\mfpe} \pdv{}{\vmag}\left(\ft^0 - \fM\right) \nonumber \\ 
  &&\, + \mu\left(\frac{\vmag}{\mfpe} \pdv{\ft^1}{\vmag} 
  - \frac{\Zbar+1}{\mfpe}\ft^1\right) .
  \label{app_eq:AWBS_model_1D}
\end{eqnarray}
One finds the~following equations if the~isotropic and anisotropic parts of 
\eqref{app_eq:LHS_kinetic_equation} and \eqref{app_eq:AWBS_model_1D} are 
compared 
\begin{eqnarray}
  \pdv{}{\vmag}\left( \ft^0 -\fM\right) &=& 
  \frac{\mfpe\qe\Ez}{\me\vmag^2}\frac{\ft^1}{\vmag} ,
  \label{app_eq:AWBS_f0} \\
  \frac{\vmag}{\mfpe} \pdv{\ft^1}{\vmag} 
  - \frac{\Zbar+1}{\mfpe}\ft^1 &=&
  \pdv{\ft^0}{z} + \frac{\qe\Ez}{\me\vmag}\pdv{\ft^0}{\vmag} .
  \label{app_eq:AWBS_f1} 
  %\\  
  %\frac{\vmag}{\Zbar}\pdv{f^1}{\vmag} + \frac{4}{\Zbar}f^1 
  %- \frac{\Zbar + 1}{\Zbar} f^1 &=&
  %\pdv{f^0}{z} + \frac{\tilde{E}_z}{\vmag}\pdv{f^0}{\vmag}
  %\nonumber
\end{eqnarray}
If we assume that $\pdv{\ft^0}{\vmag} = \pdv{\fM}{\vmag}$, i.e. $\ft^0 = \fM$,
the~anisotropic part of the~AWBS operator is governed by the~equation
\begin{equation}
  \pdv{\ft^1_{AWBS}}{\vmag} 
  - \frac{\Zbar+1}{\vmag}\ft^1_{AWBS} =
  \frac{\mfpe}{\vmag} 
  \left(\pdv{\fM}{z} + \frac{\qe\Ez}{\me\vmag}\pdv{\fM}{\vmag}\right) .
  \label{app_eq:AWBS_f1_fM}
\end{equation}
Even though it is not straightforward, the~electric field in 
\eqref{app_eq:AWBS_f1_fM} (solved numerically) providing a~zero current 
exactly matches \eqref{app_eq:BGK_Efield}. Consequently, the~deviation of
$\pdv{\ft^0}{\vmag}$ from $\pdv{\fM}{\vmag}$ in 
\eqref{app_eq:AWBS_f0} can be written as 
$Kn\frac{\vth^2}{\vmag^2}\frac{f^1}{\vmag}$.

Finally, it should be stressed, that the~concept of locality expressed as 
$Kn\ll1$ is crucial for our \textit{local diffusive regime} analysis, 
because it provides sufficient Maxwellization, i.e.  \eqref{app_eq:BGK_f0} and
\eqref{app_eq:AWBS_f0}, and correspondingly, \eqref{app_eq:BGK_f1_fM} 
and \eqref{app_eq:AWBS_f1_fM} are valid models. 
