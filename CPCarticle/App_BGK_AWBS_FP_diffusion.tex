\section{Analysis of local diffusive regime}
\label{app:DiffusiveKinetics}

In order to analyze the~local diffusive regime, 
we use the~BGK collision operator 
\eqref{eq:BGK_model_1D} 
\begin{equation}
  \frac{1}{\vmag}C(\tilde{\ft})
  =
  \frac{\fM - \tilde{\ft}}{\mfpe}
  + \frac{1}{2}\left(\frac{\Zbar}{\mfpe} + \frac{1}{\mfpe}\right)
  \pdv{}{\mu}(1 - \mu^2)\pdv{\tilde{\ft}}{\mu} 
  ,\nonumber
\end{equation}
to write explicitly \eqref{eq:1D_kinetic_equation} 
for \eqref{eq:f_approximation} 
\begin{multline}
  %\mu\left(\pdv{\tilde{\ft}}{z} 
  %+ \frac{\qe\Ez}{\me\vmag}\pdv{\tilde{\ft}}{\vmag}\right) 
  %+ \frac{\qe\Ez}{\me}\frac{1-\mu^2}{\vmag^2}\pdv{\tilde{\ft}}{\mu} = \\
  \frac{\qe\Ez}{\me\vmag^2} \ft^1 
  + \mu^2\left[\pdv{\ft^1}{z} 
  + \frac{\qe\Ez}{\me\vmag}\pdv{\ft^1}{\vmag} - \frac{\qe\Ez}{\me\vmag^2} \ft^1
  \right] \\ 
  + \mu\left[\pdv{\ft^0}{z} 
  + \frac{\qe\Ez}{\me\vmag}\pdv{\ft^0}{\vmag}\right] = 
  %\label{app_eq:LHS_kinetic_equation}
%\end{multline}
%and the~right hand side  reads
%\begin{eqnarray}
  %\frac{1}{\vmag}C_{BGK}(\tilde{\ft})
  %&=&
  %\frac{\fM - \tilde{\ft}}{\mfpe}
  %+ \frac{1}{2}\left(\frac{\Zbar}{\mfpe} + \frac{1}{\mfpe}\right)
  %\pdv{}{\mu}(1 - \mu^2)\pdv{\tilde{\ft}}{\mu} ,\nonumber\\
  \frac{\fM - \ft^0}{\mfpe} - \mu \frac{\Zbar + 2}{\mfpe}\ft^1
  .
  \label{app_eq:BGK_model_1D}
\end{multline}
The~P1 form \eqref{eq:f_approximation} represents a~low anisotropy
expansion to the~first to Legendre polynomials $P_0 = 1$ and $P_1 = \mu$, 
where the~projection of a~function $f(\mu)$ 
to a~Legendre polynomial $P_k(\mu)$ reads
$\mathcal{P}_k(f) = \int_{-1}^1 P_k(\mu) f(\mu) \dI\mu$, in particular giving
the~orthogonality $\mathcal{P}_0(P_1) = \mathcal{P}_1(P_0) = 0$.

Consequently, the~projections of the~equation \eqref{app_eq:BGK_model_1D}, 
i.e. $\mathcal{P}_0\eqref{app_eq:BGK_model_1D}$ 
and $\mathcal{P}_1\eqref{app_eq:BGK_model_1D}$, define 
\begin{eqnarray}
    \ft^0 &=& \fM - \frac{\mfpe}{3}\left[\frac{2\qe\Ez}{\me\vmag^2} \ft^1 
  + \pdv{\ft^1}{z} + \frac{\qe\Ez}{\me\vmag}\pdv{\ft^1}{\vmag}\right]
  ,
  \label{app_eq:BGK_f0} \\
  \ft^1 &=& - \frac{\mfpe}{\Zbar + 2}
  \left[ \pdv{\ft^0}{z} + \frac{\qe\Ez}{\me\vmag}\pdv{\ft^0}{\vmag} \right]
  . 
  \label{app_eq:BGK_f1}
\end{eqnarray}
It is valid to assume that $\ft^0 \approx \fM$, i.e. that $\fM \gg
\frac{\mfpe}{3}\left[\pdv{\ft^1}{z} +
\frac{\qe\Ez}{\me\vmag^3}\pdv{\vmag^2\ft^1}{\vmag}\right]$ 
in \eqref{app_eq:BGK_f0}. The~\textit{quasi-neutrality} constraint 
\eqref{eq:j0_P1} applied to \eqref{app_eq:BGK_f1} along with 
$\ft^0 = \fM$ leads to the~electric field 
(same as the~classical Lorentz electric field $\E_L$ \cite{Lorentz_1905})
\begin{equation}
  \Ez = \frac{\me\vth^2}{\qe}\left(\frac{1}{L_{n_e}} 
  + \frac{5}{2}\frac{1}{L_{T_e}} \right) 
  ,
  \label{app_eq:BGK_Efield}
\end{equation}
and the~anisotropic part of EDF takes the~form \eqref{eq:BGK_approximation}.
It should be noticed that $\ft^0$ equilibrates to $\fM$ 
as $O\left( \text{Kn}^2\right)$ since $\ft^1 \sim \text{Kn} \fM$ 
and $\mfpe \Ez \sim \text{Kn}$.
%$\frac{\mfpe\qe\Ez}{\me\vmag^2}f^1 \sim Kn^2$ and  
%$f^2 \sim Kn^2~\vmag/\vth$, when $\ft^0 = \fM$ and $\Ez=\E_L$ in 
%\eqref{app_eq:BGK_f1}.

The~AWBS operator \eqref{eq:AWBS_model} applied to 
\eqref{eq:f_approximation} reads
\begin{eqnarray}
  \frac{1}{\vmag}C_{AWBS}(\tilde{\ft})
  &=& 
  \frac{\vmag r_A}{\mfpe} \pdv{}{\vmag}\left(\tilde{\ft} - \fM\right) 
  \nonumber \\
  && + \frac{1}{2}\left(\frac{\Zbar}{\mfpe} + \frac{r_A}{\mfpe}\right)
  \pdv{}{\mu}(1 - \mu^2)\pdv{\tilde{\ft}}{\mu}  \nonumber \\
  &=& \frac{\vmag r_A}{\mfpe} \pdv{}{\vmag}\left(\ft^0 - \fM\right) \nonumber \\ 
  &&\, + \mu\left(\frac{\vmag r_A}{\mfpe} \pdv{\ft^1}{\vmag} 
  - \frac{\Zbar+r_A}{\mfpe}\ft^1\right) ,
  \label{app_eq:AWBS_model_1D}
\end{eqnarray}
where $\nue^* = r_A \nue = \frac{\vmag r_A}{\mfpe}$ with $r_A$ being a~scaling
parameter of the~standard e-e collision frequency. 
The~$\mathcal{P}_0$ and $\mathcal{P}_1$ projections 
of the~equation \eqref{app_eq:BGK_model_1D} using 
\eqref{app_eq:AWBS_model_1D} instead of BGK then define
\begin{eqnarray}
  \pdv{}{\vmag}\left( \ft^0 -\fM\right) &=& 
  \frac{\mfpe}{3}\left[\pdv{\ft^1}{z} +
  \frac{\qe\Ez}{\me\vmag^3}\pdv{\vmag^2\ft^1}{\vmag}\right] ,
  \label{app_eq:AWBS_f0} \\
  \pdv{\ft^1}{\vmag} 
  - \frac{\Zbar + r_A}{{\vmag r_A}}\ft^1 &=& \frac{\mfpe}{\vmag r_A}
  \left[\pdv{\ft^0}{z} + \frac{\qe\Ez}{\me\vmag}\pdv{\ft^0}{\vmag}\right] 
  .
  \label{app_eq:AWBS_f1} 
  %\\  
  %\frac{\vmag}{\Zbar}\pdv{f^1}{\vmag} + \frac{4}{\Zbar}f^1 
  %- \frac{\Zbar + 1}{\Zbar} f^1 &=&
  %\pdv{f^0}{z} + \frac{\tilde{E}_z}{\vmag}\pdv{f^0}{\vmag}
  %\nonumber
\end{eqnarray}
If we assume that $\pdv{\ft^0}{\vmag} = \pdv{\fM}{\vmag}$, i.e. $\ft^0 = \fM$,
the~anisotropic part of the~AWBS operator is governed by the~equation
\eqref{eq:AWBS_f1}, which can be simplified to the~form
\begin{equation}
  \pdv{\vmag^a \ft^1}{\vmag} = 
  \frac{\mfpe \vmag^{a-1}}{r_A}
  \left( b~\frac{\vmag^2}{2 \vth^2} + c\right)\fM ,
  \nonumber %\label{app_eq:AWBS_f1_ODE}
\end{equation}
with an~integral solution (using $\ft^1(\infty) = 0$)
\begin{equation}
  \ft^1(v) = - \frac{d}{\vmag^a} 
  \int_{\frac{\vmag^2}{2\vth^2}}^\infty (b \tilde{v}^{\frac{a+6}{2} - 1} 
  + c \tilde{v}^{\frac{a+4}{2} - 1}) \exp(-\tilde{v}) d\tilde{v} 
  ,
  \label{app_eq:gammaInt}
\end{equation}
where the~analytical solution to \eqref{app_eq:gammaInt} can be obtained 
in the~form of upper incomplete gamma function shown 
in~\secref{sec:AWBSDiffusiveRegime}, where the~coefficients 
$a, b, c,$ and $d$ are defined. 
However, since the~analytical formula \eqref{eq:AWBS_analytic_solution} 
is valid for $a>-4$, we also adopt the~implicit Euler 
numerical integration with $\Delta v < 0$, where we integrate from high 
electron velocity ($v_{max} = 7\vth$) to zero  
(using 10$^5$ $\Delta v$ steps). The~numerical approach is used for the~case
of $\Zbar > 1.5$ and $r_A = \frac{1}{2}$.

\begin{comment}
\begin{equation}
    b = \frac{1}{L_{\Te}},
	~ c = \frac{1}{L_{\ed}} - \frac{3}{2}\frac{1}{L_{\Te}} 
	- \frac{\qe\Ez}{\me\vth^2},
   ~d = \frac{2^{\frac{a + 2}{2}} \vth^{a + 1}}{r_A (2\pi)^{\frac{3}{2}} \Gamma}
  ,
  \nonumber
\end{equation}
However, an~analytic solution can be found to
\begin{equation}
	\pdv{\vmag^a \ft^1}{\vmag} = \left(\frac{\vmag^{3+a}}{b} 
    + \frac{\vmag^{5+a}}{c}\right) \exp\left(-\frac{\vmag^2}{2\vth^2}\right)
  ,
  \nonumber
\end{equation}
where 
\begin{equation}
  a = -\frac{\Zbar + r_A}{r_A},
  ~b = \frac{\Gamma(2\pi)^{\frac{3}{2}}\vth^3 L_{\Te}}{4},
  ~c = \Gamma(2\pi)^{\frac{3}{2}}2\vth^5 L_{\Te} .
  \nonumber
\end{equation}
Then,
\begin{equation}
  \ft^1(\vmag) = \frac{\vth^2}{\vmag^a} \int_{\vmag}^\infty   
  \left(\frac{\tilde{\vmag}^{2+a}}{b} + \frac{\tilde{\vmag}^{4+a}}{c}\right) 
  \exp\left(-\frac{\tilde{\vmag}^2}{2\vth^2}\right) 
    \frac{\tilde{\vmag}}{\vth^2} \dI \tilde{\vmag} , 
  \nonumber
\end{equation}
Substitution $\hat{v} = \frac{\tilde{\vmag}^2}{2\vth^2}$ gives 
$\dI\hat{v} = \frac{\tilde{\vmag}}{\vth^2} \dI \tilde{\vmag}$ and 
$\tilde{\vmag} = \vth\sqrt{2\hat{v}}$.
\begin{equation}
  \ft^1(\vmag) = \frac{2^{\frac{4+a}{2}}
  \vth^{6+a}}{\vmag^a} \int_{\frac{\vmag^2}{2\vth^2}}^\infty   
    \left(\frac{{\hat{v}}^{\frac{4+a}{2}-1}}{2\vth^2 b} 
    + \frac{{\hat{v}}^{\frac{6+a}{2}-1}}{c}\right) 
    \exp\left(-\hat{v}\right) \dI \hat{v} , 
  \nonumber
\end{equation}

Its purpose is to
match AWBS heat flux to results obtained by Spitzer and Harm 
\cite{SpitzerHarm_PR1953} obtained for any $\Zbar$. 
\secref{sec:FPDiffusiveRegime} shows that this match can be found with 
a~constant $r_A=\frac{1}{2}$.
\end{comment}

\section{AP1 electric field limit}
\label{app:AP1limit}

We have encountered a~very specific property of the~AP1 model
with respect to the~electric field magnitude. The~easiest way how to 
demonstrate this is to write the~model equations \eqref{eq:AP1f0} and 
\eqref{eq:AP1f1} in 1D (z-axis). Then, due to its linear nature, it is easy 
to eliminate one of the~partial derivatives with respect to $\vmag$, i.e. 
$\pdv{\fzero}{\vmag}$ or $\pdv{\fonez}{\vmag}$. 
In the~case of elimination of $\pdv{\fzero}{\vmag}$ 
one obtains the~following equation
\begin{multline}
  %\frac{2}{3\vmag\nue} 
  %\left(\left(\sqrt{3}\vmag\frac{\nue}{2}\right)^2 - \Ez^2\right)  
  \left(\vmag\frac{\nue}{2} - \frac{2\qe^2\Ez^2}{3\me^2\vmag\nue}\right) 
  \pdv{\fonez}{\vmag} 
  =
  \frac{2\qe\Ez}{3\me\nue}\pdv{\fonez}{z}  
  + \frac{4\pi\qe\Ez}{3\me}\pdv{\fM}{\vmag} \\
  + \frac{\vmag}{3}\pdv{\fzero}{z} 
  + \left(\frac{4\qe^2\Ez^2}{3\me^2\vmag^2\nue}
  + \left(\nuei + \frac{\nue}{2}\right) \right)\fonez .
  \label{eq:AP1_model_1D}
\end{multline}
It is convenient to write the~bracket on the~left hand side of 
\eqref{eq:AP1_model_1D} as
$\frac{2}{3\vmag\nue} 
\left(\left(\sqrt{3}\vmag\frac{\nue}{2}\right)^2 
- \frac{\qe^2}{\me^2}\Ez^2\right)$
from where it is clear that the~bracket is negative if 
$\sqrt{3}\vmag\frac{\nue}{2} < \frac{\qe}{\me}|\E|$, 
i.e. there is a~velocity limit for a~given magnitude $|\E|$, 
when the~collisions are no more fully dominant and the~electric field 
introduces a~comparable effect to the~collision friction in 
the~electron transport.

It can be shown, that the~last term on the~right hand side of 
\eqref{eq:AP1_model_1D} is dominant and the~solution behaves as 
\begin{equation}
  \Delta \fone \sim \exp\left(\frac{\frac{4\qe^2\Ez^2}{3\me^2\vmag^2\nue}
  + \left(\nuei + \frac{\nue}{2}\right)}
  {\vmag\frac{\nue}{2} - \frac{2\qe^2\Ez^2}{3\me^2\vmag\nue}}\, 
  \Delta\vmag\right) ,
  \label{eq:f1z_behavior}
\end{equation}
where $\Delta \vmag < 0$ represents a~velocity step of the~implicit Euler
numerical integration of decelerating electrons.
However, \eqref{eq:f1z_behavior} exhibits an~exponential growth 
for velocities above the~friction limit (bracket on the~left hand side of 
\eqref{eq:AP1_model_1D})
\begin{equation}
  \vmag_{lim}  = \sqrt{\frac{\sqrt{3}\Gamma\me}{2\qe}\frac{n_e}{|\E|}} ,
  \label{app_eq:v_limit}
\end{equation}
which makes the~problem to be ill-posed.

In order to provide a~stable model, we introduce a~reduced electric field
to be acting as the~accelerating force of electrons
\begin{equation}
  |\E_{red}| = \sqrt{3} \vmag\frac{\me}{\qe}\frac{\nue}{2} ,
  \label{eq:Elimit}
\end{equation}
ensuring that the~bracket on the~left hand side of \eqref{eq:AP1_model_1D}
remains positive. We define a~quantity $\eta_{red} = \frac{|\E_{red}|}{|\E|}$.
Then, the~AP1 model \eqref{eq:AP1f0}, \eqref{eq:AP1f1} can be formulated 
as well posed 
\begin{eqnarray}
  \vmag\frac{\nue}{2}\pdv{}{\vmag}\left(\fzero - \fM \right) &=&
  \frac{\vmag}{3}\nabla\cdot\fone + \frac{\qe}{\me}\frac{\E}{3}\cdot
  \nonumber \\
  &&\left(
  \eta_{red}\pdv{\fone}{\vmag} + \frac{2(2-\eta_{red})}{\vmag}\fone\right) , 
  \nonumber \\
  \label{eq:AP1f0_app}\\
  \vmag\frac{\nue}{2}\pdv{\fone}{\vmag}
  - \nuscat\fone &=& 
  \vmag\nabla\fzero + 
  \frac{\qe\eta_{red}}{\me}\E\pdv{\fzero}{\vmag} 
  +\frac{\qe\B}{\me c}\vect{\times} \fone
  ,
  \nonumber \\
  \label{eq:AP1f1_app}
\end{eqnarray}
while introducing the~reduction factor of the~accelerating electric field
and the~compensation of the~electric field effect via its angular term.  

\begin{comment} % delta f0.
Nevertheless, before doing so,
we introduce a~slightly different approximation to the~electron distribution 
function as
\begin{equation}
  \tilde{f} = \frac{4\pi \fM + \dafzero}{4\pi} + \frac{3}{4\pi}\vn\cdot\fone .
  \label{eq:P1_OOE}
\end{equation}
where $\dafzero$ represents the~departure of isotropic part from 
the~Maxwell-Boltzmann equilibrium distribution $\fM$. 
%which we keep 
%intentionally in the~distribution function approximation.
Then, the~stable AP1 model reads
\begin{eqnarray}
  \vmag \frac{\nue}{2}\pdv{\dafzero}{\vmag} &=&
  \vmag\nabla\cdot\fone 
  + \frac{\qe}{\me}\E\cdot\left(\omega_{red} \pdv{\fone}{\vmag} 
  + \frac{2}{\vmag}\fone\right) , 
  \label{eq:AP1f0_stable}\\
  \vmag\frac{\nue}{2}\pdv{\fone}{\vmag} 
  &=& \tnuscat\fone 
  + \frac{\vmag}{3}\nabla\left(4\pi\fM + \dafzero\right)
  \nonumber \\
  && 
  + \frac{\qe\E}{3\me}\left(4\pi \pdv{\fM}{\vmag} 
  + \omega_{red} \pdv{\dafzero}{\vmag} 
  \right) ,
  \label{eq:AP1f1_stable}
\end{eqnarray}
where $\tnuscat = \nuei + \nuscat^E + \frac{\nue}{2}$.

The~reason for keeping $\fM$ in the~distribution function approximation
\eqref{eq:P1_OOE} can be seen in the~last term on the~right hand side of 
\eqref{eq:AP1f1_stable}, which provides the~effect of electric field on
directional quantities as current or heat flux. In principle, the~explicit use
of $\fM$ ensures the~proper effect of $\E$ if $\dafzero \ll \fM$, i.e.
no matter what the~reduction $\omega_{red}$ is. Apart from its stability,
it also exhibits much better convergence of the~electric field, which is now
given by the~zero current condition of \eqref{eq:AP1f1_stable} as
\begin{equation}
  \E =
  \frac{\intv \left(\frac{\nue}{2\tnuscat}\vmag^2\pdv{\fone}{\vmag}
  - \frac{\vmag^2}{3\tnuscat}
  \nabla\left(4\pi\fM + \dafzero\right)\right) \vmag^2\, \dI\vmag}
  {\frac{\qe}{\me}\intv \frac{\vmag}
  {3\tnuscat}
  \left(4\pi\pdv{\fM}{\vmag} + \omega_{red} \pdv{\dafzero}{\vmag}\right)
  \vmag^2\, \dI\vmag} .
  \label{eq:AP1_Efield_stable}
\end{equation}
\end{comment} % delta f0.

\begin{comment} % CALDER kinetics.
\section{Calder kinetics}
\label{app:CalderKinetics}
and is written (it would be better to write Eq. (2) in its conservative forme and its non relativistic kernel U):
\begin{widetext}
\begin{eqnarray}
&&\frac{\partial f_\alpha}{\partial t}+\mathbf{v}\cdot\nabla_{\mathbf{x}}f_\alpha+q_\alpha\left(\mathbf{E}+\mathbf{v}\vect{\times}\mathbf{B}\right)\nabla_{\mathbf{p}}f_\alpha=C_{LBB}(f_\alpha,f_\alpha)+\sum_\beta C_{LBB}(f_\alpha,f_\beta),\\
&&C_{LBB}(f_\alpha,f_\beta)=-\frac{\partial}{\partial \mathbf{p}}\cdot\frac{\Gamma_{\alpha\beta}}{2}\left[\int \mathbf{U}(\mathbf{p},\mathbf{p}^\prime)\cdot(f_\alpha\nabla_{\mathbf{p}^\prime}f_\beta^\prime-f_\beta^\prime\nabla_{\mathbf{p}}f_\alpha)\right]d^3\mathbf{p}^\prime,\\
&&\mathbf{U}(\mathbf{p},\mathbf{p}^\prime)=\frac{r^2/\gamma\gamma^\prime}{(r^2-1)^{3/2}}\left[(r^2-1)\mathbf{I}-\mathbf{p}\otimes\mathbf{p}-\mathbf{p}^\prime\otimes\mathbf{p}^\prime+r(\mathbf{p}\otimes\mathbf{p}^\prime+\mathbf{p}^\prime\otimes\mathbf{p})\right],
\end{eqnarray}
\end{widetext}
with $\gamma=\sqrt{1+\mathbf{p}^2}$, $\gamma^\prime=\sqrt{1+\mathbf{p}^{\prime 2}}$ and $r=\gamma\gamma^\prime-\mathbf{p}\cdot\mathbf{p}^\prime$. The momemtum $\mathbf{p}_\alpha$ ($\mathbf{p}_\beta$) is normalized to $m_\alpha c$ (resp. $m_\beta c$). Obviously, this collision operator tends to Eq. (2) in the non-relativistic limit.
\end{comment} % CALDER kinetics.

