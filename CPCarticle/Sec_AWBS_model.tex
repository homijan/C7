\section{The~AWBS kinetic model}
\label{sec:AWBSmodel}

The~electrons in plasma can be modeled by the~deterministic model of charged
particles
\begin{equation}
  \pdv{\ft}{t} + \vv\cdot\gx \ft + \tE \cdot\gv \ft = C_{ee}(f) + C_{ei}(f) ,
  \label{eq:kinetic_equation}
\end{equation}
where $\ft(t, \vect{x}, \vect{v})$ represents the~density function of electrons
at time $t$, spatial point $\vect{x}$, and velocity $\vv$, and 
$\tE = \frac{\qe}{\me}\E$ is the~existing electric field in plasma.

The~generally accepted form of the~electron-electron collision operator 
$C_{ee}$ is the~Fokker-Planck form published by Landau \cite{Landau_1936}
\begin{equation}
  C_{FP}(\ft) =
  \Gamma\int \gv\gv(\vv - \vvb) \cdot \left(
  \ft\, \gvb \ft - \ft\, \gv \ft \right)\, \dI\vvb ,
  \label{eq:LFP_model}
\end{equation}
where $\Gamma = \frac{\qe^4\lnc}{4\pi\epsilon^2\me^2}$ and 
$\lnc$ is the~Coulomb logarithm.
In principal, the~electron-ion collision operator $C_{ei}$ could be expressed
in the~form similar to \eqref{eq:LFP_model}, but since ions are considered 
to be motionless compared to electrons, the~scattering operator, i.e.
no change in the~velocity magnitude, expressed in spherical coordinates
is widely accepted
\begin{equation}
  C_{ei}(\ft) = \frac{\nuei}{2}
  \left(\pdv{}{\mu}\left((1 - \mu^2)\pdv{\ft}{\mu}\right)
  + \frac{1}{\sin^2(\phi)}\frac{\partial^2 \ft}{\partial\theta^2} \right) ,
  \label{eq:ei_scattering}
\end{equation}
where $\mu = \cos(\phi)$, $\phi$ and $\theta$ are the~polar and azimuthal 
angles, and $\nuei = \frac{\Zbar n_e \Gamma}{\vmag^3}$ is the~electron-ion
collision frequency.

Fish introduced an~alternative form of $C_{ee}$ in \cite{Fish_RMP1987} 
referred to as high-velocity limit electron collision operator
\begin{multline}
  C_{H}(\ft) = \vmag \nue \pdv{}{\vmag}\left(\ft + 
  \frac{\vth^2}{\vmag}\pdv{f}{\vmag}\right) \\
  + \frac{\nue}{2}\left( 1 - \frac{\vth^2}{2\vmag^2}\right) 
  \left(\pdv{}{\mu}\left((1 - \mu^2)\pdv{f}{\mu}\right)
  + \frac{1}{\sin^2(\phi)}\frac{\partial^2f}{\partial\theta^2} \right)
  , \label{eq:HighVelocity_model}
\end{multline}
where $\nue = \frac{n_e \Gamma}{\vmag^3}$ is the~electron-electron collision 
frequency and $\vth = \sqrt{\frac{\kB T_e}{\me}}$ is the~electron thermal 
velocity.
The~linear form of $C_{H}$ arises from an~assumption that the~fast electrons 
predominantly interact with the~thermal (slow) electrons, 
which simplifies importantly the~nonlinear form \eqref{eq:LFP_model}.

The~aim of this work is to use a~yet simpler form of 
the~electron-electron collision operator, i.e. the~AWBS formulation 
\cite{AWBS_PRL1986}, where we propose the~following form
\begin{multline}
  C_{AWBS}(\ft) = \vmag \frac{\nue}{2}\pdv{}{\vmag}\left(\ft - \fM\right) \\
  + \frac{\nuei + \frac{\nue}{2}}{2} 
  \left(\pdv{}{\mu}\left((1 - \mu^2)\pdv{f}{\mu}\right)
  + \frac{1}{\sin^2(\phi)}\frac{\partial^2f}{\partial\theta^2} \right)
  , \label{eq:AWBS_model}
\end{multline}
where $\fM$ is the~Maxwell-Boltzmann equilibrium distribution.
$C_{AWBS}$ represents the~complete $C_{ee} + C_{ei}$ collision operator in 
\eqref{eq:kinetic_equation}.

The~complete form of collision operator \eqref{eq:AWBS_model} 
was previously introduced in \cite{Sorbo_2015, Sorbo_2016}, nevertheless, 
we intentionally use a~half of the~electron-electron collisional frequency
modification $\frac{\nue}{2}$, because this formulation provides very 
promising results compared to the~full FP operator as emphasized in Section 
\ref{sec:DiffusiveKinetics}.

The~Maxwell-Boltzmann averaged e-e scattering in 
\eqref{eq:HighVelocity_model} can be approximated as 
$\nue \int\left(1 - \frac{\vth^2}{2\vmag^2}\right)\fM 4\pi\vmag^2\, \dI\vmag = 
\frac{\nue}{2}$.
